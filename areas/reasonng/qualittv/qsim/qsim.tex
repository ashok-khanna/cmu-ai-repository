% This is a DRAFT document describing QSIM and providing further references.

\documentstyle[11pt]{article}
\pagestyle{myheadings}
\markright{QSIM \hfill }
\renewcommand{\baselinestretch}{1}
%\setlength{\oddsidemargin}{0in}
\setlength{\textwidth}{6in}
\setlength{\topmargin}{0in}
\setlength{\headheight}{.25in}
%\setlength{\textheight}{8.5in}

\newcommand{\Astrom}{\AA{}str\"{o}m}
\newcommand{\Fouche}{Fouch\'{e}}

\title{QSIM: A Tool for Modeling and Simulation \\ with Incomplete
Knowledge\thanks{This work has taken place in the Qualitative
Reasoning Group at the Artificial Intelligence Laboratory, The
University of Texas at Austin.  Research of the Qualitative Reasoning
Group is supported in part by NSF grants IRI-8904454, IRI-9017047, and
IRI-9216584, and by NASA grants NCC 2-760 and NAG 9-665.}}

\author{Benjamin Kuipers \\ Department of Computer Sciences  \\
University of Texas at Austin \\ Austin, Texas 78712 USA  \\
\mbox{\tt kuipers@cs.utexas.edu}}

\date{July 27, 1993}

\begin{document}
\maketitle

The world is infinite, continuous, and continually changing over time.
Human knowledge and human inference abilities are finite, apparently
symbolic, and therefore incomplete.  Nonetheless, people normally
reason quite effectively about the physical world.

Models of particular systems or mechanisms play an important role in
this capability.  In service of a task such as diagnosis or design,
simulation predicts the behaviors that follow from a particular model.
In diagnosis or explanation, these predictions include testable
consequences of a diagnostic hypothesis.  In design, these predictions
make explicit the consequences of a set of design choices.

A qualitative differential equation (QDE) model is a symbolic
description expressing a state of incomplete knowledge of the
continuous world, and is thus an abstraction of an infinite set of
ordinary differential equations models.  Qualitative simulation
predicts the set of possible behaviors consistent with a QDE model and
an initial state.

We have developed a substantial foundation of tools for model-based
reasoning with incomplete knowledge:  QSIM and its extensions for
qualitative simulation; Q2, Q3 and their successors for quantitative
reasoning on a qualitative framework; and the CC and QPC model compilers
for building QSIM QDE models starting from different ontological
assumptions.

The QSIM representation for qualitative differential equations (QDEs)
and qualitative behaviors was originally motivated by protocol analysis
studies of expert explanations.  A QDE represents a set of ODEs
consistent with natural states of human incomplete knowledge of a
physical mechanism.  Qualitative simulation can be guaranteed to produce
a set of qualitative behavior descriptions covering all possible
behaviors of all ODEs covered by the QDE [Kuipers, 1986].

The subsequent evolution of QSIM has been dominated by the mathematical
problems of retaining this guarantee while producing a tractable set of
predictions.  A variety of methods now exist for applying a deeper
analysis, changing the level of description, or appealing to carefully
chosen additional assumptions, to obtain tractable predictions from a
wide range of useful models.

Quantitative information can be used to annotate qualitative behaviors,
preserving the coverage guarantee while providing stronger predictions.
Quantitative information may be expressed as bounds on landmarks and
other symbolic elements of the qualitative description [Kuipers \&
Berleant, 1988], by adaptively inserting new time-points to improve
the resolution of the description and converge to a numerical function
[Berleant \& Kuipers, 1992], and by deriving envelopes bounding the
possible trajectories of the system [Kay \& Kuipers, 1993].
Observations are interpreted by unifying quantitative measurements
against the qualitative behavior prediction, yielding either a stronger
prediction or a contradiction.  As quantitative uncertainty in the QDE
and initial state decrease to zero, the resulting behavioral description
converges to the true quantitative behavior, though computational costs
can still be high with current methods.

We have developed two model-compilers for QDE models: CC, which takes
the component-connection view of a mechanism [Franke, 1989], and QPC,
which implements an extended version of Qualitative Process Theory
[Crawford, Farquhar \& Kuipers, 1990].  Other model-compilers for
QDEs, e.g. using bond graphs or compartmental models, have been
developed elsewhere.  We hope to use these model-building tools to
support deeper investigation of modeling assumptions and view
selection.

There are several inference schemes built on the set of all possible
behaviors that are particularly well-suited to reliable model-based
reasoning for diagnosis and design.  For design, desirable and
undesirable behaviors can be identified, and additional constraints
inferred to guarantee or prevent those behaviors [Franke, 1991;
Kuipers \& \Astrom, to appear].

For monitoring and diagnosis, plausible hypotheses are unified against
observations to strengthen or refute the predicted behaviors.  In
MIMIC [Dvorak \& Kuipers, 1989], multiple hypothesized models of the
system are tracked in parallel in order to reduce the ``missing
model'' problem [Perrow, 1985].  Each model begins as a qualitative
model, and is unified with {\em a priori} quantitative knowledge and
with the stream of incoming observational data.  When the model/data
unification yields a contradiction, the model is refuted.  When there
is no contradiction, the predictions of the model are progressively
strengthened, for use in procedure planning and differential
diagnosis.  Only under a qualitative level of description can a finite
set of models guarantee the complete coverage necessary for this
performance.



\section*{Details}

 \begin{itemize}

  \item The QSIM software is experimental, and is provided to
researchers as a professional courtesy, with no warranty of any kind.
Any commercial rights are retained.

  \item Installation instructions, release notes and other
documentation is provided in the documentation directory that is
created once QSIM is untarred.  The release notes provide information
regarding the platforms that are supported.

  \item The distribution includes code for QSIM and a number of its
extensions, documented and undocumented.  A draft manual is included,
but is quite incomplete.  The examples illustrate many of the
available features.  Exploring the code should be worthwhile.

  \item Once QSIM has been loaded in Lisp, evaluate {\tt (qt)} for a
partial menu of examples.

  \item There is a mailing list \mbox{\tt qsim-users@cs.utexas.edu}
for discussing QSIM modeling and simulation issues.  There is a
mailing list \mbox{\tt qphysics@cs.uwashington.edu} for qualitative
reasoning issues generally.

  \item Please refer any questions or comments to either
\mbox{\tt clancy@cs.utexas.edu} or \mbox{\tt kuipers@cs.utexas.edu}.

  \item	To obtain any additional software that is not included in this
distribution (such as MIMIC or QPC) or to inquire about other
extensions that are being developed that might be relevent to your
task please contact us.

 \end{itemize}




\section*{References on QSIM and Related Topics}

 \begin{enumerate}\small

  \item D. Berleant \& B. Kuipers.  1992.  Combined qualitative and
numerical simulation with Q3.  In Boi Faltings and Peter
Struss (Eds.), {\em Recent Advances in Qualitative Physics}, MIT Press,
1992.

  \item
 J. M. Crawford, A. Farquhar, B. J. Kuipers.  1990.
QPC:  a compiler from physical models into qualitative differential equations.
{\it Proceedings of the National Conference on Artificial Intelligence (AAAI-90)},
AAAI/MIT Press, 1990.  Revised version in 
Boi Faltings and Peter Struss (Eds.), {\em Recent Advances
in Qualitative Physics}, MIT Press, 1992.

  \item
D. Dvorak and B. Kuipers.  1989.  Model-based monitoring of dynamic systems.
In {\it Proceedings of the Eleventh International Joint Conference on 
Artificial Intelligence (IJCAI-89)}.  Los Altos, CA:  Morgan Kaufman.

  \item
 D. Dvorak \& B. Kuipers.  1991.
Process monitoring and diagnosis:  a model-based approach.
{\it IEEE EXPERT} {\bf 6}(3):  67-74, June 1991.

  \item Pierre \Fouche\ \& Benjamin Kuipers.  1992.
Reasoning about energy in qualitative simulation.
{\it IEEE Transactions on Systems, Man, and Cybernetics} {\bf 22}(1):  47-63.

  \item
D. W. Franke.  1989.  Representing and acquiring teleological descriptions.
Model-Based Reasoning Workshop, IJCAI-89, Detroit, Michigan, August 1989.

  \item
D. W. Franke.  1991.  Deriving and using descriptions of purpose.
{\it IEEE Expert}, April 1991, pp. 41-47.

  \item
D. W. Franke and D. Dvorak.  1989.  Component-connection models.
Model-Based Reasoning Workshop, IJCAI-89, Detroit, Michigan, August 1989.

  \item 
 H. Kay \& B. Kuipers.  1993.  Numerical behavior envelopes for qualitative simulation.
{\it Proceedings of the National Conference on Artificial Intelligence (AAAI-93)},
AAAI/MIT Press, 1993.

  \item
B. J. Kuipers.  1984.  Commonsense reasoning about causality:
deriving behavior from structure.
{\it Artificial Intelligence} {\bf 24}: 169 - 204.

  \item
B. J. Kuipers.  1986.  Qualitative simulation.  
{\it Artificial Intelligence} {\bf 29}: 289 - 338.

  \item
B. Kuipers.  1987.
Abstraction by time-scale in qualitative simulation.
{\it Proceedings of the National Conference on Artificial
Intelligence (AAAI-87)}.  Los Altos, CA: Morgan Kaufman.

  \item
B. Kuipers.  1989.  Qualitative reasoning:  modeling and simulation
with incomplete knowledge.  {\it Automatica} {\bf 25}: 571-585.

  \item
B. J. Kuipers.  Reasoning with qualitative models.  1993.
{\em Artificial Intelligence} {\bf 59}: 125-132.

  \item
B. J. Kuipers.  Qualitative simulation:  then and now.  1993.
{\em Artificial Intelligence} {\bf 59}: 133-140.

  \item
B. J. Kuipers and K. \Astrom.  The composition and validation of heterogeneous control laws.
{\em Automatica}, in press.

  \item
B. Kuipers and D. Berleant.  1988.  Using incomplete quantitative knowledge
in qualitative reasoning.  In {\it Proceedings of the
National Conference on Artificial Intelligence (AAAI-88)}.
Los Altos, CA:  Morgan Kaufman.

  \item
B. Kuipers and C. Chiu.  1987.
Taming intractible branching in qualitative simulation.
{\it Proceedings of the Tenth International Joint Conference
on Artificial Intelligence (IJCAI-87)}.  Los Altos, CA: Morgan Kaufman.

  \item
 B. J. Kuipers, C. Chiu, D. T. Dalle Molle \& D. R. Throop.  1991.
Higher-order derivative constraints in qualitative simulation.
{\it Artificial Intelligence} {\bf 51}:  343-379.

  \item
B. J. Kuipers and J. P. Kassirer.  1984.
Causal reasoning in medicine:  analysis of a protocol.
{\it Cognitive Science} {\bf 8}: 363 - 385.

  \item 
 W. W. Lee \& B. Kuipers.  1993.  A qualitative method to construct phase portraits.
{\it Proceedings of the National Conference on Artificial Intelligence (AAAI-93)},
AAAI/MIT Press, 1993.

  \item
W. W. Lee \& B. Kuipers.  1988.  Non-intersection of trajectories in
qualitative phase space:  a global constraint for qualitative
simulation.  In {\it Proceedings of the National Conference on
Artificial Intelligence (AAAI-88)}.  Los Altos, CA:  Morgan Kaufman.

  \item
Charles Perrow.  1984.  {\it Normal Accidents:  Living With High-Risk Technologies}.
New York:  Basic Books.

 \end{enumerate}



\end{document}
