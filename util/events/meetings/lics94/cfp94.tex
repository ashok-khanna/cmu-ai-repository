\documentstyle{article}

\topmargin-0.75in
%\textheight10in

%\oddsidemargin 0pt
%\evensidemargin \oddsidemargin
%\marginparwidth 0.5in

%\textwidth 6.5in

%\parindent0pt
%\parskip.5\baselineskip

%\pagestyle{empty}

\marginparwidth 0pt \oddsidemargin  0pt \evensidemargin  0pt
\marginparsep 0pt
%\topmargin -.75in
\textwidth 6.5in \textheight 10.0in\parskip 6pt \parindent 0pt
\renewcommand{\i}[1]{{\it #1 \/}}
\begin{document}
\thispagestyle{empty}

%\begin{flushright}
%\footnotesize August 29, 1991
%\end{flushright}
\begin{center}
{\bf CALL FOR PAPERS}\\[2ex]
{\large Ninth Annual IEEE Symposium on }\\[2ex]
{\Large\bf LOGIC IN COMPUTER SCIENCE }\\[2ex]
{\large\it July 4--7, 1994, Paris, France}
\end{center}
\vspace{.2in}
\small
\begin{minipage}[t]{2.00in}% first column
%\baselineskip10pt
\parskip 4pt

{\bf Program Chair:} \\[1mm]
Samson Abramsky\\
Attn: LICS \\
Department of Computing\\
Imperial College\\ % of Science, Technology and Medicine\\
180 Queen's Gate\\
London SW7 2BZ\\
United Kingdom\\
{\tt sa@doc.ic.ac.uk}\\
Phone: (44) 71-589-5111 ext. 5005\\
Fax: (44) 71-581-8024 \\

{\bf Conference Co-Chairs:} \\[1mm]
G\'{e}rard Huet\\
INRIA Rocquencourt\\
B.P.~105--78153\\
Le Chesnay {\sc cedex}, France\\
{\tt huet@margaux.inria.fr} \\[2mm]
Jean-Pierre Jouannaud\\
CNRS and LRI\\
Bat. 490, Universit\'{e} de Paris Sud\\
91405 Orsay {\sc cedex}, France\\
{\tt jouannaud@margaux.inria.fr} \\

{\bf General Chair:}\\[1mm]
Robert L.\ Constable \\
Department of Computer Science\\
Upson Hall \\
Cornell University \\
Ithaca, NY 14853, USA\\
{\tt rc@cs.cornell.edu}\\

{\bf Organizing Committee:}\\[1mm]
M.~Abadi, S.~Abramsky, S.~Artemov, A.~Borodin,
S.~Buss, E.~Clarke, R.~Constable (Chair), A.~Felty,
U.~Goltz, Y.~Gurevich, S.~Hayashi, D.~Howe, G.~Huet, D.~Johnson,
J.-P.~Jouannaud, D.~Kapur, C.~Kirchner, P.~Kolaitis, D.~Kozen,
D.~Leivant, A.R.~Meyer, D.~Miller, G.~Mints, J.~Mitchell, 
Y.~Moschovakis, M.~Okada, P.~Panangaden, A.~Pitts,
G.~Plotkin, J.~Remmel, S.~Ronchi della Rocca, G.~Rozenberg, 
A.~Scedrov, D.~Scott, J.~Tiuryn, M.Y.~Vardi
\end{minipage}
\hskip .25 in
\begin{minipage}[t]{4.5in}% second column
\parskip 4pt

The {\bf LICS} symposia aim to attract high quality original papers
covering theoretical and practical issues in computer science that
relate to logic in a broad sense, including algebraic, categorical and
topological approaches.

Suggested, but not exclusive, topics of interest include:
{\em abstract data types, automated deduction, concurrency, constructive
mathematics, data base theory, finite model theory, knowledge
representation, lambda and combinatory calculi, logical aspects of
computational complexity, logics in artificial intelligence, logic
programming, modal and temporal logics, program logic and semantics,
rewrite rules, logical aspects of symbolic computing, problem solving
environments, software specification, type systems, verification.}

{\bf Paper submission}: 10 hard copies of a detailed abstract (not a
full paper) and 20 additional copies of the cover page should be
{\bf received} by {\bf December 13, 1993} by the {\bf program chair}.
{\sl This is a firm deadline: late submissions will not be considered.}
Authors without access to duplication facilities may submit a single
copy of each.  Authors will be notified of acceptance
by February 21, 1994.  Accepted papers (in a specified proceedings format)
will be due April 15, 1994.

The cover page of the submission should include the title, authors, a
brief synopsis, and the corresponding author's name, address, phone
number, fax number, and e-mail address, when available.  Abstracts must be
in English, clearly written, and provide sufficient detail to allow
the program committee to assess the merits of the paper.  References
and comparisons with related work should be included.  It is recommended
that each submission begin with a succinct statement of the issues,
a summary of the main results, and a brief explanation of their
significance and relevance to the conference, all phrased for the
non-specialist.  Technical development of the work, directed to the
specialist, should follow.  While abstracts of fewer than 1500 words are
rarely adequate, the entire abstract should not exceed
10 typed pages, with roughly 35 lines per page.  If the authors believe
that more details are essential to substantiate their main results,
they may place additional details in a clearly marked appendix.
Submissions departing
significantly from these guidelines run a high risk of rejection.

The results in the abstract must be unpublished, and not submitted
for publication elsewhere, including proceedings of other symposia or
workshops.  All authors of accepted papers will be expected to sign
copyright release forms, and one author of each accepted paper will be
expected to present the paper at the conference.

LICS'94 is sponsored by the IEEE Technical Committee on
Mathematical Foundations of Computing, in cooperation with the
Association for Symbolic Logic and the European Association for
Theoretical Computer Science.  The cooperation of the ACM is anticipated.
The symposium is organized by INRIA, and hosted by 
the Conservatoire National des Arts et M\'etiers (CNAM) as part of 
its bicentennial.
Sponsorship by the CNRS and Universit\'e d'Orsay is expected.

For further announcements, contact the
{\bf Publicity Co-chairs:}
Amy Felty and Douglas Howe
AT\&T Bell Laboratories
600 Mountain Avenue,
Murray Hill, NJ 07974.
Email: {\tt felty@research.att.com, howe@research.att.com}
\end{minipage}
\end{document}
