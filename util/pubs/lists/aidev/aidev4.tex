\documentstyle[a4]{article}
\pagestyle{myheadings}
\markboth{Author:kk}{Doc.No. AID/newsletter/4}
\setlength{\textheight}{9in}
\setlength{\topmargin}{0in}
\setlength{\headheight}{.3in}
\setlength{\headsep}{.5in}
\parindent=0pt
\parskip= .35\baselineskip plus .0833333\baselineskip minus .0833333\baselineskip
\begin{document}
Artificial Intelligence for Development\\
Document No: AID/newsletter/4\\
Last Modified and by whom: 29/8/91 kk \\
Distributed on: 29/8/91\\
\LARGE
\begin{center} Artificial Intelligence for Development\\
Newsletter Number 4\\August 1991\\

\end{center}
\normalsize
\section*{Contents}
\begin{enumerate}
\item Knowledge Based Computer systems in India
\item VITA news
\item SatelLife
\item Conferences
\item Groups, bulletin boards and mailing lists
\item Contact Personnel
\end{enumerate} 

\section{Knowledge Based Computer Systems in India}

{\ S. Ramani and K. S. R. Anjaneyulu\\} 
{National Centre for Software Technology\\
Gulmohar Cross Road No 9, Juhu, Bombay 400 049, India\\ 
Email: ramani@ncst.ernet.in, anji@ncst.ernet.in}

India has a well-known tradition of study of thought, logic 
and language. There has been pioneering work by Indians in the areas of
artificial intelligence and pattern recognition. In more recent years, 
the launching of the Knowledge Based Computer Systems Project by the Department 
of Electronics, Government of India, has contributed a lot to the 
development of national capability in this field. This Project, 
assisted by the United Nations Development Programme, 
has stimulated a lot of significant 
research in many areas.  There are a number of other centres active
in AI R \& D, in addition to the centres involved in this
project.  The paper presents a list of centres where 
major work is being carried out and the areas of their interest.
A bibliography of over seventy five selected publications is included. 

\subsection{Early Work}
India has a well-known tradition of study of thought, language and logic.
Panini's elucidation of the structure of Sanskrit was one of the 
landmarks in the progress in this field. So, it is not
surprising to see the tremendous attraction that the field of 
artificial intelligence has for thousands of Indians, within the 
country as well as without. 

An early paper by an Indian, which had a world-wide impact in the field,
was on the use of syntactic methods in pattern recognition [Narasimhan, 1963]. 
The Computer Group at the Tata Institute of Fundamental Research has 
had a long tradition of 
work in the areas of natural language [Narasimhan, 1981] and
speech recognition. There has been continuing work on pattern 
recognition and computer vision 
at the Indian Statistical 
Institute spanning a few decades.

\subsection{Current Activities}
There are a few areas in AI in which researchers in India have
been working and have achieved some progress. These areas
include: 
AI in Industry (Scheduling, Productivity Management, etc), 
Expert Systems, Intelligent Tutoring Systems, 
Machine Translation, 
Natural Language Understanding,
Pattern Recognition, 
Search and Speech Recognition. 

In 1986, the Government of India initiated the Knowledge Based
Computer Systems (KBCS) Project, with assistance from
the United Nations Development Programme (UNDP).
This five-year project is a major research initiative 
in the area of Artificial Intelligence. UNDP's support 
has been instrumental in enabling the visit of eminent 
AI workers from major laboratories around the world to visit 
Indian KBCS Centres. It has also enabled Indian KBCS workers, 
particularly those in the early stages of their careers to get trained
in major laboratories abroad. 

The Ministry of
Human Resource Development has started an Intelligent
Systems Project with nodes all over India. 
Indian Institute of Management, Calcutta and the Indian Institute
of Technology, Kharagpur have made significant contributions
in the field of search.  There is 
a project on Machine Translation at the Indian Institute of
Technology, Kanpur.  The Central University in Hyderabad offers a
Masters Degree in AI and Robotics. The National Centre for Software 
Technology, Bombay offers a post-graduate diploma programme in 
KBCS.

For a partial list of institutions active in AI in India, 
see Appendix A.

\subsubsection*{Work at KBCS Project Centres}
The Proceedings of KBCS '89 [Ramani et al, 1989] and KBCS-90 [Bhatkar and Rege, 1990] have reports from the
coordinators of the KBCS Project Centres around the country.  While these
reports do not constitute a complete report of AI in India,
they are good indicators of the current state of AI in the country.
The Centres of the KBCS Project and their areas of work are
listed below:
\begin{itemize}
\item Architecture for Parallel Computer Systems: Indian Institute of
Science, Bangalore 
 \item Speech Recognition and Neural Networks:
Tata Institute of Fundamental Research, Bombay
 \item Pattern Recognition and Computer Vision: 
Indian Statistical Institute, Calcutta
 \item Expert Systems: Indian Institute of Technology, Madras
 \item Expert System Applications for Government:
Department of Electronics, Government of India, New Delhi
 \item Natural Language, Educational Applications, and KBCS in
Planning and Scheduling, Logic Programming: National Centre for
Software Technology, Bombay 
 \end{itemize}

\subsection{Comments on Areas of Work}
{\bf Computer Aided Design using KBCS Techniques}
As there is a large, well established, engineering industry in India, AI
applications are bound to flourish.  However, no major
development in this direction has so far taken place. 

{\bf Computer Vision:}
With a long-standing tradition of research in this field, India is
bound to make significant progress in this area.

{\bf Educational Applications:}
With the enormous educational challenge India is facing, 
this area is bound to gather momentum.  Multi-media
techniques and Intelligent Tutoring System (ITS) techniques will come together
to create impressive instructional systems.  New
peripherals such as CD-ROM drives are likely to make
these applications far more cost-effective than
they are today.
One problem to be faced, however, is that of cost.  Progress in educational
applications is likely to be slow because the users' ability
to pay for these applications is very poor.

{\bf Expert Systems:}
Expert systems have attracted wide attention.  Holding promise
of white-collar robots, this area is bristling with activity.
Since the current paradigm for creation of Expert Systems has
serious limitations, there is room for significant R \& D relating
to the technology itself.

{\bf Natural Language and Speech Recognition:}
To the extent language is central to intelligent behaviour,
natural language understanding will occupy a central place
in AI research.  Having fourteen official languages, India is bound to 
be very interested in machine translation!  Knowledge Based Machine
Translation is likely to be a reality by the turn of the Century.

{\bf Knowledge Based Planning and Scheduling:}
This is a very practical area of considerable economic significance.
A few projects have been completed and many more are being
undertaken.  This area is bound to grow and contribute to the general
credibility of AI among those who place importance on practical applications.

{\bf Psychology and Linguistics:}
An important weakness of Indian AI is that there is not adequate research
in the country in the related sciences of Psychology and
Linguistics.  This will have to be rectified soon.

\subsection{Conclusion}
There is a fast-growing R \& D activity in KBCS in India. There are
many potential and relevant applications. There are a number of 
educational programmes at the post-graduate level. There have been a 
significant number of international publications. The KBCS Project of 
the Government of India has provided a major stimulus to the growth 
of R \& D in this field in India. 

\subsection*{Partial List of Centres in India Working in AI}
A partial list of centres in India is given below,
along with areas they are working on.

{\bf Expert Systems:} Dept of Electronics, Delhi.
     Indian Institute of Technology, Madras.
     Jadavpur University, Calcutta.\\
{\bf Image Processing and Pattern Recognition:}
     Indian Institute of Technology, Bombay.
     Indian Institute of Technology, Madras.
     Indian Statistical Institute, Calcutta.\\
{\bf Intelligent Systems in Management and Manufacturing:}
     Indian Institute of Technology, Bombay.
     Indian Institutes of Management.\\
{\bf Intelligent Systems in Medicine:}
     Indian Institute of Technology, Bombay.
     Indian Institute of Technology, Madras.\\
{\bf Intelligent Tutoring Systems:}
     National Centre for Software Technology, Bombay.\\
{\bf Logic Programming:}
     Indian Institute of Technology, Bombay.
     Indian Institute of Technology, Kanpur.
     Institute of Mathematical Sciences, Madras.
     National Centre for Software Technology, Bombay. Tata Institute
of Fundamental Research, Bombay. \\
{\bf Machine Translation:}
     Indian Institute of Technology, Kanpur.
     National Centre for Software Technology, Bombay.
     Tamil University, Thanjavur.\\
{\bf Modelling Language Behaviour:}
     Tata Institute of Fundamental Research, Bombay.\\
{\bf Natural Language Understanding:}
     Central University, Hyderabad.
     Indian Institute of Science, Bangalore.
     Indian Institute of Technology, Bombay.
     Indian Institute of Technology, Kanpur.
     National Centre for Software Technology, Bombay.
     Tata Institute of Fundamental Research, Bombay.\\
{\bf Parallel Processing:}
     Centre for Development of Advanced Computing, Pune .
     Indian Institute of Science, Bangalore.\\
{\bf Search:}
     Indian Institute of Management, Calcutta .
     Indian Institute of Technology, Kharagpur. \\
{\bf Understanding and Synthesising Speech:}
     Indian Institute of Technology, Madras.
     Tata Institute of Fundamental Research, Bombay.\\


\section{VITA news: DevelopNet News, July 1991} 
  
DevelopNet News is published by Volunteers in Technical Assistance
(VITA) in Arlington, Virginia, USA.
See below under `Groups, Bulletin Boards,
and Mailing Lists' for how to subscribe to it.

\subsection{AFRICA: BETTER DATA COMMUNICATIONS}
 
by Gary Garriott
 
Inexpensive software can now help to solve some of the telephone
problems in Africa and other technologically remote areas.  Making
voice telephone calls from many African countries to Europe or North
America is usually simpler than calling the country "next door." But
it's even easier for people in North America or Europe to call Africa.
In fact, Africa's problems of phone communications have become
critical: they now seriously obstruct the flow of scientific
information into the continent as well as the sharing of information
among African scientists.

 Even in areas that have direct dialing, line quality suffers from
"cross-talk" and other interference during busy periods. Africa's vast
territory extends across many time zones, making it difficult to com-
plete voice calls at convenient times. Finally, the costs of phone calls
can be prohibitive, although calls initiated from the United States to
Africa generally cost about one-third as much as similar calls from
Africa.

 Inexpensive or free "mailer" software can help to send messages and data
files. It runs on personal computers, even laptops, connected to voice-
grade telephone lines. With it, computers in Africa can call each other
satisfactorily, especially during the night when rates are cheaper and
lower demand reduces interference.
 How can Africans take advantage of better international communications
facilities? They can schedule calls from a "polling computer" in Europe
or North America; it will pick up messages from, and relay them to,
other computers in African countries, including those that are next to
each other. Thus, one African country sends a message or data to another
African country through Paris or London. This is more reliable than a
direct-routed call if the foreign city initiates the calls both for
receiving and retransmitting the information. Moreover, it is usually
cheaper than any other form of data transmission and may save foreign
exchange. VITA now uses mailers in this way to communicate with its
projects in Pakistan and the Philippines.
 
Of many available mailers, FrontDoor is highly versatile, but requires
about 1 megabyte of disk space. It can schedule unattended message and
file transfers. Other mailers (Fido, Seadog) are simpler to install, but
much less convenient to use.
 
Until regional and international communications improve, mailers and
microcomputers can use the present telephone systems to meet African
needs without resorting to prohibitively expensive public data networks.
 
Product information: (Seadog) System Enhancements Associates, 21 New
Street, Wayne, New Jersey 07470. Fido and FrontDoor, in noncommercial
versions, are obtained from bulletin board system operators and are free
of cost if used noncommercially.

[The author is VITA's director of informatics.]

\subsection{SWAZILAND: CHILD NUTRITION AND COMPUTER GRAPHICS}
 Better nutrition for young children is a critical issue in most
countries. The governments of many Third-World countries try to improve 
public understanding of child nutrition through educational programs.
The programs are aimed at parents, teachers, children, and the food
industry.
 
Swaziland's educational programs for child nutrition were almost
unmanageably complex. Not only was a large variety of pamphlets, posters, 
flip-charts, and fliers needed, but ethnic and cultural variation from
one region to another required different materials that could not be
conveniently produced from a central location. According to Benedict
Tisa (Westmont, New Jersey, USA), the problem was solved by setting up a
large, computer-based "image bank" in the Ministry of Agriculture in the
capital city of Mbabane, for use by workers in all areas. This was
possible because the ministry already had Macintosh microcomputers. 
 
The Swaziland Project for the Promotion of Young Child Feeding trained
an artist at the ministry to use graphics and scanning programs to cre-
ate the image bank. Images from the bank can easily be made larger or
smaller, or changed in detail to suit any regional or other special
need. Tisa says, "The potentials of this system are just being recog-
nized. . . . High quality, small editions of training materials can be
produced easily and quickly for workshops and seminars. Designer mater-
ials can be made by the field workers themselves since the use of the
program is basically point and choose -- very simple to use."
 
The total cost of equipment and software is between \verb+$+10,000 and \verb+$+12,000.
Although the cost seems high, Tisa states it can be justified if the
project is large and the equipment can be used also for word processing,
data collection, and desktop publishing. In Swaziland, its use has
already spread to another complex effort, the Weaning Project.
 
Source: Benedict Tisa, CompuServe 71650,23 .
 
\subsection{PC GLOBE}
by Donald Ditter and others
 
PC Globe displays maps of the world, regions, groups of countries (e.g.,
ASEAN, developing countries) and 190 individual countries. You can
select country maps that show principal cities, elevations, or other
features. Tables cover demographic, climatic, socioeconomic, cultural,
health, educational, and tourist information. Bar charts (for which you
select the information) compare countries. You can add cities to the
2,000 already in the database and update currency-exchange rates. Coun-
try flags are shown and national anthems played.
 
The manual is well written. The program is easy to install and use and
occupies about 1.6 Mbytes on a hard disk. The displays are uncluttered
and easy to read; they can be sent to your printer or a disk file, or to
many other software programs. The publishers plan to send update notices
to registered users.
 
PC Globe would be highly useful to anyone interested in international
development, and to students, teachers, writers, and editors. Because of
its high quality and reasonable price, it is an outstanding value.
 
PC Globe, version 4.0. PC Globe, Inc., 4700 South McClintock, Tempe,
Arizona 85282 USA (1990). Requirements: IBM(tm) PC/XT/AT/PS2 or com-
patibles, 640 K RAM, floppy drive or hard disk, DOS 2.0+. Supports
Hercules(tm) monochrome, CGA, EGA, VGA. Available in French, German,
Spanish, and other foreign-language versions. List price, \verb+$+59.95; mail
order vendors, \verb+$+39.
 
\section{SatelLife}

Taken from SatelLife update 4/5/91, Julia Royall:

SatelLife has designed an affordable telecommunications system called
Healthnet, through which health professionals, academics, researchers
and activists can have access to each other as well as their
colleagues abroad. Initially SatelLife is focussing on Africa, where
the need is greatest, before expanding to countries in Latin America,
Asia nd the Pacific Rim.

We have chosen `store and forward' satellite technology to achieve
these objectives. A new application of tested technology, it is low
cost, dependable and appropriate for locations where telephone service
is poor or non-existent. SatelLife is presently setting up
demonstration sites in collaboration with universities in Zimbabwe,
Kenya, Zambia, tanzania and Uganda and is working locally witha
variety of organisations to develop the support systems and programs
necessary to ensure maximum utilisation. these include skills training
and technical support for the user community. The ongoing technology
transfer will be accomplished as African users train new users.

HealthNet users will be able to query colleagues, conference others in
the field, order and receive relevant medical literature or request a
dtabase search. As the turnaround time may be only a matter of hours
or days, they will be able to collaborate more effectively - perhaps
for the first time - with others in the profession. We are able to
circumvent the necessity for telephones by using simple ground
stations composed of personal computer and radio (similar to `ham'
radio) which receives signals from and transmits to a low-earth-orbit
satellite about the size of a beach ball.

Although access to information is a compelling part of SatelLife's
mission, of equal importance is providing the means by which people
can communicate with one another when telephone, telex and fax are too
expensive or simply not available. The resulting isolation can be
devastating. 

More information on SatelLife is available from:\\
 Dieter Klein\\
 SatelLife\\
 225 Fifth Street\\
     Cambridge\\
     MA 02142\\
Phon: 617 868 8522\\
FAX: 617 868 2560\\
Email: dklein@wpi.wpi.edu

\section{Conferences}
\subsection{Kenya: Social Implications of Computers in Developing Countries}
\begin{description}
\item[Name:]International Conference on the Social Implications of
Computers in Developing Countries 
\item[Date:] March 23-25 1992
\item[Venue:]International Centre of Insect Physiology and Ecology,
Nairobi, Kenya 
\item[Hosted:] International Federation for Information Processing
(IFIP WG9.4) and Kenya Computer Institute 
\item[Foci:] Appropriate technology, IT indigenisation,
Social/cultural conflicts, human resources development, opportunities
and risks 
\item[Format for papers:] 2 copies, A4, double space, not more than 20pp
\item[Proceedings:] to be published by North Holland
\item[Language:] English
\item[Registration:] KSh 1,800 (approx USdollars 100)
\item[Accommodation:] participants arrange, Kenya computer institute
assists on request
\end{description}
\subsubsection*{Deadlines:}
Abstract (1 page):	15/9/91\\
Full paper:		15/11/91\\
Acceptance:		25/1/92\\
Final manuscript:	25/2/92\\
\subsubsection*{Authors' correspondence:}
and fuller information available from:

		Dr. Mayuri Odedra\\
		16 Avon Court\\
		34 Kenswick Road\\
		London SW15 23U\\

\subsection{WORKSHOP FOR DEVELOPING COUNTRIES INET'91 }
If anyone was at this workshop, part of INET '91, Copenhagen 15-20
June 1991, could they post a summary? Thanks.

\section{Groups, Bulletin Boards and Mailing Lists}
\subsection{I.F.I.P.  working  group  9.4: 
"Social  implications  of  computers  in  developing  countries" }

(See also the note about the Nairobi conferece in March - ed.)

KORPELA@fi.UKU writes:

the International Federation for Information Processing (IFIP) is a
multinational federation of organisations concerned with
information processing.  Currently it has 45 members representing
63 countries.  One of its main objectives is to bring together
computer professionals to stimulate research, development and the
application of information processing in science and human
activity.
    
IFIP has eleven technical committees to which various member
countries nominate the members.  The technical committees cover
a broad range of computing themes like information systems,
education, and communication.  Each TC has one or more working
groups affiliated to it.

The objective of a working group is to bring together a group
of computing professionals interested in a specific area of work. 
Each WG is expected to act as a forum of exchange of ideas by
organising conferences and publishing monographs.
     TC 9, 'Computers and Society', incorporates the working
groups  9.1 Computers and Work,  9.2 Social Accountability, 
9.3 Home Oriented Informatics and Telematics,  and  9.5 Social
Implications of Artificial Intelligence.  The TC 9 working group
WG 9.4, 'Social Implications of Computers in Developing
Countries', was formalized in September 1989. 

The starting point for WG 9.4 was a scientific conference in
New Delhi, India, in November 1988.  The proceedings of the
meeting were published by Elsevier Science Publishers in 1990 as
'S.C. Bhatnagar and N. Bjrn-Andersen (eds.): Information
Technology in Developing Countries'.
    The next conference will be held in Nairobi in March 1992. 
Another major scientific event for the Working Group will be the
12th IFIP World Computer Congress in Madrid, Spain,
7-11 September 1992.  WG 9.4 will organize one session within the
track 'Diminishing the Vulnerability of the Information Society'.

WG 9.4 invites researchers and practitioners of developing
and industrialized countries to join its activities.  As the WG
is still in its making, one of the first tasks will be to
establish a strong membership on all continents, and to organize
interaction amongst the members.
The aims and scope of the Working Group are reprinted below. 
For more information, please get in touch with the officials at
the addresses below.  Requests for membership should be sent to
the secretary.

Chairman:
    Prof. S.C. Bhatnagar
    Indian Institute of Management
    Vastrapur,  Ahmedabad 380 056,  India
    Tel.       +91-272-407241
    Fax        +91-272-467396
    Internet:  root@iimahd.ernet.in

Ag. secretary:
    Mr. Mikko Korpela
    University of Kuopio,  Computing Centre
    PL 1627,  SF-70211 Kuopio,  Finland

    Tel.       +358-71-162811
    Fax        +358-71-225566
    Internet:  korpela@uku.fi
    Bitnet:    korpela@finkuo


AIMS
\begin{enumerate}
\item To collect, exchange and disseminate experiences of developing countries
\item To develop a consciousness amongst professionals, policy makers
and public on social implications of computers in developing nations 
\item To develop criteria, methods, and guidelines for design and
implementation of culturally adapted information systems 
\item To create a greater interest in professionals from
industrialized countries to focus on issues of special relevance to
developing countries through joint activities with other Technical
Committees.
\end{enumerate}

SCOPE
\begin{enumerate}
\item National computerization policy issues
\item Culturally adapted computer technology and information systems
\item Role of transnational corporations, regional and international
cooperation and self-sufficiency in informatics 
\item  Social awareness of computers and computer literacy.
\end{enumerate}

\subsection{Subscribing to VITA}

VITA's public, on-line discussion list, DEVEL-L, provides a forum for
the exchange of ideas and information on a wide range of issues and
topics related to technology transfer in international development; for
example, technologies, communications in development, sustainable
agriculture, women in development, the environment, small enterprise
development, meetings, book reviews. Subscribers to DEVEL-L automatic-
ally receive DevelopNet News. To subscribe, send this command or mes-
sage: SUB DEVEL-L your-full-name  to (BITNET:)LISTSERV@AUVM or
(Internet:) LISTSERV@AUVM.AMERICAN.EDU
 
You can also subscribe to DevelopNet News without joining the discussion
list. Just send an electronic message to VITA.
 
DevelopNet News is an electronic newsletter published monthly by Volun-
teers in Technical Assistance (VITA), a private, nonprofit international
development organization located in Arlington, Virginia. Your redistri-
bution of DevelopNet News is encouraged. Information on the approximate
size of your mailing list is useful to VITA.
 
          President: Henry R. Norman
          Editor: Patricia Mantey
          Editorial Assistant: Rafe Ronkin (VITA Volunteer)
 
VITA specializes in information dissemination and communications tech-
nology. It offers services related to sustainable agriculture, food
processing, renewable energy applications, water sanitation and supply,
small enterprise development, and information management. VITA is cur-
rently involved in long- and short-term projects in 10 countries in
Africa, Asia, and Latin America.

1815 North Lynn Street, Suite 200, Arlington, Virginia, 22209 USA.
Telephone: +1 (703) 276-1800, BBS (VITANet): +1 (703) 527-1086, Fax: +1
(703) 243-1865, Telex: 440192 VITAUI, Cable: VITAINC, BITNET:
VITA@GMUVAX , Internet:  VITA@GMUVAX.GMU.EDU


\subsection{Guide to the Latin-American Networks:Index and Retrieval
Instructions}             

PEDRO@ohstpy writes:

Hi! As you probably know, over the last year I have been sending out
periodic updates of a list of latin-american mailing lists, with
addresses, purposes, and useful tips about how to use them.

Fortunately, the list grew pretty big, and by now almost every country
of the region is included, in one way or another. So, I accepted
the generous offer of the LASPAU administrators to put it on a
listserver, and make it accessible to everyone interested in geting
it. This will avoid the intense traffic every time I send a list out,
and, more important, will save me the time to have to reply to
many requests for copies, and will help those with quota limitations,
or communication charges. I will, of course, let you know when we are
incorporating new updates.


                    HOW TO CONTACT THE LISTSERVER

Send an e-mail message to LISTSERV@HARVARDA.BITNET, with the following
lines as the only text of your message:
GET LISTS   GUIDE LASPAU
GET LISTS-P GUIDE LASPAU
You will automatically receive a copy of the current edition of the Guides.
LISTS GUIDE includes lists of general interest, while LISTS-P GUIDE has
those lists of professional interest.  You can request each of the files
individually.
To see what other information could have been stored, include in the
message the following line:
INDEX LASPAU
If you have any problems with the listserver, please send a note to
the administrators, Cesar Galindo-Legaria (cesar@husc9.bitnet) or
Alberto Oliart (oira@bu-pub.harvard.edu).

\pagebreak
\section{Contact Personnel}
Please send contributions to the newsletter or requests for addition
to the mailing list to KK. Email is the communication method of choice
(it takes so long to type the gubbins in) but communication through any
medium is welcome, especially if it contains contributions to the
newsletter!

Incidentally, if you receive this newsletter in elctronic form and do
not have LaTeX at your site to print it out in a nice format, get in
touch and we'll send you either a hardcopy form or an electronic form
without the commands, whichever you prefer.

\begin{tabular}{l|l}
{\bf Software library:} & {\bf Newsletter, overall co-ordination,
meetings:}\\ 
Howard Beck &  Kathleen King\\
Artificial Intelligence Applications Institute &  Department of
Artificial Intelligence \\
University of Edinburgh & University of Edinburgh\\
80 South Bridge & 80 South Bridge\\
Edinburgh EH1 1HN & Edinburgh EH1 1HN\\
031 650 2747 & 031 650 2726\\
hab@uk.ac.ed.aiai & kk@uk.ac.ed.aipna\\
\hline
{\bf Addresses and contacts, funding:} & {\bf Literature resource and bibliography:} \\ 
Robert Muetzelfeldt & Ehud Reiter\\
Department of Forestry and Natural Resources &   Department of Artificial Intelligence\\
University of Edinburgh & University of Edinburgh\\
Kings Buildings & 80 South Bridge\\
 Mayfield Road &  Edinburgh\\
Edinburgh   EH9 3JU & EH1 1HN\\
 031 650 5408 & 031 650 2728\\
 R.Muetzelfeldt@uk.ac.edinburgh & reiter@uk.ac.ed.aipna\\
\hline

\end{tabular}

\end{document}
