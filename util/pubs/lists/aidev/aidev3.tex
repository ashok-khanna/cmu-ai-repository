\documentstyle[a4]{article}
\pagestyle{myheadings}
\markboth{Author:kk}{Doc.No. AID/newsletter/3}
\setlength{\textheight}{9in}
\setlength{\topmargin}{0in}
\setlength{\headheight}{.3in}
\setlength{\headsep}{.5in}
\parindent=0pt
\parskip= .35\baselineskip plus .0833333\baselineskip minus .0833333\baselineskip
\begin{document}
Artificial Intelligence for Development\\
Document No: AID/newsletter/3\\
Last Modified and by whom: 17/5/91 kk \\
Distributed on: 22/5/91\\
\LARGE
\begin{center} Artificial Intelligence for Development\\
Newsletter Number 3\\May 1991\\
\end{center}
\normalsize

This newsletter is being sent by both electronic and paper mail. If
you no longer wish to receive it, or have any contributions or
comments on its content, please contact KK by any of the normal
methods. Contributions will be particularly well received!

\section{Micro Hydro Scheme Design for Nepal}
Teresa Anderson (ta@ed.emf) writes:

The decision support software for Micro Hydro design is a computer
program which typically carries out the task of combining data
supplied by a user with information in its own database, in a way
prescribed by its `rule base', in order to infer conclusions about the
situations described by the input data.

Many scheme assessors and designers have professional expertise in only
one of the disciplines contributing to micro hydro, having acquired
the rest of their knowledge `on the job'. Decision support software
provides an expert resource for such workers. Data drawn from a wide
range of experts can build up an invaluable overall picture, with the
knowledge and experience of many people stored in one system.

Decision support software can take a designer through a standardised
procedure, ensuring that nos stages of the process are missed. This
ensures that biases peculiar to each designer are removed and provides
a safety net to check fo any errors in design. It provides a method
whereby all individuals in an organisation use standard procedures and
standard components, and maintains a constant emphasis on the relative
importance of various criteria from day to day. An added advantage is
that it speeds up the process of carrying out standard calculations.

A designer who is pressed for time may find it impossible to make more
than a few passes through the iterative calculations associated with
optimisation routines. A computer can carry out the repetitive
calculations quickly and reliably, allowing the designer to see the
effect of changing the values of various system parameters without
wasting valuable time.

Where no national hydrological data base exists, building up data from
gauged small and micro hydro sites can be a first step towards
recording the flow regime of the system of small rivers and streams.
Decision support software for site evaluation would also generate
records of data on potential sites in a standardised format. Scheme
records including data such as component specifications, site layout
sketches, details of breakdowns and repairs, would be invaluable to an
installation and maintenance team. Recurring faults can be shown up
when breakdowns are logged systematically. The specifications, price,
and availability of components by various manufacturers can also be
stored. 

The practice of repair on demand results in unnecessary loss of
operation time, and therefore revenue. Routine maintenance scheduling,
where equipment and civil works are checked for problems at regular
intervals, can significantly reduce the number of occasions on which
total shutdown of a scheme is required. Decision support software can
prompt this, either on-screen, or by printing out a suggested schedule
at the commissioning date.

An extensive system of information files and explanations of how
decisions are arrived at are essential for any interactive design
software. This would be useful for building up the expertise and
confidence of practitioners new to teh field, who could use the system
to experiment with different designs for schemes and see how certain
problems may be solved.

\subsection*{The MICADO System}
The MICADO system of Decision Support software for MICro hydro
Assessment scheme Design, Optimisation and record-keeping has been
configured on the CRYSTAL expert system shell using LOTUS1-2-3 as a
database template for frequently accessed data. The system is modular
in structure, and carries out tasks such as assessment of hydrological
data and demand for power, civil works calculations, and a selection
of electromechanical equipment from databases of available models. the
software interacts with the user via natural language, asking YES/NO
questions, requesting numerical data and occasionally asking for
subjective evaluations for instances in which no quantitative data is
available. Information files are included which employ both text and
graphics, and a HELP function can be invoked while the system is
running. This general approach has such attractions that several
similar suites of software for the design of small-scale hydro schemes
in the developed world are now being considered.

However, the conditions which obtain in developing countries require
more flexibility in the mode of their solution. In operation,
therefore, an experienced practitioner can override the MICADO system
recommendations at all stages, if there are specific local
circumstances which the software cannot take into account. In this
sense the software provides decision support, rather than making
inflexible recommendations to designers which may be impossible to
implement given local or national conditions.

The system can be run on PCs in offices by engineers running
optimisation routines, or consulting scheme details, by officials of
development banks assessing feasability of applications for scheme
funding, or can be run on laptops by engineers in the field carrying
out site and demand surveys. While the design methods and rationale
have been largely dictated by the situation in Nepal, it is envisaged
that the system will also have useful applications in other countries.

A prototype version of the software has been taken to Nepal and runs
successfully on PCs and laptops belonging to various organisations
concerned with the promotion of micro-hydro designers, equipment
manufacturers or operators. The software package is presently being
refined ro include improvements and modifications which were suggested
by potential users in Nepal. Other features are also being included on
the basis of the experience of the authors working in other countries
with similar conditions and problems. Field trials for the modified
system are scheduled to begin in April 1990.

\section{Low cost global electronic communications networks for the developing world.}
Mike Jensen (mikej@uucp.gn) writes:

Electronic mailbox and messaging services offer an ideal tool for
enhancing communications in the less developed world. It is a cheaper
and more convenient technology than facsimile or telex wherever
a computer is available. However, the communications infrastructure in the
less developed countries varies from very good to abysmal. As a result, the
appropriate communications solution may vary from one location to the next.
This paper outlines the two basic means of connecting mailboxes to the
global network and discusses which method may be the most appropriate for
the various circumstances.

{\bf The Technology}\\
Basic communications links and messaging systems\\
Many of the less developed countries are now installing a packet switched
data line service, also called PSS/IPSS (International Packet Switched
Service) which uses the internationally standardised X.25 protocol. The
PTT - national post office or telephone company is almost always the
operator of such a service and usually installs connection points to IPSS
in the major cities. This allows modem users in these cities to make a
local voice grade phone call to get on to the international packet switched
network. From there commands can be issued to link up to any country with
an electronic mail or database host computer connected to the X.25 network.

To access such a service, the user orders a NUI (Network User ID) from the
local PTT. This comprises a registration fee, monthly or quarterly rental,
and usage charges to connect to the remote host. 

For regular users, NUI rental usually provides a significantly cheaper
option than making a direct dial international phone call to the electronic
host. If the host is accessed infrequently, then the cost of a NUI may not
be justified. As with a normal telephone call, there is usually a
substantially higher usage charge for connecting to a host outside the
country. Since there are  still very few internationally connected local
hosts in the less developed  world, this will generally be the case. The
host bills separately for  the use of its services but for sending
messages, up to 90per cent of the cost of the  international connection can be in
the charges made by the local PTT for use of the NUI.

Rate structures for IPSS are complex and monthly costs of using such a
service to connect to a host can vary enormously from one to country to
another. Rental charges for a NUI can vary from 20 dollars a quarter to 200.
Some PTTs expect the NUI user to rent PTT owned modems at inflated rates.
Even usage charges (which are based on time spent online and the volume of
date passed down the network) can vary by a factor of five between
different PTTs. Most of the developing
countries do not have an IPSS service. Asia and Africa are particularly
under-represented. Where it is available in these regions it is usually
considerably more expensive than in the West. On top of this certain PTTs
may only have an IPSS that connects to certain countries and not others.

As a result, international direct dialing is often the only option. With
conventional terminal software and bad telephone lines this method of
connection can be expensive, unreliable and stressful for the operator.
However, recent developments in PC based communications software have
improved the situation. It is now possible to send messages and files over
poor quality telephone lines at minimal cost using automated computer
controlled connections with file compression and error checking.

These programmes typically reduce the length of the long distance call by
80-95 per cent compared to the time taken for a standard interactive manually
controlled session with the host. They result in completely error free
transmissions, without the need for manual intervention of the operator.
Using this software is more like sending a fax than going through the
series of 'log on' procedures necessary connect to a remote host.

Developed in the amateur and academic environment over the last 10 years
much of this software is free for non-commercial use or very cheap to
purchase (ECU10-ECU100), running on any IBM compatible or Macintosh. Currently
there are over 10 000 such systems exchanging messages and files globally.
Messages can be prepared separately on any type of word processor and a
2400 baud modem costing about ECU 200 serves to link the PC to the
telephone line. The equipment does not require the installation of a
separate line - existing voice or fax lines can be temporarily diverted to
the modem while it places the call.

Any such system can also be left switched on for longer periods, in a state
ready to  receive messages from other such systems. Researchers in the
field can use the computer and modem in this fashion to place calls
directly to others in the South using the same software. Alternatively they
could make direct dial calls to a host in Europe or North America where
files and messages can be stored for pick-up by the central office, or for
onward transmission to other regions. This system is already being used by
a number of organisations in the developing world, including Kenya, Zambia,
Zimbabwe, the Phillipines and Baltic states.

With the 2400 baud modem, users are reliably achieving transmission speeds
of 220 characters per second (cps). Because the messages and files are
automatically compressed before transmission to one third of their original
size (and even more for fixed length record databases - up to 10 times) it
is possible to send or receive about 40 000 characters (about 6 500 words)
during a one minute call. Because the connection between the computers is
all under control of the machine at each end, the only time when the full
220 cps transmission speed is not being achieved is during the first 10-15
seconds while handshaking between the two computers takes place.

The file transfer protocols used between the two computers have a high
level of resiliency to line noise and satellite delays, continuously
adjusting the packet size to appropriate values. In addition, if the
connection breaks while a file transfer is taking place, the transfer is
able to pick up where it left off on the next call. This is particularly
important for transporting large binary files where the chances of losing
the connection over poor quality telephone lines is significant.

For the cost of about ECU 1000 a high speed modem can be purchased when
the volume of communication makes it more cost effective. Depending on the
quality of the phone line, a modem such as the Telebit Trailblazer (TM)
can transmit data 4 to 8 times faster than the 2400 baud modem.

Host systems to carry this traffic into the North are currently operating
24 hours a day in London, Stockholm and Toronto supporting the high speed
V.32 and PEP protocols as well as the standard V.22 (1200 baud) and V.22bis
(2400 baud) protocols. These machines provide hourly gateway connections to
the APC (Association for Progressive Communications) and GeoNet X.25 hosts
in Brazil, Australia, Sweden, Nicaragua, US and Canada, and many countries in
Europe. Messages can be sent through these machines to outbound fax and
telex servers, to commercial hosts such as Dialcom, and to other networks
like Janet, BitNet, and UseNet/UUCP.

For many purposes, sending files and messages to directly another
individual is all that is necessary, however, there is also the opportunity
to 'broadcast' the message to a select group of participants. These
'mailing lists', also known as electronic conferences or bulletin boards
can be publicly available to anyone on any of these networks, or restricted
to a select group - for example a co-ordinating committee. The sender does
not have to know the electronic address of each participant to send them
each a message, instead a single message is sent to the predefined mailing
list running on the host which then decides which systems to pass the
message to. The list could comprise an unlimited mixture of fax numbers,
telex numbers, electronic mail addresses and bulletin boards or conferences
running on certain hosts.

{\bf Implementation}\\
One of the biggest tasks facing an organisation needing access to the
electronic network will be carrying out training in its use.

Using the new communications software is definitely the most appropriate
way of connecting worldwide. However it may not operate on non-standard or
non-IBM/Apple equipment. As well, many  users may already be
familiar with an existing means of communicating with their local host.
Therefore the quickest means of bringing up the network may be to leave
those already connecting to the global electronic network with the tools
they are using. Thereby leaving the initial phase of the project to
concentrate on bringing new participants online.

For someone familiar with the computer for word processing or some other
basic application, a half day training workshop would be sufficient to
acquaint the user with all that is necessary to send and receive files and
messages.

A self installing configuration is available for IBM compatibles and a
running system can ideally be set up in half an hour by someone without any
special skills other than basic familiarity with the keyboard. Occasionally
there are a variety of problems that can crop up. Non-standard hardware
configurations may need some trouble-shooting by someone familiar with the
DOS operating system and DOS level commands. Hooking up the modem to a PABX
type telephone system can be difficult, and may require the assistance of
the phone company or PTT. Non-standard modems, telephones wired directly
into the wall and operator assisted direct dialling can also be problematic
for the inexperienced. For this reason it is probably best to consider each
installation individually.

Clearly the organisation's main office will need to install the software
and evaluate it. Following approval, a couple of remote test sites could be
established where setting up the link is going to simplest or access to
technical support is available. When lessons have been learned from this,
further sites could then be established. Attempts are being made to keep
track of consultants who may already be travelling in the field and may be
available for further training sessions.

\section{News from around the world}
Contributions to this section are particularly welcome. Personal
experience from living and working in a country, or even having
visited for a short time, is always interesting, and this is also a
good forum to pass on informal knowledge gleaned from other sources.

\subsection{SatelLife groundstation in Zimbabwe}
Deiter Klein (dklein@edu.wpi.wpi) from SatelLife writes:\\
that SatelLife are now operating on a test basis and have a ground station
in Zimbabwe, in the care of the (medical?) library at the University
of Zimbabwe. They are setting up an uplink station in St.John, NF in
Canada where they will be licensed.  This will be the link up for all
the stations through peacenet. 

\section{Groups, Bulletin Boards and Mailing Lists}
\subsection{BCS Developing Countries Specialist Group}
The BCS Developing Countries Specialist Group provides a network of
contacts for members interested in information technology practices in
developing countries. It also holds seminars and workshops which help
enhance understanding of the special issues developing countries must
resolve in their approach to IT.

Recent seminars have included `Geographical Information Systems (GIS)
1990', `IT Education for Developing Countries' March 1991 and the next
event will be `Value Added Networks (VANS) in Developing Countries'
planned for later this year.

This information is taken from an article which appeared in `The
Computer Bulletin' of March 1991. 
Anyone interested in joining the BCS developing Countries Specialist
Group should contact the Group Secretary, Mike Gunner at ICL Slough,
Tel. 0734 352 915 or Philip Veasey, DCSG chairman, ICL, Feltham tel.
081 890 1414.

\subsection{New Newsgroup}
After a successful vote we have established a new electronic news
group to be entitled comp.society.development. The group should be
coming on line very soon. Watch out for it and subscribe! 
The group will discuss applications of computer technology
in developing countries.  Topics might include: 
\begin{itemize}
\item specific stories of successful (and unsuccessful) uses of computers
    in developing countries
 \item  general discussion of the difficulties and advantages of using
    computers in developing countries
 \item  comments from developing country computer users on their experiences
 \item  requests from developing country computer users for computer-related
    advice
 \item  news about communications, especially networks and electronic mail,
    to and within developing countries
 \item  updates from aid agencies, volunteer groups and others active in
    developing countries on computer-related projects underway or under
    consideration
 \item  discussions about appropriate hardware and software
 \item  general news about books, seminars, conferences, and projects
\end{itemize}
General discussions of Third World politics or economics will be avoided.

\subsection{Latin American Mailing Lists}
We have information on electronic mail addresses and subscription
details for many latin american mailing lists. The members are people
that have a common national origin or an interest in that particular
region. They exchange information and news, and discuss about topics
such as politics, gourmet food, traveling, and many matters of daily
life. They also serve for professional exchange or just to make new
friends. 

The mailing lists listed are non-commercial (free access) and one
member volunteers to administrate a list. Areas covered include
Argentina, Brasil, Caribe, Central America, Chile, Colombia, Ecuador,
Iberoamerica, Latin America, Mexico, Peru, Quebec, Uruguay and
Venezuela. There are also professional mailinglists concerned with
astronomy, health, physics, psychology and  social sciences. 

If you would like to know more about these mailing lists, contact ER
through any of the normal channels.

\subsection{THE LASPAU NETWORK PROJECT}
The LASPAU Network Project seeks to encourage computer communications
for academic purposes throughout the Americas. 
 
The mission of LASPAU is to address the international education and
training objectives of institutions and individuals by supporting,
developing, and managing academic exchange and professional development
programs with an inter-American focus.  LASPAU is an association of more
than four hundred institutions of higher education throughout the
Americas.  Established in 1964, LASPAU is affiliated with Harvard
University and is governed by an inter-American board of trustees.
Through a variety of specialised services, LASPAU carries out programs
for several sponsors including the United States Information Agency, the
United States Agency for International Development, the Fundacion Gr n
Mariscal de Ayacucho of Venezuela, and multilateral funding sources such as
the World Bank and the Inter-American Development Bank.  LASPAU thus assists
these sponsoring agencies as they cooperate with Latin American and Caribbean
institutions, and on occasion with institutions elsewhere, that seek to assess
and fulfill their educational and training needs.  The organization also
offers specialized educational consulting services to institutions both inside
and outside the Americas.  The majority of LASPAU-administered awards support
graduate training at the master's level, but a significant number support
doctoral, undergraduate, and non-degree study.

To subscribe to this list, send an electronic message to
LISTSERV at HARVARDA (or LISTSERV@HARVARDA.BITNET), with the
following contents:
 
   SUB LASPAU-L (your name)

\section{Conferences} 
Philip Machanik (philip@edu.stanford.pescadero) writes:

AITEC SOUTH - Computer and Communications Expo and Conference, 
Harare 13-16 November 1991.\\
This will be run in partnership with the Confederation of Zimbabwe 
Industries, as well as the Computer Society of Zimbabwe and the 
Zimbabwe Institute of Public Relations. The steering committee 
includes representation from the CZI, Computer Society, Computer 
Sellers' Association, Apple Users' Group, Government Computer 
Bureau and Posts and Telecommunications Corporation.
The event will consist of a conference and an exposition, with 
international speakers. The theme is "Computers and 
Communications in Practice". The aim is to make it a regional event, 
with a combination of executive briefings and technical workshops.

AITEC EAST - Similar event in Nairobi, Kenya, 26-28 June 1991.\\
The overall conference is chaired by George Okado, Kenya Computer 
Institute. The conference planning committee includes Dr Edmund 
Katiti, president of the Uganda Computer Society, Sean Moroney, 
publisher of Computers in Africa, Alexander McNabb, publisher of 
Communications Middle East Africa, and Simon Bell, University of 
East Anglia (UK).
Conference topics include Implementing an IT Strategy, Government 
Applications, Desktop Publishing, Office Productivity Applications, 
Computer-Aided Design/Manufacturing, Medicine and Health 
Applications, Public Telecoms - African ISDN Strategies and 
Electronic Mail.

There are a few bonuses for registering by 31 May. If you want more 
information write to:\\
  Sean Moroney\\
  ITP Africa File Limited\\
  Angus House\\
  13 Tilehouse Street\\
  Hitchin\\
  Hertfordshire\\
  SG5 2DU\\
  UK\\
phone international + (44 462) 420785, fax 420786

\section{Next meeting:}
Well, it looks like we have no-one lined up at the moment.
Please send suggestions for future meetings to KK. Unfortunately we
have no finances to pay for visiting speakers, but if you know of
anyone interesting who'll be in town (or within the sort of vicinity
that it wouldn't bankrupt us to get them here) and would like to speak
to the group at some point, please get in touch. Perhaps {\em you}
would like to give a short talk...

\section{Contact Personnel}
\begin{tabular}{l|l}
{\bf Software library:} & {\bf Newsletter, overall co-ordination,
meetings:}\\ 
Howard Beck &  Kathleen King\\
Artificial Intelligence Applications Institute &  Department of
Artificial Intelligence \\
University of Edinburgh & University of Edinburgh\\
80 South Bridge & 80 South Bridge\\
Edinburgh EH1 1HN & Edinburgh EH1 1HN\\
031 650 2747 & 031 650 2726\\
hab@uk.ac.ed.aiai & kk@uk.ac.ed.aipna\\
\hline
{\bf Addresses and contacts, funding:} & {\bf Literature resource and bibliography:} \\ 
Robert Muetzelfeldt & Ehud Reiter\\
Department of Forestry and Natural Resources &   Department of Artificial Intelligence\\
University of Edinburgh & University of Edinburgh\\
Kings Buildings & 80 South Bridge\\
 Mayfield Road &  Edinburgh\\
Edinburgh   EH9 3JU & EH1 1HN\\
 031 650 5408 & 031 650 2728\\
 R.Muetzelfeldt@uk.ac.edinburgh & reiter@uk.ac.ed.aipna\\
\hline

\end{tabular}

\end{document}


