\documentstyle[a4]{article}
\pagestyle{myheadings}
\markboth{Author:kk}{Doc.No. AID/newsletter/6}
\setlength{\textheight}{8.5in}
\setlength{\topmargin}{0in}
\setlength{\headheight}{.3in}
\setlength{\headsep}{.5in}
\parindent=0pt
\parskip= .35\baselineskip plus .0833333\baselineskip minus .0833333\baselineskip
\begin{document}
Artificial Intelligence for Development\\
Document No: AID/newsletter/number 6\\
Last Modified: 8/7/92 kk\\
Distributed on: 8/7/92\\
\LARGE
\begin{center} Artificial Intelligence for Development\\
Newsletter Number 6\\July 1992\\

\end{center}
\normalsize
\section*{Contents}
\begin{enumerate}
\item New literature on AI for Development
\item Medical Informatics in Brazil
\item News from around the world
\item Conferences
\item Groups, bulletin boards and mailing lists
\item Contact Personnel
\end{enumerate}

\section{New literature on AI for Development}
\subsection{Expert Systems for Developing Countries}

{\bf Rafe Ronkin} (VITA@gmuvax.gmu.edu) writes:

EXPERT SYSTEMS IN DEVELOPING COUNTRIES?

The advanced technology called "expert systems" can help solve many
problems of developing countries; research workers say that its
application to these problems is overdue. 

Expert systems are based on computers, which use information and
reasoning techniques to solve problems that normally require human 
experts. They may rely on "artificial intelligence" software to help
human experts do their work, to train new experts, and to provide some
elements of expertise in situations where human experts are in short
supply. For at least a decade, expert systems have proved themselves in
developed countries, especially in agriculture, business, and primary
health care. Can they help people in developing countries to improve
their conditions or create wealth? This question is addressed, if not
completely answered, in a newly published collection of papers.

Health-care delivery, natural-resources management, and crop planning
present challenges not only because their problems urgently need
solving, but because they are complex enough to require expert judgement. 
In many developing countries with abundant supplies of underemployed
educated persons, computers and computer-literacy are still uncommon.
However, the rapid spread of microcomputers and electronic
communications in all parts of the world is making it easier for people to use 
these powerful tools. In addition, the software for expert systems has
become more affordable and user-friendly. Finally, in many countries
survey databases are already available that can be used by expert
systems; for example, in health-care financing in Rwanda and water-supply 
problems in Slovenia.

T. W. Fermanian (University of Illinois, Champaign-Urbana) and R. S.
Michalski (George Mason University, Fairfax, Virginia) list the features
that make today's software so adaptable to the needs of developing
countries: The programs not only accept rules of reasoning given by the 
user, but also infer new rules from the data and by "experience." They
accept data and rules that have numerical probabilities or weighting
associated with their use. They can be used on a personal computer. They
have menu-driven screens. In addition, some programs reveal their
reasoning on request, after solving a problem, thus providing valuable 
feedback to human experts, who can then further improve the reasoning
rules.

Examples: Dominican Republic, Ethiopia

D. Mendez E. and S.V. Grabski (Michigan State University, East Lansing)
are basing their agricultural studies on the Geographic Information
System that is already well developed in the Dominican Republic. Their
purpose is to help policy makers plan the best use of land for forests
and food crops. Consider onions: they thrive best under certain
conditions of soil pH, rainfall, temperature, drainage, etc.; the same kinds 
of information are part of the national geographic database. According
to Mendez, the matching of crop and geographic databases "will be used
when a farmer wants to know what crops a specific site is capable of
supporting."

Already the system has produced one benefit: it has demonstrated that
the administrative boundaries of the country's agricultural regions have
little relation to physical "agroecological" regions. Accordingly, the
authors plan to use both kinds of mapping to help policymakers. The
system now provides only feasibility analyses, but the next stage will
include cost / benefit analysis. A sensitivity analysis to be developed
will show how robust the answers produced by the expert systems are to
changes in the underlying estimates. And every time a recommendation is
implemented, the system will stored its outcome as part of its
"experience."

G. Porenta (University Hospital, Vienna, Austria) and colleagues are
designing a decision support system to serve village health workers in
Ethiopia. Diarrhea, worm infestations, eye disease, skin afflictions,
and malaria make up about half of the daily work. Like most progams
based on artificial intelligence, their software contains a large number
of "if ... then" statements. For example, "if the patient is not a
child," then ask "if a dry tongue is present." If so, then see a doctor.
Symptoms are given weights to arrive at the treatment of choice. If
several treatments-of-choice are indicated, the software selects the
least costly.

Porenta's program is based on particular assumptions about the
environment: a highland region, civil war, poverty and famine, and closed 
rural-health centers. Transplanting the program to another region or
country may require changing the assumptions. According to Porenta's
team, the most important challenge now is to develop a more
user-friendly, graphical interface. Other research workers are developing a 
health-related system in Chad.

Practice and Promise

In a summary chapter, V.S. Doherty, C.K. Mann, and J.J. Sviokla
summarize existing knowledge about the use of expert systems in international 
development. First of all, they have found no case studies that clearly
show success or failure of expert systems in this context. Indeed, all
indications point to the critical need for field testing of the
prototype systems that are described in the book's descriptive chapters. 
These systems were mainly developed under conditions in affluent
countries, but the ready availability of microcomputers and expert software 
strongly suggests that developing countries can and should develop their
own prototypes and test them in the field. Moreover, the low startup
costs for installing expert systems should favor their local use; for
example, by associations of farmers. In developing regions, expert
systems can be used as a training aid, for expertise development, and to 
assist policy makers in articulating problems. According to Doherty,
growth in the use of expert systems is favored by the investment that
developing countries are making in computers and communications.

This thoughtful and well edited book is based on a symposium held in
Boston in 1988 and sponsored by the American Association for the
Advancement of Science. In view of rapid developments, it is fortunate
that several contributing authors have updated their papers and added
current references. Of the 13 research chapters, one concerns water
analysis, one is on enterprise development, three are on health-related
systems, and the rest are related to agriculture.

Charles K. Mann and Stephen R. Ruth (eds.), 1992, Expert Systems in
Developing Countries; Practice and Promise. Boulder: Westview Press
(5500 Central Avenue, Boulder, Colorado 80301. Inquiries and orders: +1
(303) 444-3541; orders only: +1 (800) 456-1995.)

\subsection{Expert Systems in Health for Developing Counbtries}

Dayo Forster: `Expert Systems in Health for Developing Countries:
Practice, Problems and Potential'

May 1992

ISBN: 0-88936-637-3

Manuscript report published by the International Development Research
Centre (IDRC), Ottowa (64 pages)

Dr Forster was commissioned by the IDRC to do a survey on expert
systems in health care for developing countries as a follow-on from her
related thesis work. This is a review of several expert systems for
health in developing countries. The document examines 15 pieces of
software and the author has compiled a bibliography of more than 150 articles
and pieces on this and related areas. More than half the document is
devoted to two very useful lists; the large bibliography and a list of
people throughout the world connected with this area.

\section{Medical Informatics in Brazil}
(Reproduced from  mailing list SMDML without permission)
 
The Center of Biomedical Informatics is an
interdisciplinary research center affiliated to the State
University of Campinas (UNICAMP).
 
AIMS
 
The Center of Biomedical Informatics has the following aims:
 
- to promote interdisciplinary research and
  development in all fields related to the application of
  Informatics to the Biological and Health Sciences;
 
- to promote specialized training and education in
  these areas, by means of courses and graduate programs in
  collaboration with other UNICAMP units;
 
- to make available and to extend this knowledge to
  the participating and user's community, through
  publications, seminars, technical consultancy, etc.
 
The Center of Biomedical Informatics has played a leading
role in the development and progress of Health Informatics
in Brazil. Its members were instrumental in forming the
first professional society in the area (the Brazilian
Society of Health Informatics), its first and only journal
(Brazilian Journal of Health Informatics), in organizing and
hosting the First Brazilian Congress of Health Informatics
(Campinas, 1986), and in actively helping federal, state and
research/higher education and research granting agencies in
setting up programs in the area.
 
RESOURCES
 
Currently, NIB has a staff of 2 full-time researchers, 5
associated researchers, 3 full-time systems analysts and 2
administrative assistants. There are an additional 20
persons involved with NIB activities, including other
investigators, undergraduate and graduate students, etc.,
associated to ongoing research projects. NIB is located in a
building belonging to the huge in-campus complex comprising
the 1,000-student Medical School and the 500-bed University
Hospital complex.
 
The majority of projects within NIB are oriented towards
practical applications of microcomputers to health care and
research. NIB is connected to the University's Computing
Center facilities, which are one of the most advanced in the
country. They consist of a cluster of four high-performance
Digital VAX mainframes, a Cyber 700 and an IBM 3090
mainframe with a 300-mips vector facility, all connected to
a local optical fiber Ethernet network which integrates
other VAX and IBM superminicomputers located in several
Institutes.
 
 
RESEARCH AND DEVELOPMENT
 
The most important areas of research and development at NIB
are:
 
- hospital and laboratory automation;
 
- general clinical applications (history-taking,
  clinical database management, health risks appraisal,
  medical decision-making);
 
- scientific applications in several areas
  (neurology and neurobiology, cardiology, anesthesiology,
  ophthalmology, internal medicine, dentistry, nursing,
  intensive therapy, obstetrics and gynecology, etc.)
 
- computer-assisted health education and training,
  and teaching of Informatics to health students and
  professionals; including intelligent CAI, hypertext and
  hypermedia applications
- biological image and signal processing
- knowledge engineering, intelligent data bases in
  Health, expert systems and neural networks applications in
  Biology and Medicine.
 
The main line of excellence of the Center's research is
devoted to applications in the Neurosciences.
 
Many of the Center's research and development projects are
carried out in close collaboration with other Departments
and Centers in or outside UNICAMP, including the Medical,
Dentistry and Nursing Schools, the University General
Hospital, the Institute of Biology, the School of Physics
Education, the School of Electrical Engineering, the
Institute of Mathematics, Statistics and Computer Science,
the Biotechnology Center, the Center of Informatics in
Education, etc.
 
TRAINING AND EDUCATION
 
NIB offers regularly many lectures, courses and technical
seminars to students and users in the health care area, both
internal and external audiences. Some of the courses
described below have been  given in programs of permanent
collaboration between NIB and several professional and
scientific associations in the health sector, such as the
Medical Association of Sao Paulo, the Dentistry Association
of Sao Paulo, the Brazilian Nursing Association, etc.:
 
- Basic and Advanced Concepts in Health Informatics;
- Utilization of Microcomputers in the Clinical Practice
- Applications of Computers in Nursing
- Applications of Computers in Dentistry
- Introduction to Computer Programming in the Health Sciences
- Applications of Microcomputer Database Management
  Systems in Health Care
- Medical Applications of Image Processing
- Microcomputer Applications in Data Processing and
  Analysis in the Health Sciences
- Artificial Intelligence Techniques and Applications in Medicine
- Artificial Neural Networks in Medicine
- Applications of Computers to Medical Education
 
Starting 1986, an intensive Summer Course on Health
Informatics has been offered to participants coming from
other Brazilian Universities, with the aim to disseminate
the Center's experience in setting up R and D centers in Health
Informatics.
The Center provides also the main support to the Chair of
Medical Informatics at the School of Medicine, which is
responsible for courses at undergraduate and graduate levels
to Medical and Nursing students, since 1988.
At international level, NIB has collaborated with the
World Health Organization (Division of Information Systems
Support) in developing teaching materials and in providing
faculty resources to seminars and courses on microcomputer
applications to health care management.
In order to provide better opportunities for specialization
studies in Health Informatics, the Center has established a
collaborative program with the Medical and Engineering
Schools at UNICAMP, designed to attract undergraduate
students of Medicine, Nursing, Dentistry, etc., who are
interested in a long term career in this area. Training is
carried out through special courses developed by NIB, and by
supervised research work on an individual basis. The
students who show a higher potential are supported by state-
granted fellowships until they graduate. There are currently
5 students enrolled in the Centaur Project.
Finally, there is an ongoing collaboration between NIB and
other University faculties, with the aim of forming
specialists in Health Informatics, at the graduate level.
There are currently 6 candidates preparing their Masters and
doctoral dissertations under this program. Preparations are
also being made to establish an autonomous, separate joint
graduate studies program in Health Informatics, which will
lead to a Masters degree.
 
DOCUMENTATION AND INFORMATION SUPPORT
 
With the aim to provide services of information support and
documentation to researchers, lecturers, students and end-
users in Health Informatics at a national level, NIB has
established  a clearinghouse and a central library facility
with books, journals, product  specifications and folders,
technical  manuals, software listings, etc. A national,
public-domain software library for microcomputers has also
been formed (MEDSOFT), and already includes a great variety
of application software packages in health care and
administration, which are distributed to users under a
nominal cost.
 
Some of these materials (software and bibliographic
references, as well an information service about new
products, congresses, new publications, etc.) are also
disseminated to end-users through electronic mail.
Finally, the Center's staff is responsible for the editing
and publishing, in a joint effort with a private publisher,
of the Brazilian Journal of Health Informatics, which is the
only publication of its kind in Latin America, and is
sponsored by the Brazilian Society of Health Informatics.
 
ADDRESS FOR CONTACTS
 
Prof. Dr. Renato M.E. Sabbatini\\
Director, Nucleo de Informatica Biomedica\\
Universidade Estadual de Campinas\\
P.O. Box 6005\\
13081 CAMPINAS, Sao Paulo, Brazil\\
Tel. (+55 192) 39 7130\\
Telex +55 19-1150 uec br\\
Fax (+55 192) 39 4717\\
Email  INFOMED@CCVAX.UNICAMP.BR\\
       INFOMED@BRUC.BITNET

\section{News from around the world}
\subsection{Kenya linkup}
(Reproduced from newsgroup soc.culture.african without permission)
Andrew Perrin (aperrin@igc.org) writes:

Rates and Local Information: There is a FIDO system in Kenya gatwaying
through GreenNet.  This system carries about 50 APC Conferences, and
has full Email capabilities. 

Contact information: ORGANISATION: ELCI, ADDRESS: Environment Liaison
Centre International, Box 72461, TOWN: Nairobi, PHONE: +254 2 562 015,
COUNTRY: Kenya, PHONE2: +254 2 562 022, CONTINENT: Africa, FAX: +254 2 562 175,
SYSOP1: Doug Rigby, SYSOP2: Protus Muteshi, USERS: 19, DESCRIP: ELCI
is a large International Environmental NGO, currently hosting one of
the NGONET nodes.

There is also a packet switching service in Kenya.  Please contact Doug
Rigby at the above address for further information.

\subsection{Networking in Africa}
(Reproduced from newsgroup soc.culture.african without permission)

{\bf Tony Putnam} (tonyp@ucthpx.uct.ac.za) writes:

Electronic networkers from all over Africa, with providers of support
from the more technologically-developed countries, took part in a
workshop last month in Toronto, Ontario, Canada to discuss future
developments of the fast-growing use of electronic mail in Africa.
Many of the participants were operators of Fido systems.  Fido is a
distributed networking system running on personal computers that
collect messages automatically during regular calls to a "host"
computer (like IGC).  Fido users may then read and respond to their
messages at leisure "offline."  Responses are collected and sent during
the next call to the host.  Messages are compressed into "packets"
and sent using the most efficient means possible, given poor
telephone lines.

Over the last eighteen months, Fido has emerged as one of the
predominant technologies providing electronic networking in Africa,
especially where UUCP-based systems (a more traditional means of
computer networking) are unavailable.  There are now gateways between
Fido and APC networks, for both mail and conferences, and there are several
well-established Fido-based networks that are working closely
with the APC.

The Toronto workshop was very successful, covering both technical and
organisational aspects and laying the foundations for solid growth based
on the existing networks.  Fido hosts connecting to APC gateways on
GreenNet and Web (Canada) are already operating in Senegal, Kenya,
Burkina Paso, Egypt, Uganda, Tanzania, Ethiopia, Ghana, Zimbabwe, South
Africa, and Zambia.  It is expected that this will expand to forty
African countries within the next three years.  For more information,
contact karenb@gn.apc.org or arni@web.apc.org

\section{Conferences}
\subsection{THE 13TH NATIONAL COMPUTER CONFERENCE and EXHIBITION: Saudi arabia}
{\bf Name:} 13th  National  Computer  Conference and Exhibition\\
{\bf Date:} 27/5-2/6, 1413 AH, 21-26 November, 1992 AD.\\
{\bf Venue:} Riyadh\\
{\bf Hosted:} King Abdulaziz City for Science and Technology (KACST)
and the Saudi Computer  Society\\
{\bf Topics:} HUMAN ASPECTS: Human Computer Interaction, Training,
Legal Aspects, Social Aspects, Computer Literacy and Education,
Special Interest Groups. RESEARCH AND DEVELOPMENT: Information
Industries, Tools and Infrastructures, Role of Research, Planning and
Management. OTHER: Computer Networks, VLSI, Systems Architecture,
Software Engineering, Artificial Intelligence, Data Bases.\\
{\bf Format for papers:} 5 copies mailed to address below. Abstract
not more than 200 words, paper length not more than 15 ss A4 pages.
Margins approx. 3 cm.  References thus: {Number}, author name, title,
journal or publisher, volume, number, place and date, page numbers.\\
{\bf Deadlines:} Full text submission: 26/12/1412 AH  (27 June 1992
AD)\\
Notification of acceptance: 10/4/1412AH (6 October, 1992 AD).\\
{\bf Correspondence:} Chairman of Research Committee,\\
The 13th National Computer Conference\\
Directorate of Information Systems\\
King Abdulaziz City for Science and Technology (KACST)\\
P.O. Box 6086, Riyadh 11442, Saudi Arabia\\
Tel:    (966-1) 481 3273\\
Fax:    (966-1) 488 3118\\
E-Mail: NCC13@SAKACS00.BITNET\\

\subsection{First International Working Conference on Health
Informatics in Africa (Provisional) (REPEAT)} 
{\bf Name:} HELINA'93 \\
{\bf Date:} 19-23 April 1993\\
{\bf Venue:} Ile-Ife, Nigeria.\\
{\bf Hosted:} International Medical Informatics Association (IMIA)
jointly with O.A.U. Teaching Hospitals Complex, Ile-Ife, Nigeria;
Computer Science Department, Obafemi Awolowo University, Ile-Ife,
Nigeria; Computing Centre, University of Kuopio, Finland; and
prospective other co-organizers, co-sponsors and supporters.  \\
{\bf Topics:} The programme will cover the whole range of existing
understanding and experience on the field of computers in health care
in Africa, from Primary Health Care to hospitals and national
planning, from records-keeping to research and telecommunications.\\
{\bf Proceedings:} will be published by Elsevier North-Holland.\\
{\bf Language:} English.\\
{\bf Deadlines:}30 April 1992:      Deadline for letters of intent.\\
31 July 1992:       Deadline for extended abstracts.\\
September 1992:     Letters of acknowledgement sent. Final decision on
                    holding the conference depending on funding.\\
30 November 1992:   Deadline for full texts of papers.\\
{\bf Correspondence:}
Mikko Korpela/HELINA'93\\
Univ. of Kuopio, Computing Centre\\
P.O.Box 1627\\
SF-70211 Kuopio\\
FINLAND\\
E-mail:     helina@uku.fi\\
Telefax:    +358-71-225566\\
Telex:      42218 kuy sf (Attn. HELINA'93)

\subsection{SECOND NATIONAL EXPERT SYSTEMS AND DEVELOPMENT WORKSHOP ESADW-93}
{\bf Name:} ESADW-93\\
{\bf Date:} May 2-6, 1993\\
{\bf Venue:} Cairo, Egypt\\
{\bf Hosted:} Expert Systems for Improved Crop Management Project (ESICM)
{\bf Topics:} Knowledge Acquisition Techniques, Knowledge
Representation and Reasoning, Machine Learning, Verification and
Validation of Expert Systems, User interfaces, Expert Systems Tools,
Applications in Agriculture, Engineering, Business, Medicine, and
others, Impact of using Expert Systems on Society Development,
Education, Management, and  others, Interfacing expert systems with
other software \\
{\bf Format for papers:} Three  copies of extended abstract (1000-2000 Words)
should be received by Nov., 1st 1992. The abstract should be
accompanied with a cover page that includes paper title, authors names
and their affiliations, mailing address (and Email), and a maximum of
five keywords. The Abstract should contain motivations for work in the
paper, its significance, the followed methodology and relationship to
current trends and technology in the field of E.S.\\ 
{\bf Proceedings:} At workshop\\
{\bf Deadlines:} extended abstract (1000-2000 words) by Nov. 1 1992.\\
Notification of acceptance by Jan. 1st, 1993.\\
Full paper in camera ready format March 1 1993\\
{\bf Correspondence:}
Dr. Ahmed Rafea\\
Expert Systems for Improved Crop\\
Management Project(EGY/88/024)\\
FAO  Representation\\
11 El Eslah El Zerai St.,\\
P.O. Box 100 - Dokki, Cairo, Egypt,\\
email: esic@ egfrcuvx. bitnet\\
Tel.:    (20)(2) 3611477, (20)(2) 360 47 27\\
Fax.: (20)(2) 360 47 27\\

\section{Groups, Bulletin Boards and Mailing Lists}
\subsection{IFIP WG9.4}

The electronic mailing list of the IFIP Working Group 9.4
'Social Implications of Computers in Developing Countries' is hereby
formally opened for your use. There are 51 names on the list. Our
Newsletter will be distributed by ordinary mail as before; the
electronic mailing list is just an additional medium for those who
have access to it. Maybe we can copy parts of the electronic
discussion in the Newsletter. When you want to send something to *all* the people on the list, address your message to:

    wg9.4@luotsi.uku.fi

If your mailing software does not like the dot in 'wg9.4', use 'wg94'
instead. If you are using a Bitnet computer and do not know how to send
to an Internet address like the one above, try 'wg9.4%luotsi.uku.fi@FIGBOX'.
If you are in U.K., reverse the right-hand side as 'wg9.4@fi.uku.luotsi'.

If you want to be added or removed from the list, or to change your address,
or the like, then *do not* use the address above because then all the
people on the list will get your message.  Instead, send your request to
Mikko Korpela to 'korpela@uku.fi' ('korpela@FINKUO' in Bitnet).

All topics that the participants accept can be discussed; there is no
moderator. However, please remember that some of the participants are
accessed via expensive and slow lines. Keep your message shorter than this
one, and remember that slightly nasty remarks tend to sound like outright
insults when distributed by e-mail.

Mikko Korpela, IFIP WG 9.4 Secretary   \\
Internet: korpela@uku.fi   Bitnet: korpela@FINKUO   Fax: +358-71-225566\\
University of Kuopio, Computing Centre, PL 1627, SF-70211 Kuopio, Finland

\section{Contact Personnel}
Please send contributions to the newsletter to Kathleen King. Send
requests for addition to the mailing list to Mandy Haggith. Email is
the communication method of choice 
(it takes so long to type the gubbins in) but communication through any
medium is welcome, especially if it contains contributions to the
newsletter!

\begin{tabular}{l|l}
{\bf Software library:} & {\bf Newsletter, overall co-ordination,
meetings:}\\ 
Howard Beck &  Kathleen King\\
Artificial Intelligence Applications Institute &  Department of
Artificial Intelligence \\
University of Edinburgh & University of Edinburgh\\
80 South Bridge & 80 South Bridge\\
Edinburgh EH1 1HN & Edinburgh EH1 1HN\\
031 650 2747 & 031 650 2726\\
hab@uk.ac.ed.aiai & kk@uk.ac.ed.aisb\\
\hline
{\bf Contacts and Funding:} & {\bf Literature resource and bibliography:} \\ 
Robert Muetzelfeldt & Ehud Reiter\\
Department of Forestry and Natural Resources &   Department of Artificial Intelligence\\
University of Edinburgh & University of Edinburgh\\
Kings Buildings & 80 South Bridge\\
 Mayfield Road &  Edinburgh\\
Edinburgh   EH9 3JU & EH1 1HN\\
 031 650 5408 & 031 650 2728\\
 R.Muetzelfeldt@uk.ac.edinburgh & reiter@uk.ac.ed.aisb\\
\hline
{\bf Mailing Lists:} &\\ 
Mandy Haggith & \\
Department of Artificial Intelligence &\\
University of Edinburgh &\\
80 South Bridge &\\
Edinburgh &\\
EH1 1HN &\\
031 650 2721\\
hag@uk.ac.ed.aisb &\\
\hline

\end{tabular}

\end{document}



