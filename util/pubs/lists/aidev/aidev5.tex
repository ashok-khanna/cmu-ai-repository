\documentstyle[a4]{article}
\pagestyle{myheadings}
\markboth{Author:kk}{Doc.No. AID/newsletter/5}
\setlength{\textheight}{9in}
\setlength{\topmargin}{0in}
\setlength{\headheight}{.3in}
\setlength{\headsep}{.5in}
\parindent=0pt
\parskip= .35\baselineskip plus .0833333\baselineskip minus .0833333\baselineskip
\begin{document}
Artificial Intelligence for Development\\
Document No: AID/newsletter/number 5\\
Last Modified and by whom: 14/2/92 kk \\
Distributed on: 14/2/92\\
\LARGE
\begin{center} Artificial Intelligence for Development\\
Newsletter Number 5\\February 1992\\

\end{center}
\normalsize
\section*{Contents}
\begin{enumerate}
\item Report on Some AI Initiatives Around the World
\item New project: Representing Indigenous Knowledge
\item Call for information: AIDev Practical Applications
\item News from around the world
\item Conferences
\item Groups, bulletin boards and mailing lists
\item Contact Personnel
\end{enumerate} 

\section{Report on Some AI Initiatives Around the World}
\subsection{India}
With assistance from the United Nations Development Program, the
Indian government introduced a five year Knowledge Based
Computer Systems Project in 1986. Its main drive has been to finance
visits from AI workers from major laboratories around the world to
Indian KBS centres, and reciprocal visits by Indian KBCS workers,
particularly those in the early stages of their careers for training
in major AI centres abroad. The Ministry of Human Resource Development
has also started an Intelligent Systems Project as part of this
initiative. The Intelligent Systems project has multiple institutions
participating throughout the country; a project on Machine Translation at
the Indian Institute of Technology, Kanpur, a Masters Degree in AI and
Robotics being taught at the Central University in Hyderabad,  a
post-graduate diploma programme in KBCS taught at the National
Centre for Software Technology, Bombay and the Indian Institute of
Management, Calcutta and the Indian Institute of Technology, Kharagpur
are involved in projects on intelligent search.  

\subsection{China}
Work in AI began in the early 1960s in China, although it is only since
the 1970s, when computer science departments focussed on the problems
of improvement in production process control that AI really got off
the ground. It is not clear whether there is a national AI program,
but \cite{xinsong83} points to projects in many areas of 
application in AI in China. In the area of Natural Language
Understanding there is  the English-Chinese
Automated Translation (ECTA) system and some other machine translation
projects. Work in Theorem Proving is being undertaken at the Institute
of System Science, Academia Sinica. There has been successful
Expert Systems work in both traditional Chinese medicine (the General
Diagnosis System from the Shanghai Institute of Computing Technology)
and Western medicine, an expert control system for blast furnaces,
and expert assistance for cross-breeding programs for silkworms and
wheat. In education a mathematics training program, based on modelling
human problem solving, has been developed at Jinlin University and
there is robotics research at the Shenyang Institute of Automation.

\subsection{Brazil}
The Rio Scientific Centre in Brazil is responsible for all the
scientific research of IBM Brasil, and is associated with the Latin
American Technology Institute. In Artificial Intelligence, much of
their work focuses on expert systems research, with particular
projects on Knowledge Acquisition (developing a tool for knowledge
acquisition in equipment diagnosis), knowledge base management
(integrating AI and database techniques) and heuristic learning
(developing models for building `self-modifying' diagnostic expert
systems). Some projects are co-operative, such as the heuristic
learning system which is a collaborative project between the Rio
Scientific Centre and the Informatics Centre in Health Care, and
expert systems research in collaboration with a large manufacturing
company, Villares Industries. The Universidade de Campina Grande has a
strong AI team, teaching AI courses at undergraduate and postgraduate
level and with research specialists in expert systems, remote sensing
and numerical algorithms. They also have consultative and
collaborative profiles in industry and medicine. There is currently no
national AI program, although there is a government plan for a three
year national computer science initiative (PROTEM) due to commence in
March 1992, one of whose main thrusts is `Computer Science Theory and
Artificial Intelligence'. 

\subsection{International enterprises}
As well as individual countries' enterprises, there have been some
international initiatives towards encouraging AI for developing
countries. In 1989, the Royal Nepal Academy of Science and Technology
in Kathmandu, Nepal hosted the first International Conference on
Expert Systems for Development (\cite{sadananda89}). A large
binational AI project has been undertaken collaboratively between
Brazil and Argentina for the dissemination of AI expertise, involving
AI training courses, mainly in expert systems techniques. In 1990 the
AI for Development group was set up at the University of Edinburgh
with a global membership. Various
journals, mailing lists and bulletin boards such as VITA (Volunteers
in Technical Assistance), African Technology Forum, Computers in
Africa, Information Technology for Development and AI and Society
carry occasional articles concerning AI for developing countries. The
report of a panel on the use of microcomputers for developing
countries organised in 1988 by the Board on Science and Technology for
International Development (BOSTID) includes many applications of AI
including population programs, design and manufacture, civil
engineering and education (\cite{bostid88}).

\bibliography{phrefs,AIDrefs}
\bibliographystyle{scribe}

\section{New project: Representing Indigenous Knowledge}
  Local people possess a great deal of knowledge about their environment and
 about suitable land management practices.   This knowledge can complement
 external scientific knowledge.   However, there are considerable problems
 in disseminating this indigenous knowledge to others who could benefit from
 it, there are difficulties of integrating it with external scientific
 knowledge, and there is a real danger that much of this knowledge will be
 lost through social changes and the imposition of external solutions.
 
 The aim of this research is to develop computer-based methods for eliciting,
 testing and representing indigenous ecological knowledge, so that it can be
 used for supporting sustainable agroforestry practices.
 
 The work involves collaboration between 6 institutions:
 - The School of Agricultural and Forest Sciences, University of Wales,
   Bangor, is responsible for overall co-ordination and management of 
   the project
 - The Department of Artificial Intelligence, and the Institute of Ecology
   and Resource Management at the University of Edinburgh are developing
   the formal methods for knowledge representation and the software for
   eliciting and reasoning with the indigenous knowledge.
 Each of the following four institutions is providing a Study Fellow who will
 spend most of his or her time in the home country, obtaining and structuring
 knowledge about a particular aspect of agroforestry practice, and some time
 in the UK for training in knowledge-based methods
 - Department of Biology, University of Chiang Mai, Thailand (mixed farming)
 - Faculty of Agriculture, University of Perdeniya, Sri Lanka (home gardens)
 - Division of Forestry and Beekeeping, Tanzania (trees in rangelands)
 - Pakribas Agricultural Centre, Nepal (production of browse for cattle).
  The project is funded by the UK Overseas Development Administration.

More information on this project is available from Mandy Haggith or
Robert Muetzelfeldt.

\section{Call for Information: AIDev Parctical Applications}
Here at the AIDev goup we are trying to set up a resource database, one
of whose elements is a list of all known AI systems which have been,
or are being, developed in or for developing countries (nothing like
having a modest aim!). We feel that such a database, especially if kept
up to date, provides an invaluable resource for those working in this
sort of area. 

If you have any information about any such system, however
fragmentary, we would be very pleased to hear about it. Please contact
Howard Beck through email or any other channel including as many as possible of
the following, and any other information:

Country of development\\
Country of application\\
Current status\\
Trials information\\
Source of funding\\
AI tools or techniques used \\
Implementation language\\
Human language\\
Area of application\\
Associated literature\\
Machine requirements of developed system

\section{News from around the world}
News from Zimbabwe:
The mail connection to the University of Zimbabwe in Harare should now
be `robust'. The mail software is elm + smail3.  The link is 2400 baud
dialup uucp to Rhodes University in South Africa.
The current configuration can support 8 terminals (excluding
the uucp dialup port).  The intention is to provide async terminal
access in various strategic departments, and to provide a terminal
cluster in the computer centre building for general use.
An attempt at setting up Usenet news was abandoned when it was
discovered that a 170mb SCSI disk drive would cost Zdollars24,000-00.

For more information send email to:  postmaster@zimbix.uz.zw \\
or route to postmaster\verb+%+zimbix.uz.zw@quagga.ru.ac.za

There are two other nodes on the campus, hyaena.uz.zw (a 386 Xenix
system) and compsci.uz.zw (an NCR Tower system), but these are
experimental. If direct mail routing fails, you can try sending to:

     postmaster\verb+%+zimbix.uz.zw@quagga.ru.ac.za

\section{Conferences}
\subsection{Expert Systems and Development, 20-23 April 1992, Cairo,
Egypt}
This conference is organised by the Expert Systems for Improved Crop
Management Project (ECSICM).

Specific topics include:

Knowledge acquisition techniques, Knowledge representation and
reasoning, Machine learning, User interfaces, Expert systems tools,
Applications in agriculture, engineering, business, medicine, Impact
of using expert systems on society development, education, management,
Interfacing expert systems with other softwaretools (databases,
graphics, simulation models) 

Deadlines:

Up to 20 pp. manuscript by November 1st 1991\\
Acceptance notices by January 1st 1992\\
Camera ready copy by February 15th 1992

Papers should be sent to:

Dr. Ahmed Rafea\\
Expert Systems for Improved Crop Management project (EGY/88/024)\\
FAO Country Rep. Office\\
P.O. Box 2223, Cairo, Egypt.\\
Tel: (20)(2) 360 47 27\\
Fax: (20)(2) 360 47 27\\
Email: esic@egfrcuvx.bitnet

\subsection{Distributed Computing and Artificial Intelligence, 6-8 May
1992, Jos. Nigeria}
Specific topics include:

Distributed operating systems, Parallel computer systems (parallel
processing, multiprocessorcomputing, image signal processing,
animation and graphics), Fault tolerant systems, Neural networks,
Neuro-computing, Expert Systems, Applied distributed computing,
Databases, Robotics, Natural language systems, Software engineering
environments, Future trends in Distributed Computing and Artificial
Intelligence  

Deadlines:

Submission of 200-400 word abstracts in English, 30th Nov. 1991\\
Notification of acceptance, 15th January 1992\\
Submission of full-length paper, 29th February 1992

Further details from
\begin{verse} 
Dr. Adebayo D. Akinde\\
Dept of Computer Science\\
Obafemi Awolowo University\\
Ile-Ife, Nigeria\\
Fax 01-831210
\end{verse}

\subsection{International Conference CISNA 92, Windhoek, Namibia
6th-8th May 1992}

CALL FOR PAPERS: Information Technology: A vehicle for growth and development
 
Topics: Computers in education, Financial services in a changing era,
The process of software development, Constraints on information
technology - problems and solutions, Technology transfer - is it
happening?, Business efficiency - affected by information technology
or not, Telecommunications, Open systems and international standards,
Employment issues in information technology, Technology trends, Office
automation, Information technology and the environment, Computers in
medicine and health
 
 Abstracts should be submitted to:
\begin{verse} 
     Mr J du Toit\\
     Chairman - Conference organising committee\\
     P O Box 2184\\
     Windhoek\\
     9000\\
     Namibia
 
     Fax +264/61/36518\\
     tel +264/61/34161
\end{verse}
 
 TUTORIALS:

     Parallel to the papers, six tutorials will be given. Abstracts
     for these tutorials with inclusion of equipment needed are also
     invited. A tutorial is intended as an introduction to a specific
     field of general interest.
 
 DEADLINES:

       Abstracts                  : 15 November 1991\\
       Notification of acceptance : 15 December 1991\\
       Full camera ready paper    : 31 March 1992
 
 For abstract/tutorial submission form:
 
     Fax/telephone/write to Mr J du Toit\\
     E-mail Conrad Mueller e-mail address 122cons@wits.witsvma.ac.za
 

\subsection{Toward a Truly Global Network}
A Section of the 36th Annual Meeting of the International Society for
the Systems Sciences.

University of Denver, Denver, colorado, USA, July 12-17, 1992

We wish broad participation, with papers from nuts-and-bolts to
the visionary.  Suitable topics include, but are not restricted
to:

Descriptions of networks, packet radio, satellite communication,
communication protocols, connection options, surveys of the current
state of affairs in global networking, descriptions of current
applications, descriptions of proposed applications, education in a
networked world, education for a networked world, social implications
of a global network, economic implications of a global network, haves
vs. have-nots, politics and funding for a global network, political
implications of a global network, free speech on the global network,
environmental implications of a global network, global networks in the
context of Gaian evolution, depictions of the global network in
science fiction. 

Deadlines:

   Notice of intention to participate, ASAP\\
   Submission of paper, February 15, 1992\\
   Notice of acceptance or rejection, March 15, 1992\\
   Final, camera-ready copy due, May 1, 1992

Send 3 copies of your paper to:

   Professor Larry Press,\\
 CSUDH, 10726 Esther Avenue,\\
Los Angeles,\\ 
CA  90064\\
USA\\
Tel: +(310) 475-6515\\
Fax +(310) 516-3664\\
Internet: lpress@venera.isi.edu
Tel: +468-663-6302.

\subsection{1st African Conference on Research in Computer Science,
Yaounde, Cameroon October 14-20, 1992}

TOPICS: Software engineering,   Parallel Computing, Scientific
Computation, Architecture, Databases, Networks, Computer Vision,
Artificial Intelligence, Programming Languages

INSTRUCTIONS TO AUTHORS

Authors are invited to submit, in French or in English, contributions on
their most recent research results.
5 copies of the full text (20 double-space pages, title and abstract
included) should be sent to the Conference Secretariat, before
February 15, 1992. The front page should include the title of the
communication, the name of the author (or authors) and the full address.
In case of co-authors, clearly indicate the name of the person in charge
of the correspondence with the organisers.
Every submitted paper will be reviewed by the Programme Committee with the
help of experts. The accepted papers will be published for the Conference.
A financial support (travel and sojourn expenses) will be granted to the
authors of the best papers.


Deadlines:

     February 15, 1992: full paper (5 copies)\\
     May 1st, 1992: Notification of acceptance or refusal\\
     June 15, 1992: Distribution of the program\\
     June 30, 1992: Final version

INFORMATION
University of Yaounde\\
Computer Science Department\\
BP  812\\
YAOUNDE (Cameroon)\\
Tel : + 237 22 47 95\\
Fax : + 237 22 13 20

Brigitte KERHERVE\\
Ecole Nationale Superieure des Telecommunications\\
46, rue Barrault\\
75634 PARIS Cedex 13\\
Tel: +(33) 1.45.81.78.81\\
Fax: +(33) 1.45.81.31.19\\
e-mail : kerherve@inf.enst.fr

\subsection{First International Working Conference on Health
Informatics in Africa (Provisional)} 
HELINA'93, 19-23 April 1993, Ile-Ife, Nigeria.

Organized by International Medical Informatics Association (IMIA)
jointly with O.A.U. Teaching Hospitals Complex, Ile-Ife, Nigeria;
Computer Science Department, Obafemi Awolowo University, Ile-Ife,
Nigeria; Computing Centre, University of Kuopio, Finland; and
prospective other co-organizers, co-sponsors and supporters.  

PROGRAMME AND TOPICS

The programme will cover the whole range of existing understanding and
experience on the field of computers in health care in Africa, from
Primary Health Care to hospitals and national planning, from records-
keeping to research and telecommunications. Researchers and reflective
practitioners throughout the continent will be drawn together. The
speakers will be the best experts in their respective topics. The
conference is thus an ideal opportunity for newcomers to get an
overview of the state of the art.
    Each session will start with a tutorial overview of the topic, and
end with a discussion summing up the theme. Practical country cases
from all parts of Africa are encouraged. A conversational and multi-
voiced mood is strived for. The conference also aims to establish a
network of researchers and practitioners who want to cooperate, share
their experiences, and learn from each other in the future also.
    Proceedings will be published by Elsevier North-Holland Publishers
under IMIA sponsorship.
    The official language will be English. If enough interest and
sponsoring will be found, simultaneous interpretation to and from
French will be provided for.

IMPORTANT DATES\\
30 April 1992:      Deadline for letters of intent.\\
31 July 1992:       Deadline for extended abstracts.\\
September 1992:     Letters of acknowledgement sent. Final decision on
                    holding the conference depending on funding.\\
30 November 1992:   Deadline for full texts of papers.\\
19-23 April 1993:   HELINA'93 in Ile-Ife?

Please send all correspondence concerning registration, papers, and
funding to:-

Mikko Korpela/HELINA'93\\
Univ. of Kuopio, Computing Centre\\
P.O.Box 1627\\
SF-70211 Kuopio\\
FINLAND\\
E-mail:     helina@uku.fi\\
Telefax:    +358-71-225566\\
Telex:      42218 kuy sf (Attn. HELINA'93)

\section{Groups, Bulletin Boards and Mailing Lists}
\subsection{Artificial Intelligence in Mexico and Latin America}
1. Cultural/geographical region: Mexico, Latin America

2. Description: a listserver devoted to the exchange of ideas and discussion
of practical applications of AI. It is located at the Instituto Tecnologico
y de Estudios Superiores de Monterrey.

3. Service: listserver (IAMEX-L)

4. E-mail addresses:

-Subscriptions/information: (see additional information below)\\
-Messages to the members: IAMEX-L@tecmtyvm.bitnet;\\
                      or: IAMEX-L@tecmtyvm.mty.itesm.mx\\
-Administrators: J. M. Gomez Puertos (PL500368@tecmtyvm);\\
             or: F. Careaga Sanchez (PL57961@tecmtyvm)

5. Status: ACTIVE

6. Additional information: students and researchers with some experience
the field are eligible for membership. To subscribe, please send e-mail
to one of the administrators asking for the registration form.

\subsection{International Networking Task Force}

Sender: Technology Transfer in International Development, 
\verb+DEVEL-L@AUVM.BITNET+
 
     As a first step toward the establishment of an
``International Networking Task Force'', we decided to create a
mailing list as a forum where our colleagues based in developing
countries could post their requests for help. The subscribers of
the list would either provide direct answers, or forward the
request to other persons able to answer.
     This list will be advertised and distributed to many remote
networks that are linked by expensive telephone lines. We should
therefore use this list strictly for the purposes for which it
was created, limiting activity to the minimum.
 
      At some point, if the traffic increases considerably, we
expect to split the list into several others which are more
"topic oriented". The subscription
process is not automated, partially to make it easier for us to
deal with addresses from developing areas.
 To subscribe, unsubscribe, or for other administrative requests:

   Send mail to intf-request@infoods.mit.edu.  

To post messages to the list\\
   Send mail to intf@infoods.mit.edu
 
Please feel free to circulate the information about the list to
anyone you consider relevant.
 
John C. Klensin, Ph.D.\\
klensin@infoods.mit.edu\\
International Network of Food Data Systems\\
  (A United Nations University project)\\
MIT Room N52-457                       tel: +1 617 253 1355\\
77 Massachusetts Avenue                fax: +1 617 491 6266\\
Cambridge, MA 02139 USA
 
Enzo Puliatti\\
puliatti@acfcluster.nyu.edu\\
United Nations Development Programme\\
One U.N. Plaza 2284                     tel: (212) 906 5426\\
New York, NY 10017                       fax: (212) 906 5892


\section{Contact Personnel}
Please send contributions to the newsletter to Kathleen King. Send
requests for addition to the mailing list to Mandy Haggith. Email is
the communication method of choice 
(it takes so long to type the gubbins in) but communication through any
medium is welcome, especially if it contains contributions to the
newsletter!

\begin{tabular}{l|l}
{\bf Software library:} & {\bf Newsletter, overall co-ordination,
meetings:}\\ 
Howard Beck &  Kathleen King\\
Artificial Intelligence Applications Institute &  Department of
Artificial Intelligence \\
University of Edinburgh & University of Edinburgh\\
80 South Bridge & 80 South Bridge\\
Edinburgh EH1 1HN & Edinburgh EH1 1HN\\
031 650 2747 & 031 650 2726\\
hab@uk.ac.ed.aiai & kk@uk.ac.ed.aisb\\
\hline
{\bf Contacts and Funding:} & {\bf Literature resource and bibliography:} \\ 
Robert Muetzelfeldt & Ehud Reiter\\
Department of Forestry and Natural Resources &   Department of Artificial Intelligence\\
University of Edinburgh & University of Edinburgh\\
Kings Buildings & 80 South Bridge\\
 Mayfield Road &  Edinburgh\\
Edinburgh   EH9 3JU & EH1 1HN\\
 031 650 5408 & 031 650 2728\\
 R.Muetzelfeldt@uk.ac.edinburgh & reiter@uk.ac.ed.aisb\\
\hline
{\bf Mailing Lists:} &\\ 
Mandy Haggith & \\
Department of Artificial Intelligence &\\
University of Edinburgh &\\
80 South Bridge &\\
Edinburgh &\\
EH1 1HN &\\
031 650 2721\\
hag@uk.ac.ed.aisb &\\
\hline

\end{tabular}

\end{document}



