
%% Current version: August 10, 1993

\typeout{Authors! Before beginning to work, please ftp the README file from}
\typeout{the directory /Kluwer/styles/journals at world.std.com}
\typeout{to see if you have the current version of this file!}
\typeout{This version is dated August 10, 1993}
\typeout{\space\space\space\space\space\space\space\space\space}
\typeout{\space\space\space\space\space\space\space\space\space}

%%%%%%%%%%%%%%%%%%%%%%%%%%%%%%%%%%%%%%%%%%%%%%%%%%%%%%%%%%%%%%%%%%%%%%%%%%
%% Template File for Small Journal Format                               %%
%% Kluwer Academic Publishers                                           %%
%%                                                                      %%
%% Prepared by Amy Hendrickson, TeXnology Inc.                          %%
%%                                                                      %%
%% Inquiries to Suzanne M. Rumsey, net address: prod@world.std.com      %%
%%%%%%%%%%%%%%%%%%%%%%%%%%%%%%%%%%%%%%%%%%%%%%%%%%%%%%%%%%%%%%%%%%%%%%%%%%

% Use this for Computer Modern Fonts
\documentstyle{smjrnl}

\begin{document}

%% To be entered at Kluwers: ==>>

\journame{Small Journal Name}
\volnumber{??}
\issuenumber{??}
\issuemonth{??}
\volyear{??}

%%%%  Issue Table of Contents %%%%

%% \begin{issuetoc}
%% \TOCarticle{<article title> }{<author or authors> }{<starting page no.> }
%% \end{issuetoc}

%%%% End of Issue Table of Contents %%%%

%% Individual article commands:

\begin{article}

\authorrunninghead{Author Name or Names}
\titlerunninghead{Article Title}

%\setcounter{page}{275} %% This command is optional. 
                       %% May set page number only for first page in
                       %% issue, if desired.

%% <<== End of commands to be entered at Kluwers 


%%  Authors, start here ==>>

\title{ }

%% Author name and email address
%% Please enter your email address if you have one.
\authors{ }
\email{}

%% Affiliation address
\affil{ }

%% Article editor
\editor{ }

%% Abstract
\abstract{ }

%% Keywords
\keywords{}

\section{ }


%% End matter:

% \acknowledgements

% \appendix

% \notes

% \begin{references}
% \bibitem{xxx}
% \end{references}

%% Do not delete this! ===>>>
\end{article}
\end{document}


% Samples of commands you may use:

\section{ }
\subsection{ }
\subsubsection{ }
Our theory of the structure of the environment has been focused the
structure of living things (arguably, the largest portion of the
objects in the world) because of the aid biology gives in objectively
specifying the organization of these objects.

%% If you want to make a wide equation, precede \begin{equation}
%% with \begin{wideequation} and follow \end{equation} with
%% \end{wideequation}:

% \begin{wideequation}
% \begin{equation}
% (equation)
% \end{equation}
% \end{wideequation}

% In this wide equation the `array' command is used to split 
% the math into two lines, moving the top half to the left
% and the bottom to the right.
% \begin{wideequation}
% \begin{equation}
% \begin{array}{lr}
% \sum_k P(k) \sum_i \sum_y f_i(y|k)^2\\
% &\sum_k P(k) \sum_i \sum_y f_i(y|k)^2
% \sum_k P(k) \sum_i \sum_y f_i(y|k)^2
% \end{array}
% \end{equation}
% \end{wideequation}


% To indent text, use the following commands:
% \begin{itemize}
% \item[] 
% text...
% \end{itemize}


%% Algorithm for exhibiting code. Indent lines with one or more `\ '.
% \begin{algorithm}
% \ start line here
% \ \ indent line here
% \end{algorithm}


% Sample figure
% \begin{figure}[h]
% \vspace*{.5in}
% \caption{This is a figure caption.
% This is a figure caption.
% This is a figure caption.}
% \end{figure}

% Sample table, this kind of table preamble will spread 
% table out to the width of the page:
% \begin{table}[h]
% \caption{This is an example table caption. As you can
% see, it will be as wide as the table that it captions.}
% \begin{tabular*}{\textwidth}{@{\extracolsep{\fill}}lcr}
% \hline
% $\alpha\beta\Gamma\Delta$ One&Two&Three\cr
% \hline
% one&two&three\cr
% one&two&three\cr
% \hline
% \end{tabular*}
% \end{table}

%% Make examples:
% \begin{example}
% text...
% \end{example}

%% Make theorems:
% \begin{proclaim}{Theorem <number>}
% text...
% \end{proclaim}

%% Make proof:
% \begin{proof}
% text...
% \end{proof}

% If proof ends with math, please use \inmathqed at the end of the
% equation:
% \begin{proof}
% ....
% \[
% \alpha\beta\Gamma\Delta\inmathqed
% \]
% \end{proof}

%% Proof with a title. Enter name of proof in square brackets:
% \begin{proof}[Proof of Theorem A.1]
% ...
% \end{proof}

%% Assumption or similar kind of environment
% \begin{demo}{Assumption <number>}
% text...
% \end{demo}

%% End Matter:

%%%%%%%
%% Acknowledgements here

% \acknowledgements
% text...

% Appendices:
% \appendix

%%%%%%%
%% Endnotes

%\notes

%%%%%%%
%% Make references as standard Latex.
%% for example:

%% Trying `cite', \cite{jacobs}, \cite{francis}.

% \begin{references}
% \bibitem{jacobs}Jacobs, E., ``Design Method Optimizes Scanning
% Phased Array,'' Microwaves, April 1982, pp.\ 69--70.

% \bibitem{francis} Francis, M., ``Out-of-band response of array antennas,''
% Antenna Meas.  Tech. Proc., September 28--October 2, 1987, Seattle, p.~14.
% \end{references}

% Or, to make alphabetical references, with no number preceding entries:

% Maude Francis, (Francis, 87) showed important new results
% with array antennas.

% \begin{alphareferences}
% Francis, M., ``Out-of-band response of array 
% antennas,'' Antenna Meas.  Tech. Proc., September 28--October 2,
% 1987, Seattle, p.~14.

% Jacobs, E., ``Design Method Optimizes Scanning
% Phased Array,'' Microwaves, April 1982, pp.\ 69--70.
% \end{alphareferences}

% See smjrnl.doc for documentation on using Bibtex.





