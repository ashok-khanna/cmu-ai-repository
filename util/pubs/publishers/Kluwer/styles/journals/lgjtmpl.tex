
%% Large Size Journal Article Template 
%% Version: June 28, 1993

\documentstyle{lgjrnl}
\begin{document}

%%==================================================================
%% To be entered at Kluwers. Authors start below: ==>>

\journame{??}
\volnumber{??}
\issuenumber{??}
\issuemonth{??}
\volyear{??}

%% For this article:

%\editor{}
%\editors{}

\received{??}
\revised{??}

\authorrunninghead{??}
\titlerunninghead{??}

%\setcounter{page}{5}  %% This command is optional. 
                       %% May set page number only for first page in
                       %% issue, if desired.

%% <<== End of commands to be entered at Kluwers 

%%==================================================================
%%  Authors, start here ==>>

%% May use \\ to start a new line in title
\title{}

\author{}

% Optional: use \email{} for email address, i.e., \email{garcia@cata.spn.edu}
%\email{}

\affil{}
%% i.e., 
%% \affil{Artificial Intelligence Programs, Department of Computer Science, 
%% GSRC 111, University of Georgia, Athens GA 30602}

%% Repeat \author{}\email{}\affil{}
%% for as many authors with separate affiliations as needed.
%% You may also use 
%    \authors{author one, author two, etc.}
%    \email{one email address 
%             or a number of email addresses separated by commas}
%    \affil{same affil.}

%% You may use \\ to start a new line in a group of authors.
%% To acknowledge funding, etc, you may use the \thanks{} command, i.e.,
%% 
% \authors{Miguel Valero-Garc\'ia, Juan J. Navarro, Jos\'e M. LLaber\'ia,\\
% Mateo Valero and Tom\'as Lang\thanks{This work is
% supported by the Ministry of Education of Spain (CICYT TIC 299/89).}}

\begin{abstract}

enter abstract here

\end{abstract}

\keywords{ }

%% \begin{article} is necessary! It starts double columns and is also used to
%% make \thanks work correctly. \end{article} is also necessary! Please be
%% sure to use both commands!

\begin{article}

Text of article here.



% uncomment if appropriate
%\appendix

%\acknowledgements

\endnotes

\begin{references}
\end{references}

%% \bio{Author Name} Text of biography...
%% Repeat as many times as necessary.
\bio{}

\end{article}
\end{document}

(Material below here may be deleted if you do not need it)

Commands that may be used, and quick samples:

Section heads:
\section{}
\subsection{}
\subsubsection{}
\paragraph{}


Double Column commands:

\goodbreak % Added to encourage next column to start here.

End double columns temporarily:
\endtwocolumns

Restart double columns: 
\twocolumns

These commands will let you span both columns.
\spanbothcolumns
   ... (figure, table, algorithm, etc)
\endspanbothcolumns

To print material spanning both columns where it is found in the text:

\spanbothcolumns[h]
   ... (figure, table, algorithm, etc)
\endspanbothcolumns

To send two column wide figure to the top of the page:
\spanbothcolumns
\begin{figure}[t]
\caption{ }
\end{figure}
\endspanbothcolumns

(Same for tables:
To send two column wide table to the top of the page:
\spanbothcolumns
\begin{table}[t]
\caption{ }
\end{table}
\endspanbothcolumns
etc.)

To send two column wide figure to the bottom of the page:
\spanbothcolumns
\begin{figure}[b]
\caption{ }
\end{figure}
\endspanbothcolumns

( Be careful to place a full page figure  after the other
  figures on the page so that the captions are numbered correctly. )
\spanbothcolumns
\begin{figure}[p]
\caption{ }
\end{figure}
\endspanbothcolumns

Single column figures (same for tables, just subtitute `table' for `figure'):

Print here:
\begin{figure}[h]
\caption{}
\end{figure}

Print at top of current column
\begin{figure}[t]
\caption{}
\end{figure}

Print at bottom of current column
\begin{figure}[b]
\caption{}
\end{figure}


%% Math:

\begin{equation}
math
\end{equation}

\begin{wideequation}
\begin{equation}
math
\end{equation}
\end{wideequation}


\[
\begin{splitmath}
top part, sent to the left
\\
bottom part, sent to the right
\end{splitmath}
\]

Or, if you want a numbered split equation:

\begin{equation}
\begin{splitmath}
top part, sent to the left
\\
bottom part, sent to the right, but room left for equation number
\end{splitmath}
\end{equation}

Theorems, Proofs, Examples, etc.:

Proclaim for theorem-type environments:
\begin{proclaim}{(title)}
...
\end{proclaim}

i.e.,
\begin{proclaim}{Theorem 1}
...
\end{proclaim}

For less showy type of environment, use `demo', i.e.,
\begin{demo}{Assumption 1}
...
\end{demo}

Proofs:
\begin{proof}
\end{proof}


If a proof ends with displayed math \verb+\inmathqed+
must be used, i.e.,

\begin{proof}
To prove that...
\[
\alpha\beta\Gamma\Delta\inmathqed
\]
\end{proof}

If you want a proof of something particular, make title in square brackets:
\begin{proof}[Proof of Theorem A.1]
\end{proof}

Example:
\begin{example}
\end{example}

Example also needs to use \inmathqed if it ends in display math, i.e., 

\begin{example}
\[
\alpha\beta\Gamma\Delta\inmathqed
\]
\end{example}


Listing Environments:

bulleted list:
\begin{itemize}
\item Some text here. 

\item Some text here. Some text here. Some text here.  
\end{itemize}


Numbered List, nested lettered and roman numeral lists,

\begin{enumerate}

\item Some text here. 

\item Some text here. 

\begin{enumerate}
\item Some text here. 

\item Some text here. 

\begin{enumerate}
\item
Some text here. 

\item
Some text here. 
\end{enumerate}
\end{enumerate}
\end{enumerate}


To indent text:
\begin{itemize}
\item[] 
text...
\end{itemize}

Indented description list:
\begin{itemize}
\begin{description}
\item[Descriptive Words...] text...

\item[Descriptive Words...] text...
\end{description}

Steps:
\steps
\step
step description text...

\step
step description text...
\endsteps

\boldsteps
\step
step description text...

\step
step description text...
\endboldsteps

Steps with itemized list included:

\steps
\step
step description text...

\step
step description text...

\step 
step description text...

\begin{itemize} 
\item[--] 
Indented text...

\item[--]
Indented text...
\end{itemize}
\endsteps


Tables:

When making the table, please
use \hline at the top of the table, underneath the column headers
and at the end of the table.

You are discouraged from using vertical lines in tables, but
it you must include vertical lines, you must also use 
\savehline instead of \hline or there will be a
gap between the vertical and horizontal lines.
(\hline has been redefined to add some vertical space above and
below it.)

\begin{table}[h]
\caption{A short table caption.}
\begin{tabular}{lcr}
\hline
$\alpha\beta\Gamma\Delta$ One&Two&Three\cr
\hline
one&two&three\cr
one&two&three\cr
\hline
\end{tabular}
\end{table}


This table form will make table as wide as column or page if it
is found in \spanbothcolumns...\endspanbothcolumns.

\begin{tabular*}{\hsize}{@{\extracolsep{\fill}}lcr}
\hline
$\alpha\beta\Gamma\Delta$ One&Two&Three\cr
\hline
one&two&three\cr
one&two&three\cr
\hline
\end{tabular*}

Table notes can be used before \end{table}. \tableterms makes
a little table for listing terms mentioned in the table.

\begin{tablenotes}
These terms are used in the table above:
\tableterms
term & description\\
term & description\\
term & description\\
\endtableterms
\end{tablenotes}
\end{table}

Examples of Programming Code

\begin{codesamp}...\end{codesamp}
Spaces and lines seen on screen will be maintained.
Commands starting with backslash will still work, i.e., \underline{xxx},
or font changes.

 Notice that \begin{codebox}...\end{codebox} can be inserted within
   codesamp, and will be positioned at the same distance from right
   margin as text. codebox needs an argument for the width of the box,
   as in  \begin{codebox}{2.5in} below. Codebox will make a ruled box
   around the code found within.


The command \string is needed to make curly bracket print: i.e., \string{ 

(See sample pages for full example.)
\begin{codesamp}
code...
    \begin{codebox}
      boxed code
    \end{codebox}
\end{codesamp}

Sample code example spanning both columns:
\spanbothcolumns[h]
\begin{codesamp}
	
           <rule-name>: <activation-pattern>$\Rightarrow$
           (if <condition>) (then <action>) (continuation <cont.>)

\end{codesamp}
\endspanbothcolumns


To Illustrate an Algorithm:

\begin{algorithm}...\end{algorithm} are the commands to use
when you want to illustrate an algorithm
with some pseudo code. Math and font changes may be used. 

The command
\bit will produce bold italics in final PostScript version,
boldface in Computer Modern. 

\note{} will position the note on the right margin. 
A backslash followed by a space will provide a 2em space.

\spanbothcolumns
\begin{figure}[t]
\begin{algorithm}
{\bit Evaluate-Single-FOE} ({\bf x$_f$, I$_0$, I$_1$}):
\ {\bf I}+ := {\bf I}$_1$;
\ ($\phi,\theta$) := (0,0);
\ {\it repeat}\note{/*usually only 1 interation required*/}
\ \ (s$_{opt}${\bf E}$_\eta$) := {\bit Optimal-Shift}% 
({\bf I$_0$,I$^+$,I$_0$,x$_f$});
...
\end{algorithm}
\caption{ }
\end{figure}
\endspanbothcolumns

Endnotes:
Footnotes are printed at the end of the article, before references.

\footnote{This is a footnote.} will produce number in text, note
at end of article.

----------------------
Window Description Environment:
(See sample pages for full example)

Start by ending the two columns;
\endtwocolumns

Start environment;
\begin{interface}

\winline{(window number, if desired)}{(window description)}

box will appear around window:
\window
...
\endwindow

User comments: ...

For Help window description:
\winline{Help window:}{}
\window

\endwindow
...
\end{interface}

%% After \end{interface} restart two columns:
\twocolumns
----------------------

\appendix
Appendix text...
(No \chapter{Appendix...} is necessary)


Sample reference:

(See lgjrnl.doc for info on using BibTeX)

\cite{francis}, \cite{jacobs}, will yield: [1], [2],

\begin{references}
\bibitem{francis}Francis, M., ``Out-of-band response of array antennas,''
Antenna Meas. Tech. Proc., September 28--October 2, 1987, Seattle, p.~14.

\bibitem{jacobs}Jacobs, E., ``Design Method Optimizes Scanning
Phased Array,'' Microwaves, April 1982, pp.\ 69--70.
\end{references}


Biographies:

Space will be added for photo when article is in production

\bio{Name}
Bio description.

i.e,
\bio{Bruce Tonn}
is Group Leader of the Intelligent Cognitive Systems
Group of Oak Ridge National Laboratory in Oak Ridge Tennessee. Dr.~Tonn 
conducts research in probability kinematics, automated knowledge
acquisition, and realtime expert systems as well as in applications
of advanced computing to social sciences. He has received degrees
from Stanford, Harvard and Northwestern Universities.










