\documentstyle[11pt]{article}
\setlength{\parindent}{0in}
\setlength{\parskip}{5pt}
\nofiles
\begin{document}
\begin{center}
    { \Large \bf Software Release Agreement } \\
    \vspace{1cm}
    { Licence No. ...... }
    \vspace{1.5cm}
\end{center}

The undersigned, representing the institution identified below and
hereafter referred to as the Licensee, accepts the software and
associated documentation known under the name

\vspace{5mm}
\centerline{\Large \bf PCE version 4}
\vspace{5mm}

hereafter called the work, and agrees to the following conditions set
out below and in Schedule A regarding its use and/or distribution. The
department of Social Science Informatics (SWI) of the University of
Amsterdam, hereafter referenced as the Licensor, grants to the Licensee
a non-exclusive and non-transferrable licence to use the work only for
internal educational, evaluation and research activities.

The Licensee shall not distribute the work or any part thereof to
others or sell products or services based on the work, including
educational and research services, without the express permission of the
Licensor.

\newlength{\tag}
\settowidth{\tag}{Authorised Signature: }
\newlength{\rest}
\setlength{\rest}{\textwidth}
\addtolength{\rest}{-\tag}

\newcommand{\fillin}{\dotfill\mbox{}}
\newcommand{\onlydots}{\mbox{}\fillin}
\newcommand{\next}{\\[7mm]}

\vspace{1cm}
\makebox[\tag][l]{Licensee:}\fillin \next
\parbox[t]{\tag}{Name and address of \\ Institution:}%
\parbox[t]{\rest}{\onlydots \next \onlydots \next \onlydots} \next
\makebox[\tag][l]{Authorised Signature:}\fillin \next
\makebox[\tag][l]{Date:}\fillin \next
\vspace{0.5cm}

For the department of Social Science Informatics~(SWI), University of
Amsterdam, Roetersstraat 15, 1018 WB~~Amsterdam, The Netherlands. 
\vspace{1cm}

\makebox[\tag][l]{Authorised Signature:}\fillin \next
\makebox[\tag][l]{Date:}\fillin
\vspace{1cm}

\begin{center}
    \Large \bf Schedule A: \\
    Conditions of the Software release Agreement
\end{center}
\begin{enumerate}
    \item[Prerequisites]
The licensee is responsible for obtaining any further licence that may
be necessary to provide the computing environment required by the said
work, such as for Unix\footnote{Unix is a trademark of AT\&T} and Prolog
or Lisp.
Rights implied by this Software Release Agreement shall never exceed the
rights implied by such further licence(s) nor shall any rights implied
by such further licence(s) exceed the rights implied by this Software
Release Agreement.
    \item[Limitations on use]
This Software Release Agreement does not permit the Licensee to use the
work or any part thereof for any commercial purposes.
    \item[Non-disclosure]
The licensee shall take all precaution to maintain the confidentiality
of the work; these precautions shall be at least equivalent to those
employed by the receiving organisation to protect its own confidential
information. 
    \item[Non-exclusivity]
The Licensee recognises that the work is released on a non-exclusive
basis and the Licensor shall have the exclusive right to grant licences
to others or to make such other use of the work as it shall desire.
    \item[Credits]
All credits in the work, both in listings and/or documentation, whether
names of individuals or organisations, will be retained in place by the
Licensee. The Licensee will acknowledge in any published copy or in any
other use of the work the authorship of the work and the fact that the
work was developed at the Licensor.
    \item[Product warranty]
The work is released on an ``as is'' basis, and there is no warranty
expressed or  implied as to the functioning, performance or effect on
hardware or other software. The licensee recognises that the Licensor
is not obliged to provide maintenance, consultation or revision of the
work.
    \item[Future releases]
This licence does not entitle the Licensee to further releases of the
work.
    \item[Legality]
All parties will not be bound by any statement other than those included
in this copy of the Contract and its accompanying Schedule and this
Contract shall be governed by the Dutch law.
\end{enumerate}

\end{document}
