 
To: welty@cs.rpi.edu (confirm)
Date: Fri, 30 Jul 1993 17:03:30 UTC
From: Ramon Lopez de Mantaras <mantaras@ceab.es>
Subject: CFP: Uncertainty in AI 


\documentstyle{article}
\setlength{\parskip}{0.10in}
\setlength{\parindent}{0in}
\setlength{\textheight}{10.0in}
\setlength{\textwidth}{6.5in}
\setlength{\topmargin}{-1in}
\setlength{\oddsidemargin}{0.0in}
\setlength{\columnsep}{0.3in}
%\renewcommand{\baselinestretch}{1.1}
\newcommand{\comment}[1]{}

\def\thepage{}

\begin{document}

\twocolumn[
\protect{
\begin{center}
{\LARGE{\bf Tenth Annual Conference on}}\\*[\medskipamount]
{\Huge{\bf Uncertainty in Artificial Intelligence}}\\*[\medskipamount]
{\Large July 29-31, 1994, Seattle, Washington}\\*[\smallskipamount]
{\Large CALL FOR PAPERS}
\medskip
\end{center}
}
]

Reasoning under uncertainty is pervasive in all areas of Artificial
Intelligence. The Uncertainty in AI conference is the major forum for
advances in the theory and practice of reasoning under uncertainty.
We are seeking contributions both from researchers interested in
advancing the technology and from practitioners who are using
uncertainty techniques in applications.

The tenth annual Conference on Uncertainty in Artificial Intelligence
will be devoted to methods for reasoning under uncertainty as applied
to problems in artificial intelligence.  The conference's scope covers
the full range of approaches to automated and interactive reasoning
and decision making under uncertainty, including both qualitative and
numeric methods.

We seek papers on fundamental theoretical issues, on representational
issues, on computational techniques and on applications of uncertain
reasoning, using traditional and alternative paradigms of uncertain
reasoning.  Topics of interest include (but are not limited to):

%\begin{description}

{\bf Methods and Techniques}

foundations of uncertainty concepts,
representation languages for uncertain knowledge,
knowledge acquisition,
construction of uncertainty models from data,
uncertainty in machine learning,
automated planning and acting,
uncertainty in ill-defined environments,
decision making under uncertainty,
algorithms for uncertain inference,
empirical studies of reasoning strategies,
pooling of uncertain evidence,
belief updating and inconsistency handling,
summarization of uncertain information, and
control of reasoning and real-time architectures.

{\bf %real-world
 Applications}

 Questions of particular interest include:\\
Why was it necessary to represent uncertainty in your domain?
What kind of uncertainties does your application address?
Why did you decide to use your particular uncertainty formalism?
What theoretical problems, if any, did you encounter?
What practical problems did you encounter?
Did users of your system find the results or recommendations useful?
Did your system lead to improvements in reasoning
    or decision making?
What methods were used to validate the effectiveness of the systems?
What did you learn about what was or was not effective in your domain?
%\end{description}

Papers will be refereed for originality, significance, technical
soundness, and clarity of exposition. Application papers will be
judged according to criteria appropriate for application papers, such
as those related to the questions above.  Papers may be accepted for
presentation in plenary or poster sessions.  Some key applications
oriented work may be presented both in a plenary session and in a
poster session where more technical details can be discussed.  All
accepted papers will be included in the published proceedings.
Outstanding student papers may be selected for special distinction.

{\bf Submission of Papers}

Five copies of complete papers (hardcopy only) should be sent to one
of the Program Co-Chairs by {\bf February 1, 1994}.  The first page
should include a descriptive title, the names, addresses (regular mail
and email), and student status of all authors, a brief abstract, and
salient keywords or other topic indicators. To aid in finding
appropriate reviewers, the title, abstract and keywords should be
e-mailed to {\sf uai94@cs.ubc.ca}. Acceptance notices will be sent by
March 31, 1994.  Final camera-ready papers, incorporating reviewers'
suggestions, will be due approximately four weeks later.  There will
be an eight-page limit on proceedings papers, with one extra page
available for a fee.

{\bf Program Co-Chairs (paper submissions):}

Ramon L\'opez de M\'antaras\\
Artificial Intelligence Research Institute, CSIC\\
%Cami de Santa Barbara\\
17300 Blanes, Spain\\
Tel: +34-72-336101,
Fax: +34-72-337806\\
e-mail: mantaras@ceab.es

David Poole,\\
Department of Computer Science,\\
2366 Main Mall, Room 201,\\
University of British Columbia,\\
Vancouver, B.C., Canada V6T 1Z4\\
Tel: +1 (604) 822-6254,
Fax:  +1 (604) 822-5485\\
email: poole@cs.ubc.ca


{\bf General Chair (conference inquiries):}

David Heckerman\\
One Microsoft Way\\
Building 9S/1024\\
Redmond, WA 98052-6399\\
Tel: (206) 936-2662, Fax: (206) 644-1899\\
email: heckerma@microsoft.com


\end{document}

----------------------------- End of body part 2
