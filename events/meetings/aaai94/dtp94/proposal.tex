%%%%%%%%%%%%%%%%%%%%%%%%
%%%%%%%
%%%%%%% Still to do:  
%%%%%%% --More participant names, esp from outside AI.  
%%%%%%% --Recent conferences/workshops outside of AI
%%%%%%% --Complete ``Pollock'' name
%%%%%%% --Possible fourth organizer
%%%%%%%
%%%%%%%%%%%%%%%%%%%%%%%%

=====================
1994 Spring Symposium Proposal
=====================

Title:  Decision-Theoretic Planning

=====================

Organizing Committee:

   Steve Hanks---University of Washington (chair)
   Stuart Russell---University of California, Berkeley
   Michael Wellman---University of Michigan

=====================

Description:

Both AI planning and decision theory are devoted to the problem 
of how an agent chooses a good course of action, based on information 
about the world, about the agent's capabilities, and about 
the agent's preferences. 

The AI community has concentrated on the task of synthesis: given a 
description of a situation, schematic descriptions of actions and 
methods for composing them, and some objectives, generate a course 
of action that furthers those objectives.   Classical planning algorithms
have for the most part assumed that the agent has perfect information 
about and control over the world, and that the objectives are described 
by a symbolic goal state that either is or is not satisfied.

These simplifying assumptions conspire to make it difficult to reason about
tradeoffs in the planning process, in particular tradeoffs involving the
relative likelihood and desirability of possible plan outcomes.  Under the
classical assumptions, both plan success and plan quality are
all-or-nothing propositions.

Decision theory provides a language for expressing richer notions of
success and quality, along with a normative criterion for making tradeoffs
between the likelihood of achieving an objective and the cost of 
doing so. Decision theory lacks a computational model of how to generate 
those plans in the first place, however, thus does not address the synthesis 
task.

This symposium aims to unify current lines of research in the AI planning
community with research in related disciplines---decision analysis,
economics, control theory---by exploring how the richer constructs offered
by the decision-theoretic methodology for describing preference and
uncertainty can be applied to the problem of plan synthesis.  Although AI
and decision theory have sometimes been viewed as competing approaches, a
growing number of researchers have begun to appreciate their complementary
character, and are starting to address the challenge of integrating these
ideas.

The call for participation will solicit contributions pertaining to all
aspects of decision-theoretic planning, particularly those making
substantive connections between the two fields.  Topics will include, but
are not limited to:

 -- Representing and reasoning about preferences
      How can the notion of planning goals be extended to 
      richer utility models?  How do we represent concepts such as
      partial goal satisfaction, the cost of consuming resources, 
      multiple objectives, and so on?

 -- Uncertain effects of actions 
      How can causal or action models be extended to incorporate richer models
      of change, taking into account uncertainty about the state of the world,
      the effects of actions, and exogenous events?  How do we cope with the
      complexity of conditional planning?

 -- Model construction
      How can representations and techniques from the decision sciences 
      and other disciplines, e.g., graphical decision models such as 
      influence diagrams and solution techniques such as policy 
      iteration, be applied to the problem of plan generation?  In
      particular, how can we exploit these technologies without unduly
      restricting the expressiveness of our representation for actions and
      their effects?

 -- Specialized problems and representations
      What is the relationship between decision-theoretic techniques
      for path- and motion-planning problems that generally describe 
      their state spaces and operators numerically and the problems 
      involving symbolic state spaces and operators more typically 
      addressed by AI planning research?

 -- Decision-theoretic meta-level reasoning
      How can decision-theoretic criteria be applied to the 
      problem of controlling the reasoning of an agent so that 
      it behaves rationally without (necessarily) using 
      decision-theoretic calculations to make its decisions. 
      Possible techniques include the use of decision theory 
      to control the plan-generation process itself, and 
      using the methodology to pre-compile rational behaviors.

Various groups in AI, in the decision sciences, in control 
theory, and in economics have been pursuing research efforts 
using the concepts and techniques underlying decision-theoretic 
planning.  The efforts have varied in the issues they consider
important and in the simplifying assumptions they make. 
We want to provide a forum to explore these differences 
and arrive at a better understanding of the common enterprise.

===============================

Evidence of Interest:

We have observed an increased interest in the AI planning community over
the past several years in issues of uncertainty and partial goal
satisfaction, as well as a renewed openness to consider alternative
paradigms such as decision theory.  At the same time, researchers in
uncertain reasoning and the decision sciences have begun to focus far more
directly on computational and representational issues salient to plan
generation.  

  -- Several recent workshops and conferences have had 
     panels on this topic:
       *  UAI-92 (Conference on Uncertainty in AI)
       *  AAAI-91 workshop on knowledge-based construction of 
          probabilistic and decision models
       *  1990 Spring Symposium on Planning in Uncertain... Domains
       *  1989 Spring Symposium on AI and Limited Rationality
  -- Recent book by Dean and Wellman directly addresses this area.
  -- Eight relevant papers in AIPS 92, as well as some in recent
     NCAIs and other AI conferences and workshops address the topic.
  -- (D)ARPA/Rome Lab Planning Initiative has an explicit component in
     decision-theoretic planning.
  -- A significant number of papers in 

===============================

Possible participants:  

  Fahiem Bacchus, Waterloo
  Jack Breese, Microsoft
  Tom Dean, Brown
  Ed Durfee, Michigan
  Keiji Kanazawa, UBC
  Mark Boddy, Honeywell
  Peter Haddawy, UWM
  Jon Doyle, MIT
  Eric Horvitz, Microsoft
  Ross Shachter, Stanford
  Max Henrion, CMU
  Mark Peot, Rockwell
  Marie desJardins, SRI
  Paul Lehner, GMU
  Bruce D'Ambrosio, Oregon State
  Ron Loui, WashU
  Greg Provan, UPenn
  Dave Smith, Rockwell
  Josh Tenenberg, Indiana
  James Allen, Rochester
  Nat Martin, Rochester
  Reid Simmons, CMU
  Pollock                   ????????????/
  Piotr Gmytrasiewicz, Hebrew University
  Jeff Rosenschin, Hebrew University
  Lonnie Chrisman, CMU
  Peter Ramadge, Princeton
  Pravin Varaiya, UCB
  Ron Howard, Stanford
  Anil Nerode, Cornell
  Wolf Kohn, Cornell


  


