From crabapple.srv.cs.cmu.edu!fs7.ece.cmu.edu!europa.eng.gtefsd.com!howland.reston.ans.net!xlink.net!gmd.de!mltsparc1!wrobel Sat Aug 14 19:59:07 EDT 1993
Article: 8317 of comp.lang.prolog
Xref: crabapple.srv.cs.cmu.edu comp.ai:18356 comp.object.logic:64 comp.lang.prolog:8317
Newsgroups: comp.ai,comp.object.logic,comp.lang.prolog
Path: crabapple.srv.cs.cmu.edu!fs7.ece.cmu.edu!europa.eng.gtefsd.com!howland.reston.ans.net!xlink.net!gmd.de!mltsparc1!wrobel
From: wrobel@mltsparc1.gmd.de (Stefan Wrobel)
Subject: CfP Int. WS on Inductive Logic Programming 1994 (ILP94) LaTeX
Message-ID: <wrobel.745350180@gmd.de>
Sender: news@gmd.de (USENET News)
Nntp-Posting-Host: mltsparc1
Organization: GMD, Sankt Augustin, Germany
Date: Sat, 14 Aug 1993 17:43:00 GMT
Lines: 156


\documentstyle[11pt]{article}
\title{The Fourth International Workshop on 
Inductive Logic Programming (ILP94)}
\date{Announcement and First Call for Papers}
\author{September 12 -- 14, 1994\\
Bad Honnef/Bonn, Germany}
\begin{document}
\maketitle
\subsection*{General Information}

Originating from the intersection of Machine Learning and Logic
Programming, Inductive Logic Programming (ILP) is an important and
rapidly developing field that focuses on theory, methods, and
applications of learning in relational, first-order logic formalisms.
ILP94 is the fourth in a series of international workshops designed to
bring together developers and users of ILP in a format that allows a
detailed exchange of ideas and discussions.  Reflecting the growing
maturity of the field, ILP94 for the first time will offer a systems
and application exhibit as an opportunity to demonstrate the practical
results and capabilities of ILP.

\subsection*{Submission of papers}

Reflecting the broadening scope of the field, ILP94 invites papers
covering on the three main aspects of ILP, namely inductive data
analysis and learning in first-order formalisms, inductive synthesis
of non-trivial logic programs from examples, and inductive tools for
software engineering.  Possible topics include, but are not restricted
to:
% to be used inside of tabular environments for 12pt font
\newcommand{\bulletbox}[2]{\parbox{#1}{\begin{list}{$\bullet$}%
{\setlength{\leftmargin}{10pt}
\setlength{\labelsep}{4pt}}\item #2\end{list}
\vspace{-5.5pt}}}
\begin{quote}
\begin{tabular}{ll}
\bulletbox{6cm}{complexity of learning in logical formalisms} &
\bulletbox{6cm}{relationships between ILP and neighboring areas}\\
$\bullet$ higher-order learning & 
$\bullet$ predicate invention\\
$\bullet$ learning of integrity constraints &
$\bullet$ theory revision and restructuring\\
$\bullet$ multiple predicate learning &
$\bullet$ learning in relational formalisms\\
$\bullet$ handling of noise &
$\bullet$ declarative bias\\
$\bullet$ architectures for ILP &
$\bullet$ comparative analyses of ILP methods\\
$\bullet$ application discussions
\end{tabular}
\end{quote}%\end{list}
Ideally, papers should fit into one of the following categories:
\begin{description}
\item[Theory.] Theory papers prove results about a new or known
ILP problem or method, discuss the relationship with neighboring
fields, or present a unified analysis of several methods.
\item[Methods.] Method papers present details of new algorithms,
ideally including theoretical and complexity analysis, and empirical
results on important applications.  Ideally, a method paper would be
accompanied by a system demo.
\item[Applications.] Application papers describe one or more real-life
ILP applications in detail, justifying the use of ILP techniques, and
giving a reproducible presentation of experiments and results.
Ideally, an application paper would be accompanied by an application
demo.
\end{description}
Please submit {\em four paper copies} of your paper to the workshop chair
\begin{quote}
Stefan Wrobel\\
GMD, I3.KI\\
Schlo\ss\ Birlinghoven\\
53757 Sankt Augustin, Germany.\\
E-Mail: ilp-94@gmd.de\\
Fax: +49/2241/14-2889  Tel: +49/2241/14-2670
\end{quote}
to be received on or before {\bf May 31, 1994}.  There is no fixed page limit on
submissions, but length should be reasonable and adequate for the
topic.  Please use LaTeX if at all possible.  Authors will be notified of
acceptance or rejection until {\bf July 15, 1994}, and camery-ready copy will
be due on {\bf August 9, 1994}.
\subsection*{Program Committee}
\begin{quote}
\begin{tabular}{ll}
Francesco Bergadano (Italy) \hspace*{1.5cm} &
Ivan Bratko (Slovenia)\\
Wray Buntine (USA) &
William W. Cohen (USA)\\
Luc de Raedt (Belgium) &
Koichi Furukawa (Japan)\\
J\"org-Uwe Kietz (Germany) &
Nada Lavra\v{c} (Slovenia)\\
Stan Matwin (Canada) &
Stephen Muggleton (UK)\\
C\'eline Rouveirol (France) &
Claude Sammut (Australia)
\end{tabular}
\end{quote}
\subsection*{Proceedings}
To keep submission dates close to the workshop, accepted papers will
be published as a GMD technical report to be distributed at the
workshop and officially available to others from GMD afterwards.
Publication of an edited book is planned for after the workshop.
\subsection*{Systems and Applications Exhibition}
ILP94 offers participants an opportunity to demonstrate their systems
and/or applications.  Please announce your intention to demo to the
conference office until {\bf August 1, 1994}, specifying precisely
what type of hardware and software you need.
\subsection*{Location}
ILP94 will take place in Bad Honnef, a small resort town close to Bonn
in the Rhine valley and adjacent to the Siebengebirge nature park.
Participants will be able to take advantage of Bad Honnef's vicinity
to medieval castles and of the new wine season that starts at the time
of the workshop.
\subsection*{Registration and Conference Office}
Please address all correspondence regarding registration to:
\begin{quote}
Christine Harms\\
ILP94\\
c/o GMD\\
Schlo\ss\ Birlinghoven\\
53757 Sankt Augustin, Germany\\
Tel. +49/2241 14-2473, Fax +49/2241 14-2472 or 2618\\
E-Mail ilp-94@gmd.de
\end{quote}
If you send (preferably by E-Mail) the following information to
Christine Harms, you will be sent a complete registration brochure as
soon as it is available:
\begin{quote}
Last name:\\
First name:\\
Institution:\\
Zip code, city:\\
Country:\\
E-Mail:\\
Fax:\\
Intend to submit a paper?\\
\end{quote}
\subsection*{Important Dates}
\begin{quote}
\begin{tabular}{ll}
Paper submission deadline:&{\bf May 31, 1994}\\
Notification of acceptance:&{\bf July 15, 1994}\\
Demo requests:&{\bf August 1, 1994}\\
Camera-ready copy due:&{\bf August 9, 1994}\\
Early registration:&{\bf August 9, 1994}\\
Workshop:&{\bf September 12 -- 14, 1994}
\end{tabular}
\end{quote}
\end{document}

--
--------------------------------------------------------------
Stefan Wrobel,
GMD (German National Research Center for Computer Science)
FIT.KI (Artificial Intelligence Research Division)


