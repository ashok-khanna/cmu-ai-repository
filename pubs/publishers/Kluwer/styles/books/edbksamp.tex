
%% February 8, 1993

%% Sample File, for author of article to appear in edited book
%% to be published by Kluwer Academic Publishers.

%% Comments included.

%%%%%%%%%%%%%%%%%%%%%%%%%%%%%%%%%%%%%%%%%%
% Choose and uncomment one of the next two commands, 
% depending on whether you are
% using PostScript or Computer Modern fonts:

%\documentstyle[psfonts,editedbk]{kluwerbk}
\documentstyle[editedbk]{kluwerbk}

\begin{document}

%%   You may use \\ to break lines in title, chapter title, and section heads.

%% The following section between arrows is needed only when typesetting
%% a complete book: ======>
\title{Qualitative\\ Motion\\ Understanding}
\titlepage

\author{Wilhelm BURGER}
\affil{Johannes Kepler University}
\location{Linz, Austria}

\author{Bir BHANU}
\affil{University of California}
\location{Riverside, California, USA}

%%   \title, \author, \affil, and \location should be entered before
%%    calling for \titlepage and \authortitlepage

\authortitlepage


\tableofcontents

%\listoffigures

%\listoftables

\begin{contributorpage}
 William B. Baringer
Dept. of EECS
UC Berkeley

Barry Boes
327 Arlington Circle
Ridgeland, MS 39157
\end{contributorpage}


% End of commands needed for entire book <======


\TOCauthor{Lotfi A. Zadeh, Sarah Smithe, Sanders McKinley, 
Sally Ann Moe, Charles Ting and Wu Leung}

\articleauthor{Lotfi A. Zadeh, Sarah Smithe, 
Sanders McKinley*, Sally Ann Moe, 
Charles Ting** and Wu Leung}

\articleaffil{Computer Science Division, Department of EECS
University of California, Berkeley, California 94720
\\
* UC San Diego, San Diego, California
** University of Texas at Austin
USA
}

\articletitle{Framework for Qualitative\\ 
Motion Understanding}

%% To make a shortened version of title for running head use \shortenedtitle;
\shortenedtitle{Framework for Motion Understanding}

\begin{abstract}
The conventional approach to knowledge representation, e.g., semantic
networks, frames, predicate calculus and Prolog, are based on bivalent
logic.
\end{abstract}


\section{First Section}
The images that are acquired by a camera mounted on a moving vehicle 
are subject to continuous change. 
This is normal text. This is normal text. This is normal text.
This is normal text. This is normal text. \label{thistest}
(((\ref{thistest}))).



\subsection{First Subsection}
Here is our first subsection.
This is normal text. This is normal text. This is normal text.
This is normal text. This is normal text. This is normal text.
This is normal text. This is normal text. This is normal text.
This is normal text. This is normal text. This is normal text.


\subsubsection{First Subsubsection}
This is normal text.  This is normal text. This is normal text.
This is normal text. This is normal text. This is normal text.
This is normal text. This is normal text. This is normal text.
This is normal text. This is normal text. This is normal text.


%% Example of broken lines in subsubsection

\subsubsection{This is a\\ Subsubsection}
This is normal text. This is normal text. This is normal text. 
This is normal text. This is normal text. This is normal text.
This is normal text. This is normal text. This is normal text. 
This is normal text.
\begin{equation}
\alpha\beta\Gamma\Delta\left\{\sum\right\}
\end{equation}

%% Default level will produce square bullets

\begin{itemize}
\item
This is a sample list item. This is a sample list item.
This is a sample list item. This is a sample list item.
This is a sample list item. This is a sample list item.
This is a sample list item. This is a sample list item.
This is a sample list item.

\item
This is a sample list item. This is a sample list item.
This is a sample list item. This is a sample list item.
This is a sample list item. This is a sample list item.
This is a sample list item. This is a sample list item.
This is a sample list item.
\end{itemize}

\section {All the Things that can be Done with Figure Captions}

\begin{figure}[h]
\caption{This is the figure caption.
This is the figure caption. This is the figure caption.
This is the figure caption. This is the figure caption. 
$\alpha\beta\Gamma\Delta\sum_{123}^{345}$
This is the figure caption. This is the figure caption.\label{Firstfig}}
\end{figure}

%% Double captions:
\begin{figure}[h]
\dblcaption{This caption will go on the left side of
the page. It is the first of two side-by-side captions.}
{This caption will go on the right side of
the page. It is the second of two side-by-side captions.}
\end{figure}

%% For continued caption. Same figure number used as for last caption.
\begin{figure}[h]
\contcaption{This is a continued caption.}
\end{figure}

\begin{figure}[h]
\contcaption{This is another continued caption.}
\end{figure}

%% Not a continued caption, new figure number used.

\begin{figure}[h]
\caption{This caption is not continued so it has a new caption number.
\label{secondfig}}
\end{figure}


%% To make narrow caption:

\begin{figure}[h]
\narrowcaption{This is a narrow caption.
This is a narrow caption. This is a narrow caption.
This is a narrow caption. This is a narrow caption.
This is a narrow caption. This is a narrow caption.
This is a narrow caption. This is a narrow caption.
This is a narrow caption. This is a narrow caption.}
\end{figure}


%% To make narrow continued caption:

\begin{figure}[h]
\narrowcontcaption{This is a narrow continued caption.
This is a narrow continued caption. This is a narrow continued caption.
This is a narrow continued caption. This is a narrow continued caption.
This is a narrow continued caption. This is a narrow continued caption.
This is a narrow continued caption. This is a narrow continued caption.}
\end{figure}


%% To use boxed code example in figure. \begin{codebox} needs a dimension
%% for its argument. Here the dimension is \textwidth, the width of the
%% printed page, but any dimension can be used, i.e., 2in.
%% Blank space at the beginning of the line will be preserved.
%% Lines will be printed as seen on the screen. 
%% Notice that to print a curly bracket, precede it with \string, i.e.,
%% `\string{'     or  `\string}'

\begin{figure}[h]
\begin{codebox}{\textwidth}
DO*(j=(l+1), p)\string{
          float t; int i;
          t=0.0
              DO(i=1, n)\string{
                  t=t+col[i]
\string}
\end{codebox}
\caption{This is an example of some programming code.}
\end{figure}



%% This is how to make a figure caption to be turned sideways on page.
\begin{widefigure}
\vspace*{2in}
\caption{This is a wide figure caption.
It is meant to be printed in landscape mode (sideways).
This page should be turned sideways when the driver program
is used to translate the .dvi file to the file that is
sent to the printer.}
\end{widefigure}


\section{The Same Things Can be Done\\ with Table Captions}

\begin{table}[h]
\begin{center}
\begin{tabular}{||c||c||}
\hline
  %% On the next line is an example of how to get extra vertical space in
  %% a line: Use a \vrule with width 0pt and the height or depth that you
  %% want.
\bf Cell\vrule height 14pt width 0pt depth 4pt
&\bf Time (sec.)\cr
\hline
\hline
1&432.22\vrule height 12pt width0pt\cr
%%
%% On the next line, see how to line up numbers aligned on their decimal point
2&\phantom{3}32.32\cr
3&\phantom{33}2.32\cr
\hline
\end{tabular}
\end{center}
\caption{Here is the first table caption.\label{tabcaption}}
\end{table}


\begin{table}[h]
\contcaption{This is a continued caption.}
\end{table}

\begin{table}[h]
\contcaption{This is another continued caption.}
\end{table}


\begin{table}[h]
\caption{This caption is not continued so it has a new caption number.}
\end{table}

\begin{table}[h]
\narrowcaption{This is a narrow table caption.
This is a narrow table caption.
This is a narrow table caption.
This is a narrow table caption.
This is a narrow table caption.
This is a narrow table caption.
This is a narrow table caption.
This is a narrow table caption.
This is a narrow table caption.}
\end{table}


\begin{table}[h]
\narrowcontcaption{This is a narrow continued table caption.
This is a narrow continued table caption.
This is a narrow continued table caption.
This is a narrow continued table caption.
This is a narrow continued table caption.
This is a narrow continued table caption.
This is a narrow continued table caption.
This is a narrow table caption.}
\end{table}

\begin{table}
\caption{This is not a continued table and so it has a new number.}
\end{table}


%% This is how to make a table caption to be turned sideways on page.
\begin{widetable}
\vspace*{2in}
\caption{This is a wide table caption.
It is meant to be printed in landscape mode (sideways).
This page should be turned sideways when the driver program
is used to translate the .dvi file to the file that is
sent to the printer.}
\end{widetable}


\section{An Example of Some Programming Code}
When you want to demonstrate some programming code, these are
the commands to use. Lines will be preserved as you see them
on the screen, as will spaces at the beginning of the line.


% Notice that to produce printed `{' brackets, precede them with \string

% Notice that \begin{codebox}...\end{codebox} can be inserted within
%   codesamp, and will be positioned at the same distance from right
%   margin as text. codebox needs an argument for the width of the box,
%   as in  \begin{codebox}{2.5in} below.

\begin{codesamp}
sqrdc(a, n)(a, qraux)\string{
  \underline{DARRAY float[180] a[180];}
  float qraux[180], col[180], nrmxl,t;
  int n,i,j,k,l;
  DO(1=0, n)\string{
         \underline{ALIGN*(i=1, n) col[i]=a[l][i];}
         \begin{codebox}{2.5in}
         init*\string{ nrmxl=0.0;\string}
         DO*(i=l, n)\string{
           nrmxl += col[i]*col[i];\string}
         combine*\string{nrmxl;\string}
         \end{codebox}
         nmxl=sqrt(nrmxl);
         if (nrmxl != 0.00)\string{
            if (col[1]=1.0+col[1];
\newpage
%% You may need to insert a \newpage to start a new page to keep the
%% code and box on the same page.


            \begin{codebox}{1.5in}
            DO*(j=(l+1), p)\string{
              float t; int i;
              t=0.0
              DO(i=1, n)\string{
                  t=t+col[i]
            \string}
            \end{codebox}
            qraux[1]=col[1]
            col[1]= -nrmxl;
            \underline{ALIGN*(i=1, n)a[1][i]=col[i];}
        \string}
   \string}
\string}
\end{codesamp}

\subsubsection{To Illustrate an Algorithm}

This is the command to use when you want to illustrate an algorithm
with some pseudo code. A backslash followed with a space will
indent the line. Every line will be printed
as it is seen on the screen. Blank lines will be preserved.
Math and font changes may be used. 

%% \bit will produce bold italics if you are using PostScript fonts, 
%% boldface in Computer Modern.

\begin{algorithm}
{\bit Evaluate-Single-FOE} ({\bf x$_f$, I$_0$, I$_1$}):
\ {\bf I}+ := {\bf I}$_1$;
\ ($\phi,\theta$) := (0,0);
\ {\it repeat}\note{/*usually only 1 interation required*/}
\ \ (s$_{opt}${\bf E}$_\eta$) := {\bit Optimal-Shift} ({\bf I$_0$,I$^+$,I$_0$,x$_f$});
\ \ ($\phi^+$, $\theta^+$) := {\bit Equivalent-Rotation} ({\bf s}$_{opt}$);
\ \ ($\phi$, $\theta$) := ($\phi$, $\theta$) + ($\phi^+$, $\theta^+$);
\ \ {\bf I}$^+$:= {\bit Derotate-Image} ({\bf I}$_1$, $\phi$, $\theta$);
\ \ {\it until} ($\|\phi^+\|\leq\phi_{max}$ \& $\|\theta^+\|\leq\theta_{max}$);
\ {\it return} ({\bf I}$^+$, $\phi$, $\theta$, E$_\eta$).

End pseudo-code.
\end{algorithm}


\appendix

%% After the \appendix command, \chapter, \section, equation numbering
%% and caption numbering will all be adjusted to work correctly as appendix.

\chapter{This is a Sample\\ Appendix}
\section{First Appendix Section}
Some text.

%% Appendix without a Title
\chapter{} %% Appendix 
\section{First Section in New Appendix}
Some text.


\begin{figure}[h]
\caption{This is the caption for a figure found in
an appendix.}
\end{figure}


\begin{table}[h]
\caption{This is the caption for a table found in
an appendix.}
\end{table}

\begin{equation}
\alpha\beta\Gamma\Delta\left\{\sum\right\}\end{equation}


\acks
This is an optional section, that may be used for acknowledgements.


Here is a citation from the reference section: \cite{jacobs}.

%% The terms below that are arguments of \bibitem can be cited in
%% the text above with \cite, i.e., \cite{jacobs}, which will
%% produce the appropriate reference number, [1] in this case.

\references

\bibitem{jacobs}Jacobs, E., ``Design Method Optimizes Scanning
Phased Array,'' Microwaves, April 1982, pp.\ 69--70.

\bibitem{yaw}Yaw, D. F., ``Antenna Radar Cross Section,'' Microwave
Journal, September 1984, p.\ 197.

\bibitem{francis} Francis, M., ``Out-of-band response of array antennas,''
Antenna Meas.  Tech. Proc., September 28--October 2, 1987, Seattle, p.~14.

\bibitem{shrank6/86} Shrank, H., ``Comparison of low sidelobe 
distributions,'' IEEE AP
Newsletter, Jun 1986, pp.~22-23.

\bibitem{shrank8/86} Shrank, H., ``Gain factor vs 
sidelobe level for circular Taylor
with optimum n-bar,'' IEEE AP Newsletter, Aug 1986.

\endreferences

\end{document}








