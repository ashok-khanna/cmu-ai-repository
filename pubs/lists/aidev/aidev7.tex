\documentstyle[a4]{article}
\pagestyle{myheadings}
\markboth{Author:kk}{Doc.No. AID/newsletter/4}
\setlength{\textheight}{9in}
\setlength{\topmargin}{0in}
\setlength{\headheight}{.3in}
\setlength{\headsep}{.5in}
\parindent=0pt
\parskip= .35\baselineskip plus .0833333\baselineskip minus .0833333\baselineskip
\begin{document}
Artificial Intelligence for Development\\
Document No: AID/newsletter/number 7\\
Last Modified and by whom: 20/1/93 kk \\
Distributed on: 20/1/92\\
\LARGE
\begin{center} Artificial Intelligence for Development\\
Newsletter Number 7\\ January 1993\\

\end{center}
\normalsize
\section*{Contents}
\begin{enumerate}
\item Cheerioh!
\item News from around the world
\item New Literature
\item Jobs
\item Conferences
\item Groups, bulletin boards and mailing lists
\item Contact Personnel
\end{enumerate} 

\section{Cheerioh!}
Apologies as usual for a for anything that's out of date. Please check
the deadlines, if there are any, for jobs and conferences with the
employer/organiser. Apologies also for the mailinglist failures of the
past few issues. If you want copies of newsletters prior to this one
(i.e. 1, 2, 3, 4, 5 or 6) mail me. I am sending this one in ordinary
ascii format to those of you with email. If you want a LaTeX copy,
please email me for it. 
This is the last AI development newsletter I will 
be producing for the forseeable future. If anyone else would like to
take over the newsletter and organising the group please contact me
through any of the usual channels. kk@uk.ac.ed.aisb

\section{News from Around the World}
\subsection{Public access medical informatics software}
Renato M.E. Sabbatini SABBATINI@BR.UNICAMP.CCVAX writes:
 
The Center for Biomedical Informatics of the State University of
Campinas, Brazil, announces the availability of a new public-access
directory containing files for distribution via anonymous FTP (File Transfer
Protocol) to all Internet remote sites.
Its domain is MEDICAL APPLICATIONS OF INFORMATICS and contains documents
in ASCII format, as well public-domain software. As of October 1992,
the directory contains ca. 40 programs.
To access this directory, your local system must have the FTP program.
Execute this program by entering the following line:
 
FTP CCSUN.UNICAMP.BR or FTP 143.106.1.5  (the numeric address is better)
 
When you the prompt FTP appears, enter: USER ANONYMOUS
 and when prompted for your password, enter your e-mail address.
Having gained access to the system (it's UNIX-based), use
cd pub/medicine
 
Read the README and 00-INDEX.TXT files first, in order to
acknowledge yourself with the norms and contents of this
directory.\\ 
Attention: This directory continuously receives additional
files. You will receive an e-mail message whenever this
happens, but only if you ask us to do so.
 We gladly accept contributions to the public-domain directory. Please
address all requests for submissions (uploadings) to the directory-owner
below.
It is our intention to foster educational and practical uses of
computers in Medicine and Health Sciences by providing free access to
software resources and documentation, to the international community.

\subsection{More AI in Medicine results from Brazil}
Renato M.E. Sabbatini SABBATINI@BR.UNICAMP.CCVAX writes:

A group of physicians, nurses and medical informatics specialists
at the Medical School of State University of Campinas, Brazil,
are using artificial neural networks to predict death risk
for critical patients in ICU's. The neural network was
trained with physiological, clinical and pathological data
(30 variables, such as worst pCO2, coma level, arterial
pressure, etc.) based on a sample of more than 300 patients,
whose outcome was known.
The paper will be presented at the MEDINFO'92 workshop on
`Applications of Connectionist System in Biomedicine',
September 8, 1992, in Geneva, Switzerland. Preprints are
available from R.M.E. Sabbatini.


    OUTCOME PREDICTION FOR CRITICAL PATIENTS UNDER INTENSIVE
         CARE, USING BACKPROPAGATION NEURAL NETWORKS

         P. Felipe Jr., R.M.E. Sabbatini, P.M. Carvalho-
           Jknior, R.E. Beseggio, and R.G.G. Terzi

   Center for Biomedical Informatics, State University of
            Campinas, Campinas SP 13081-970 Brazil

Several scores have been designed to estimate death
probability for patients admitted to Intensive Care Units,
such as the APACHE and MPM systems, which are based on
regression analysis. In the present work, we have studied
the potential of a model of artificial neural network, the
three-layer perceptron with backpropagation learning rule,
to perform this task. Training and testing data were derived
from a Brazilian database which was previously used for
calculating APACHE scores. All networks were able to reach
convergence with a small global prediction error. Maximum
percentages of 75% correct predictions in the test dataset
and 99.6 % in the training dataset, were achieved. Maximum
sensitivity and specificity were 60% and 80%, respectively.
We conclude that the neural network approach has worked well
for outcome prognosis in a highly `noisy' dataset, with a
similar, if slightly lower performance than APACHE II, but
with the advantage of deriving its parameters from a
regional dataset instead from an universal model.

\subsection{Cross-Cultural database}
Roberto Evaristo (EVARISTO @ UMNSOM.BITNET): writes

How many times have you considered getting involved in cross-
cultural research in information systems but had difficulties
finding an appropriate partner with similar interests?
We have implemented a database with the objective of bringing
together researchers with interest in cross-cultural research in
the IS field.  We hope that this will be an added incentive for
getting more people involved in cross-cultural research.  This
database, although including people with research interests in
developing countries, includes people that are interested in
developed countries as well.  There are currently about 105 people
listed in the database.
An e-mail list is being implemented to provide a forum for all
interested researchers to express their ideas or simply exchange
information on any topic of interest to the community.  All those
who sign-up for the database will (a) be included in this e-mail
list and (b) be sent a complete electronic listing of the database,
including all other members' research interests and addresses.
If you are interested, please send the following information to
Roberto Evaristo (at the e-mail address EVARISTO @ UMNSOM.BITNET):
Name, Mailing Address, E-mail address, Research Interests, and a
list of countries (or areas of the world) that you are particularly
interested in for getting involved in cross-cultural research.
OBSERVATION:   Please also mention if you have limitations in file
sizes to be received over e-mail, because the full database listing
is about 2000 lines long (and still growing).

\subsection{AI in Iran}
anoosh@com.mips writes:

Informatic Society of Iran - Artificial Intelligence Specialist Interest Group
			An Initial Proposal

This is an initial assessment of interest for formulating a proposal directed
to the Informatic Society of Iran (ISI) with the aim of setting up a SIG 
covering the area of Artificial Intelligence and its related fields. The 
intended outcome is a collection of ideas and visions on what could the SIG 
do, what problems it might encounter and how those problems could be handled.
For those who are not familiar with the ISI there is a brief introduction to 
the Society in the appendix below.

Why AI-SIG?

- There is currently no AI society or sub-society active in Iran, therefore 
  no duplication of activity/effort will be done.\\
- There are formal courses on AI being taught in universities (e.g. MSc in AI)
  and there are increasing number of research papers authored by Iranians 
  appearing in international conferences and journals: therefore there is a 
  suitable ground to work on. (I do not honestly know why there has not been 
  an AI-SIG established by now.)\\
- AI-SIG can offer an international  forum for the Iranian AI community 
  through, for example, publishing a periodical, holding seminars (regional), 
  participating in the organisation of conferences. Also it could become the 
  technological point of contact in the field of AI in Iran (i.e. for 
  collaboration with other AI groups or individuals). Essentially; `a means 
  to receive, gather and disseminate technology'.\\
- The list of benefits that such AI-SIG could offer, both at individual 
  level and national level, could be much longer. The list of implementational 
  activities could also be enriched and that is what I hope we will achieve 
  through your replies.

ISI (see the appendix) already enjoys a well established administrative and 
technical support structure evolved over its 14 years of activity. Further, 
it has the benefit of having contacts in the industry and academia. As part 
of its activity spectrum, ISI is interested in setting up SIGs, and I know 
of at least one such SIG in ISI (C programming). ISI  is the obvious choice 
if one believes that one strong society can be more productive than few 
small ones.

I feel that some of the key issues to touch on include:
- how to make it successful amongst Iranians abroad, \\
- how could Iranians abroad contribute to its success,\\
- the desired relationship between the SIG and its parent society,\\
- the nature and content of the activities that the AI-SIG could undertake 
  (periodical, seminars, etc.)\\
- the issues of language (Farsi vs. others) and geographical dispersion,\\
- costs: (A word on the potential cost of membership:  Current membership fee
 of ISI is 25 USDollars (15 Pounds Sterling) payable abroad, or, 5000 Rials 
 payable in Iran. So I would imagine that ISI and its AI-SIG membership would
  cost about 30 USDollars per annum.)\\
- in order to assess the potential of the SIG, it would be a great help if you
  could indicate whether you would be willing to take on membership of an 
  AI-SIG, take on an active role (e.g. by contributing to a periodical and/or
  regional seminars), and also what could the SIG offer to make it a more 
  attractive forum to its members.

* Please MAKE COPIES AND PASS ON to any Iranian whom you might feel could be 
  interested in the topic.\\
* Your IDEAS, COMMENTS AND THOUGHTS will be most gratefully appreciated.\\
- I can be contacted by: Email: amir@case.co.uk, Tel: (+44) (923)
258292 (work), (+44) (71) 2213119 (home) Address: Amir S. Tabandeh,
Research Department, Dowty Communications Ltd.
Caxton Way, Watford, Herts.  WD1  8XH, England

Appendix: Few words on ISI:\\
ISI is a non-profit-distributing and independent society (no affiliation to 
the government, universities or companies) purely devoted to promoting 
computer science and its related topics in Iran. It has been active since 1357
(1978). Its executive body is elected bi-annually. All active members work on
voluntary basis.  ISI has just under one thousand members from universities 
and companies in Iran and abroad (the figure is a guesstimate). Apart from the
class of `student membership', members should hold at least a BSc.  ISI 
activities include, but not limited to, publishing `Computer Report (ISI-CR)'
(a bi-monthly magazine), holding monthly seminars, organising conferences 
(the latest was on `Computers in Education'), organising educational and 
awareness courses, compiling a Computer Dictionary (English-Farsi), 
maintaining a library, etc. ISI activities are fundamentally based on its 
members' initiatives and endurance.  Its financial accounts are published 
annually. The latest was in CR Vol.13, No.4 (Nov. 1991), indicating that the 
main sources of income are subscription dues and members' donations.

\section{New Literature}
Reprinted without permission from the AI in Medicine mailinglist.
Source for this information was humphrey@gov.nih.nlm (Susanne M
Humphrey). 

 University Microfilms Order Number ADG92-34334.\\
 SCHOENHOFF, DORIS MARIE.\\
 THE BAREFOOT EXPERT: THE INTERFACE OF COMPUTERIZED KNOWLEDGE SYSTEMS
   AND INDIGENOUS KNOWLEDGE SYSTEMS.\\
 Washington University Ph.D. 1992, 228 pages.\\
 DAI V53(07), SecB, pp3607.\\
 Computer Science.  Education, Bilingual and Multicultural.
   Artificial Intelligence.\\

 Computerized expert systems, implementations of Artificial
   Intelligence, are unique from earlier forms of technology because
   they seek to externalize man's reasoning ability and to breed, not
   simple store, knowledge outside of the human mind.  This
   knowledge-based technology, which purports to symbolically encode
   knowledge and expertise, can never be completely extricated from the
   language, culture, and context in which it is designed and
   implemented.  To be humanly meaningful in Third World environments,
   expert systems must incorporate indigenous knowledge and indigenous
   reasoning about knowledge.

   Imprecise, unsystematic local knowledge will be pitted against
   Western prejudices that `real' knowledge, `technical' knowledge is
   formalized, logical, verifiable, externalized, and impersonal.
   Knowledge, however, can never be completely articulated or
   formalized.  Expert systems can only encapsulate what experts say
   about their knowledge.

   While respect for the local knowledge and wisdom is critical, it is
   also important not to idealize or romanticize it.  Traditional
   systems may, in fact, be improved by incorporating certain elements
   of Western knowledge and science.

   The decision to implement knowledge-based systems places ethical
   demands and risks upon those in power within Third World nations and
   upon those from the industrialized nations that trade with and for
   that power.  These knowledge-based systems can potentially be useful
   tools in developing the critical consciousness of local community
   leaders because the design process forces individuals to think about
   their knowledge and not just act upon it.

   While they are potential tools, expert systems are also a potential
   threat to personal and social identity.  Whether in a traditional
   society or in a modern society, our identity and role are largely
   determined by what we know.  To convince someone that most of their
   knowledge is useless or even wrong is to proffer death not
   development.  Western experts must be open to the possibility of
   their own ignorance and not let their self-perceived expertise
   preclude local participation and decision.  If we do not seek the
   empowerment of the local communities, then our own technological
   knowledge, including our attempts to recreate ourselves in silicon
   and symbols, diminishes us and puts wisdom and global peace further
   from our reach.

\section{Jobs}
\subsection{Costa Rica}
 
Program Officer, Costa Rica.\\ 
Acceso, a non-profit service organization based in Costa Rica, 
provides communication services, information, training and 
technical assistance to NGOs in Central America.  The program 
officer will be responsible for contributing to the design, start-
up, and consolidation of an electronic network for communications 
and information exchange between NGOs in the region; recruitment 
and orientation of new organizations on the network; coordinating 
the organizations that will contribute specific information 
products to the network; working closely with the network steering 
committee; training new users; assisting with the development of 
training materials; and participating in the administration and 
institutional development of Acceso.  \\ 
Essential Requirements:  A personal commitment to the regionUs 
social and economic development, bilingual (English/Spanish), 
minimum of 4 years professional experience with NGOs, university 
degree, familiarity with computers and word processing programs, 
ability to write clearly and analytically.  \\ 
Preferred Requirements:  Central American origin or experience, 
experience with electronic communication, applied experience in 
management of development organizations, experience in training 
and knowledge of training programs and methods.\\ 
Remuneration:  Competitive salary and benefits commensurate with 
experience.\\ 
Please submit:  Resume, an original essay in Spanish (maximum two 
typed pages) on the topic 'Factores Criticos para el Trabajo de 
Organizaciones Privadas de Desarrollo en los Annos Noventa,' and 
the names and phone numbers of three professional references.  
 
Send application to:  Acceso, P.O. Box 025216-34, Miami, FL 33102-
5216, or by e-mail to acceso@nicarao.apc.org.
 
Application deadline:  January 15, 1993.
 
\subsection{Nigeria}
das@castle.ed.ac.uk (D Stewart) writes:

Eclipse compuing need someone to work on a contract in
Nigeria. This would involve looking after a BP computing (unix ?)  
system in Nigeria. This would initially be for one year
(could be two). UKL 40,000 tax free, banked in the UK. All
expenses paid plus flights home. 10 weeks on 2 weeks off.

If at all interested then please contact:-\\
Colin Garrod,
Eclipse Computing,
Unit 17,
Thistle Business Park North,
Ayr Rd,
Cumnock,
Ayrshire,
SCOTLAND,
KA18 1EQ, Phone: 
0290 20987
Fax: 0290 20233

\subsection{Vietnam}
fair@iss.nus.sg (Kim Michael Fairchild) writes:

I have been asked by SCITEC/LOTUS in Ho Chi Minh City (Saigon) to post
this message.
They are interested in hosting lecturers at this leading computer
technology institute. 
If you have a desire to see Vietnam, this is a marvelous opportunity.
I have lectured there myself several times.
Basically, in exchange for lecturing, they will arrange your visas,
hotels, cars, drivers, guides, and trips out of Saigon. 
They are interested in most computer technology, in particular
multimedia, and technologies that are emerging. 
If you would like more information, write to me at fair@iss.nus.sg or
fax directly to Vietnam at 84.8.30059, Attention: Director Hiep. 

I will be going myself on Jan 20th (with my mom who will guest lecture
at some English classes) and if you wish, I can bring
hard-copy of any messages you might have. 

\subsection{Latin America (Location Boston)}
ccrazy@athena.mit.edu (Ellen Kranzer) writes:

Position Open:  Project Officer for Electronic Communications\\
Organization:  Latin American Scholarship Program of American Universities 
(LASPAU).\\
Responsibilities:  Cooperates with LASPAU staff to ensure that Fulbright 
scholars are connected to BITNET and the Internet at their host 
universities.  Develops an electronic network of Fulbright scholars in the 
United States and Fulbright alumni in Latin America.  On behalf of and in 
cooperation with Latin American and Caribbean universities -- with the 
assistance of consultants to LASPAU -- develops and oversees plans to 
provide `value added services' to increase  electronic-mail message volume 
and the number of users in Latin American and Caribbean academic and 
research institutions.  Maintains electronic mail lists for other projects. 
Reports to Program Director.   \\
Qualifications:  Bachelor's degree in a related discipline.  Two to three 
years of experience with electronic-communications technology and 
communications Fluency in Spanish is required.  Knowledge of Latin American 
higher education systems a plus.\\
Hiring Salary: US dollars 26,400 - 33,800/year

Send Resumes to:
Joyce Lamensdorf, Personnel Officer,
LASPAU,
25 Mt. Auburn Street,
Cambridge, MA  02138

The Latin American Scholarship Program of American Universities (LASPAU), 
affiliated with Harvard University, is a nonprofit organization which 
assists in the development of universities and other public and private 
institutions in Latin America and the Caribbean.  LASPAU administers 
approximately 1,000 scholarships per year for Latin American and Caribbean 
professionals to pursue graduate studies in the United States.  The 
organization also offers specialized educational consulting services to 
institutions both inside and outside of the Americas.

\section{Conferences}
\subsection{Social science research, technical systems and cooperative
work: Workshop}
{\bf DATE:} March 8-11, 1993\\
{\bf VENUE:} French Research and Space Ministry, 1 rue Descartes,
75005 Paris, France \\
{\bf HOSTED:} The Delagation for Scientific and Technical Information
of France's Research and Space Ministry, The British Council\\
{\bf TOPICS:}     The number of people participating in the workshop will be
limited in order to ensure optimal working conditions.  Each
participant will be asked to provide `notes' summarizing his or her
experience in building technical systems to support cooperative work,
especailly on processes of collaboration across the divide.  `Notes'
are not `papers': the goal is to identify and critically analyze the
obstacles that have to be overcome in order to build an academic
working culture in the area.  The `papers' will be requested after the
workshop from a certain number of participants.  The goal is to
publish a book on `Social Science Research, Technical Systems and
Cooperative Work' which will help lay the foundations of work in this
growing area.\\
{\bf DEADLINES:} The Organization Committee select contributions to
the workshop by Dec. 15.\\
{\bf CORRESPONDENCE:}  People wanting to participate should send a
one-page description of their professional activity indicating the
questions they would like to address to :
        G. Bowker, Center for the History of Science,
        Technology and Medicine, Mathematics Tower, University
        of Manchester, M13 9PL, UK
            email:  g.c.bowker@manchester.ac.uk

\subsection{International Conference on Information Technology and People - ITAP'93}
{\bf DATE:} May 24-26, 1993\\
{\bf VENUE:} Moscow, Russia \\
{\bf HOSTED:} International Centre for Scientific and Technical
Information (Moscow), The Institute of Electrical Engineers (UK),
British Computer Society (UK), Open University (UK), Human Factors
Society, ACM SIG on Computers and Society, Journal `Information
Technology and People'\\
{\bf TOPICS:} IT and changing society, Strategic implications of IT,
The revolution in the mass-media, IT and quality of life, IT in the
home, IT's impact on working life and skills, IT in retail
distribution and banking, Health and safety conditions - implications
of IT, IT and disabled people, Electronic social and culture
information, Computer networks and human-to-human interaction,
Barriers to communication, Social protocol and social machines, Future
media and evaluation of forthcoming IT, Interactive experience,
Creative/innovative lifestyles and IT, Intellectual property, Human
and organizational factors in information technology, IT in education
and training, IT and government, IT in manufacturing\\
{\bf FORMAT FOR PAPERS:}
Submissions for papers and poster sessions are invited in the form
of abstracts  of about 500 words to be sent by email OR FAX to the
conference    secretaries.
Each submission  should state whether for paper or poster session,
it should  identify a topic area to which it is addressed, and any
supporting facilities required. Submissions should include name(s)
of authors,  postal and  email addresses,  and a list of keywords.\\
{\bf PROCEEDINGS:} Authors of  papers accepted  are expected to send a
5000 word text by March 15th for publication in English in the
Proceedings of the conference.\\
{\bf LANGUAGE:} English and Russian\\
{\bf DEADLINES:} 15th December 1992\\
{\bf CORRESPONDENCE:} enir@ccic.icsti.msk.su or FAX No
(7095)943-00-89, and m.allott@open.ac.uk or FAX No (44)998 653744).\\

\subsection{XIX Latin American Informatics Conference}
The XIX Latin American Informatics Conference will be held along with
the 22nd Argentine Meeting of Informatics and Operational Research.\\
{\bf DATE:} 2nd - 6th of August, 1993\\
{\bf VENUE:} Buenos Aires, Argentina\\
{\bf TOPICS}:
    Algorithms and Data Structures.
    Artificial Intelligence.
    Databases and Knowledge Bases.
    Networks and Distributed Systems.
    Non-Conventional Architectures.
    Software Engineering.
    Programming Paradigms.
    Connectionism and Neural Networks.
    Human-Computer Interfaces.
    Optimization and Simulation.
    Informatics in Education.
    Informatics and Government.\\
{\bf FORMAT FOR PAPERS:}
Prospective authors should submit three copies of their work, of up to
20 A4 pages in lenght (21 x 29.7cm) double-spaced, with an abstract of
no more than 200 words\\
{\bf LANGUAGE:} Spanish, Portuguese or English.\\
{\bf DEADLINES:} Papers by APRIL 1st, 1993. Notification of acceptance or
rejection will be communicated by May 31st, 1993.\\
{\bf CORRESPONDENCE:}
Papers should be sent to:
  Program Committee
  PANEL '93/22 JAIIO, SADIO
  Uruguay 252, 2do. piso D
  1015 Buenos Aires, Argentina\\
For additional information, contact: 
jaiio@sadio.edu.ar, fax +54-1-476-3950,
telephone +54-1-40-5755.

\subsection{The 1993 East-West International Conference on Human-Computer Interaction}
{\bf DATE:} 3-6 August, 1993 (tentative)\\
{\bf VENUE:} Moscow, Russia\\
{\bf TOPICS:} human factors and ergonomics, formal methods, modeling,
and simulation, object-oriented languages and systems, knowledge-based
and expert systems and interfaces, hypertext and hypermedia,
networking and computer-mediated human communication, graphics and
video, marketing and use of interactive systems, architectures for
interactive systems, user testing practices, user interface
standardization, both corporate and industrial, user interface tools
and development environments \\
{\bf FORMAT FOR PAPERS:}  For those intending
to submit a paper, tutorial, poster, demonstration or video the
notification should include the format, title and a brief outline (100
words) of the proposed submission. Proposals for a demonstration
should include details of the equipment required.  PC based
demonstations will be especially appropriate.  Those interested in
attending but who do not intend to submit a presentation should
include a brief statement summarising their interests (300 words).
Submissions must include Name, Address, Telephone number, Fax number
and e-mail address. For those in the West, all other materials must be sent in hardcopy to: 
Claus Unger,
Praktische Informatik II,
Fernuniversitat,
Feithstr. 140,
D-5800 Hagen,
GERMANY\\
{\bf PROCEEDINGS:} A Proceedings of the conference will be published.\\
{\bf LANGUAGE:} English, translation provided when necessary.  \\
{\bf DEADLINES:} 
Feb 15, 1993    Notification of intention to participate,
March 15        Submissions,
May 1           Notification of acceptance,
June 1          Submission of camera-ready copy for the Proceedings\\
{\bf CORRESPONDENCE:} The notification of intention to participate
should be e-mailed to: ew-submit.chi@xerox.com. \\
Questions should be directed to:  ew-info.chi@xerox.com.\\

\subsection{Intellectual Property Rights in Computer Software
and their Impact on Developing Countries.}
{\bf DATE:} Fri 20-21, Aug 1993\\
{\bf VENUE:}  Indian Institute of Science (IISc), Bangalore INDIA\\
{\bf HOSTED:} International Federation for Information Processing
(IFIP WG9.4) and Computer Society of India (CSI) in cooperation with
Institution of Engineers (IE) Indian Institute of Science (IISc)  \\
{\bf TOPICS:} How well is the existing IPR regime in software
currently working in developed and developing countries?, Do
developing countries need alternative IPR regimes?, Comparative
evaluation and impact on a country's economy of different patent
regimes, Creation, classification and access to software patent
databases, Current defensive techniques for avoiding IPR difficulties,
Landmark legal case studies of IPR disputes and their resolution, Case
studies of the workings of national patent offices in the area of
software, Impact of Dunkel proposals in developed and developing
countries, Nature and distribution of software patents and actual
ownership, Perspectives of established software companies in developed
countries wrt developing countries in the area of IPR, Licensing of
technologies to developing countries, High cost of productivity
software, Control of unauthorised duplication of software in developed
and developing countries, Strategies and options for developing
countries post-Dunkel, Impact of ``open systems'' like IBM PC for
world economy, IPR in automatically generated code,
Philosophical/equity issues in assigning IPR to ``common property
resources'', Case studies of how IPR were articulated and enforced in
new technologies in the past, Similarities and differences with IPR in
other areas like biotechnology    \\
{\bf FORMAT FOR PAPERS:} E-mail submission is strongly encouraged, esp. in
Latex / Wordstar  format.  (All e-mail submissions will be
acknowledged - please remail if there is no ack.) If e-mail facility
is not available, authors should submit five copies of the manuscript
in A4 size paper not exceeding 20 pages.\\
{\bf PROCEEDINGS:} Will be published by North Holland under an
arrangement with IFIP. \\
{\bf LANGUAGE:} English\\
{\bf DEADLINES:} 	Full Paper 		May 1, 1993,
	Acceptance notice 	June 15, 1993,
	Final manuscript	July 15, 1993\\
{\bf REGISTRATION:} The registration fee is Rs. 600 / US dollars 100 which will
cover lunches and teas. Participants from developing countries may
apply for subsidy from the registration committee.\\
{\bf ACCOMMODATION:}
Accommodation arrangements should be made directly by participants. It is 
expected that some housing will be made available in the Institute
Guest House. Details will be made available later. \\
{\bf CORRESPONDENCE:}	K. Gopinath, Asst. Prof.
	Computer Science and Automation (CSA), 
	Indian Institute of Science (IISc),
	Bangalore 560012 INDIA 
     Telex: 0845-8349 IISc IN   Fax: 091-0812-341683 
     e-mail: ipr@maitreyi.csa.iisc.ernet.in 

\subsection{Support for attending IJCAI}
Alan Bundy (bundy@uk.ac.ed.aisb) writes:

		    TRAVEL GRANTS FOR IJCAI-93

IJCAII has established a program to provide support for travel to and
participation in IJCAI-93, the 13th International Joint Conference on
Artificial Intelligence, which will take place in Chambery, France from
29 August to 3 September 1993.  The amounts awarded will vary depending
on location and on the number of persons applying.  The intent is to
help about 100 people.  Priority will be given (1) to younger members
of the AI community who are presenting papers or are on panels at the
conference and who would not otherwise be able to attend because of
limited travel funds and (2) to members of the AI community whose
countries have currency problems and cannot provide support for
attendance at IJCAI-93.

To insure consideration, applications should be received by 1 May
1993.  They should briefly identify the expected form of conference
participation; describe benefits that would result from attendance;
specify current sources of research funding; and list travel support
from other sources.  A brief resume should be attached, and students
should include a letter of recommendation from a faculty member.

Awards from the grants will be issued after the conference following
submission of receipts for expenses incurred and a brief report on
participation in the conference.  Although grants will be awarded in
specified denominations, payment will be made on the basis of actual
expenses.

Three copies of the application should be sent to:

	Priscilla Rasmussen, IJCAI-93 Travel Grants
	Rutgers, the State University
	Laboratory for Computer Science Research
	Hill Center, Busch Campus
	New Brunswick, NJ 08903, USA
	+1-908-932-2768 phone
	+1-908-932-0537 fax
	rasmussen@aramis.rutgers.edu


\subsection{IFIP WG 9.4, 1994 The impact of Informatics on Society:
Key issue for Developing Countries}

{\bf DATE:} February 21-23, 1994.\\
{\bf VENUE:} the International Conference Centre, Havana, Cuba\\
{\bf HOSTED:}
   Working Group 9.4 of the International Federation of 
   Information Processing Societies,
   INSAC, The National Institute for Automated Systems and Computer 
   Technics, (Electronics and Informatics Ministry), Cuba,
   National Electronic Front and National Informatics Committee, 
   Non Governmental Organizations, and Havana University\\
{\bf TOPICS:}
   Social Implications,
   Strategies and Policies for Developing Countries,
   Impact of I. T. Applications,
   Improving Social Services,
   Increasing the Competitiveness of Business,
   Export of Services and Software
{\bf PROCEEDINGS:}   Xerox copies of the full papers will be available at the 
   conference.  An edited proceedings will be produced after the 
   conference.\\
{\bf DEADLINES:} Notice of intent to participate, as soon as possible.
   Submission of 2-3 page abstract, June 15, 1993.
   Notice of acceptance, September 1, 1993.
   Full papers due: December 1, 1993.\\
{\bf CORRESPONDENCE:}
Submit papers and program queries to:
   Prof. Sam Lanfranco,
   Centre for Research on Latin America and the Caribbean (CERLAC),
   York University (Room 240YL),
   4700 Keele Street, North York,
   Ontario, Canada, M3J 1P3,
   phone: (416) 736-5237 (English or Spanish),
   fax:   (416) 736-5737,
   email: lanfran@vm1.yorku.ca\\
Volunteers to work with the Cuban Organizing Committee contact:
   Prof. Ruben A. Lopez Santana,
   INSAC,
   5ta Ave No. 4602 entre 46 y 60 ,Playa, Habana 11300, La Habana, Cuba,
   Phone: + 53-7 -331191,222530-40,221463,222885,
   Fax: + 53-7-331233,332510,331191,
   Telex:  511258 and 511706 dric cu,
   E. Mail: insac%ceniai@web.apc.org\\
Volunteers to work with the International Organizing Committee or 
Program Committee contact:
   Prof. Larry Press,
   California State University, Dominguez Hills,
   10726 Esther Avenue,
   Los Angeles, CA  90064,
   phone: (310) 475-6515,
   fax:   (310) 516-3664 (fax),
   email: lpress@isi.edu

\section{Groups, Bulletin Boards and Mailing Lists}

\subsection{Niblist}
Renato M.E. Sabbatini SABBATINI@BR.UNICAMP.CCVAX writes:
 
The State University of Campinas announces the availability of NIBNews,
an electronic newsletter dedicated to Medical Informatics.
 The Newsletter has the objetive of disseminating information about
Brazilian and Latin-american activities, people, news, scientific
events, publications, software, etc., in the area of computer
applications in healthcare, medicine and biology. It is issued
on a monthly basis to voluntary subscribers, only. Subscription is
free. Past issues are available by FTP (issue 1 is dated June 1992).
Periodically, NIBNews contains notices about new public-domain medical
software added to the Internet FTP public directory CCSUN.UNICAMP.BR,
maintained by the State University of Campinas.
NIBNews is distributed automatically through LISTSERV@CCSUN.UNICAMP.BR,
which is a Sun-based server located at the State University of Campinas,
in Campinas, Brazil.
 
To subscribe, send an one-line message to the listserver above, containing:\\
 SUBSCRIBE NIBNEWS Your Name\\
 NIBNews is also a moderated list, so you may submit contributions to it,
addressing NIBNEWS@CCSUN.UNICAMP.BR
 In case you do not have access to Internet, send your request for
subscription or submission directly to my email address.

\subsection{Southern Africa Net}

jim@sytex.com (Jim Arnold) writes:

I would like to draw people's attention to a new network
being set up under the auspices of the U.S. Agency for International
Development, and the World Food Program, called `SAFIRE' or
Southern Africa Food Information Resources Exchange.
This network is now being put into place, using FTSC (fidonet)
technology, and links with MANGO Net and other existing African
networks.
For information, contact john.glazer@f183.n109.z1.fidonet.org
I am sure he will be happy to respond to any queries.

\subsection{Gnet}
Laurence Press (LPRESS@ISI.EDU) writes:

GNET: an Archive and Electronic Journal, Toward a Truly Global Network

Computer-mediated communication networks are proliferating and 
growing rapidly, yet they are not truly global -- they are 
concentrated in affluent parts of North America, Western Europe, 
and parts of Asia.  \\
GNET is an archive/journal for documents pertaining to the effort 
to bring the net to lesser-developed nations and the poorer parts 
of developed nations.  (Net access is better in many `third 
world' schools than in South-Central Los Angeles).  GNET consists 
of two parts, an archive directory and a moderated discussion.\\
Archived documents are available by anonymous ftp from the 
directory global\verb+_+net at dhvx20.csudh.edu (155.135.1.1).  To 
conserve bandwidth, the archive contains an abstract of each 
document, as well as the full document.  (Those without ftp 
access can contact me for instructions on mail-based retrieval).\\
In addition to the archive, there is a moderated GNET discussion 
list.  The list is limited to discussion of the documents in the 
archive.  It is hoped that document authors will follow this 
discussion, and update their documents accordingly.  If this 
happens, the archive will become a dynamic journal.  Monthly 
mailings will list new papers added to the archive.\\
We wish broad participation, with papers from nuts-and-bolts to 
visionary.  Suitable topics include, but are not restricted to:\\
   low-cost, appropriate-technology networks,
   satellite and terrestrial packet radio,
   communication protocols,
   connection options,
   host and user software,
   the current state of global networking,
   current applications,
   proposed applications,
   education in a networked world,
   education for a networked world,
   social implications of a global network,
   economic implications of a global network,
   politics and funding for a global network,
   political implications of a global network,
   free speech on the global network,
   environmental implications of a global network,
   directories and lists of people and resources

To submit a document to the archive or subscribe to the moderated 
discussion list, send a message to gnet\verb+_+request@dhvx20.csudh.edu.  

\subsection{Africa-l}

To subscribe to africa-l,a  listserv list dealing with Africa, send
the following message:\\
	SUBSCRIBE AFRICA-L name name area of interest\\
	SET AFRICA-L REPRO\\
to this address: LISTSERV@BRUFBP.BITNET

\subsection{Comnet-IT}

THE COMMONWEALTH NETWORK OF INFORMATION
TECHNOLOGY FOR DEVELOPMENT (COMNET-IT)

COMNET-IT is a newly formed Commonwealth organisation working  to
build networks of people and institutions cooperating to harness Information
Technology for development.
It is intended that COMNET-IT will share the experiences, successes and
failures, of Information Technology applications and of national policies; 
will identify key areas of research; and will assess training opportunities.
COMNET-IT will make the maximum use of existing computer networks for
information sharing within and between Commonwealth countries.
COMNET-IT is an umbrella organisation within which a series of related
Task Forces will be working in specific areas:

	The Technical Advice Task Force will be providing advice, support and
encouragement to key people, particularly in developing countries, enabling them to take maximum advantage from existing networks.
	Contact:			Dr. S. Ramani
					phone: +91-22-6200590
					fax:   +91-22-6210139
					email: ramani@ncst.ernet.in
					       ramani@shakti.ernet.in

	The Directory/Partnership Task Force will be enhancing the value and
attractiveness of existing professional networks by providing directories and by developing and supporting communities of interest.
	Contact:			Mr. J. Seymour
					phone: +44-71-405-8400
					fax:   +44-71-242-1845
					email: j.seymour@noc.ulcc.ac.uk

	The Management Development Task Force will strengthen and
develop the lines of communication between Commonwealth national and regional Management Development Institutions.
	Contact:			Mr. P. Schmidt
					phone: +61-3-2151156
					fax:   +61-3-2151166
					email:
mez113d@vaxc.cc.monash.edu.au

	The Research Task Force will share experiences of IT applications,
and will develop individual and collaborative research proposals.
	Contact:			Dr. S. Bhatnagar
					phone: +91-272-407241
					fax:   +91-272-467396
					email: subhash@iimahd.ernet.in

	The Publications/Newsletter Task Force will facilitate the sharing of
papers from personnel within the developing countries, supported by a
system of active moderation.  The Task Force will also be collaborating with
IFIP Working Group 9.4 to produce a shared newsletter 'Information
Technology in Developing Countries'.
	Contact:			Dr. S. Bhatnagar
					phone: +91-272-407241
					fax:   +91-272-467396
					email: subhash@iimahd.ernet.in

	The Caribbean Region Task Force is the first of an intended series of
regional Task Forces concerned to identify and support local Networking
 Centres.
	Contact:			R. Sanatan
					phone: +592-2-69281
					fax:   +592-2-67816

	The COMNET-IT Secretariat will respond to general enquiries.
	Contact:			Ms S. Qureshi
					phone: +44-71-839-3411 (x8322)
					fax:   +44-71-930-0827
					email: comsec@noc.ulcc.ac.uk

\section{Contact Personnel}
Please send contributions to the newsletter or requests for addition
to the mailing list to KK. Email is the communication method of choice
(it takes so long to type the gubbins in) but communication through any
medium is welcome, especially if it contains contributions to the
newsletter!

Incidentally, if you receive this newsletter in elctronic form and do
not have LaTeX at your site to print it out in a nice format, get in
touch and we'll send you either a hardcopy form or an electronic form
without the commands, whichever you prefer.

\begin{tabular}{l|l}
{\bf Software library:} & {\bf Newsletter, overall co-ordination,
meetings:}\\ 
Howard Beck &  Kathleen King\\
Artificial Intelligence Applications Institute &  Department of
Artificial Intelligence \\
University of Edinburgh & University of Edinburgh\\
80 South Bridge & 80 South Bridge\\
Edinburgh EH1 1HN & Edinburgh EH1 1HN\\
031 650 2747 & 031 650 2726\\
hab@uk.ac.ed.aiai & kk@uk.ac.ed.aipna\\
\hline
{\bf Addresses and contacts, funding:} & {\bf Literature resource and bibliography:} \\ 
Robert Muetzelfeldt & Ehud Reiter\\
Department of Forestry and Natural Resources &   Department of Artificial Intelligence\\
University of Edinburgh & University of Edinburgh\\
Kings Buildings & 80 South Bridge\\
 Mayfield Road &  Edinburgh\\
Edinburgh   EH9 3JU & EH1 1HN\\
 031 650 5408 & 031 650 2728\\
 R.Muetzelfeldt@uk.ac.edinburgh & reiter@uk.ac.ed.aipna\\
\hline

\end{tabular}

\end{document}

