\documentstyle[a4]{article}
\pagestyle{myheadings}
\markboth{Author:kk}{Doc.No. AID/newsletter/2}
\setlength{\textheight}{9in}
\setlength{\topmargin}{0in}
\setlength{\headheight}{.3in}
\setlength{\headsep}{.5in}
\parindent=0pt
\parskip= .35\baselineskip plus .0833333\baselineskip minus .0833333\baselineskip
\begin{document}
Artificial Intelligence for Development\\
Document No: AID/newsletter/2\\
Last Modified and by whom: 1/3/91 kk \\
Distributed on:  1/3/91\\
\LARGE
\begin{center} Artificial Intelligence for Development\\
Newsletter Number 2\\March 1991\\
\end{center}
\normalsize

This newsletter is being sent by both electronic and paper mail. If
you no longer wish to receive it, or have any contributions or
comments on its content, please contact KK by any of the normal
methods. Contributions will be particularly well received!
We seem to be having a problem getting through to some folk by email. If you
receive a paper copy of this and not an electronic one, but think you
{\em should} be receiving an electronic one, make yourself known.

\section{ LUPES - knowledge-based land use allocation.}
For those of you who didn't make it to the last meeting, {\bf Robert
Muetzelfeldt} here outlines the elements of the project he spoke on and
demonstrated. If you would like to see or use LUPES, contact RM.

LUPES is a program designed to allocate a land use to each of a number of
patches of land ("zones").   The aim is to mimic the reasoning methods of the
human land use planner to the extent possible.
The major constraint that LUPES applies is that you must represent an area in
terms of a number of discrete zones.   Each zone is assumed to be internally
uniform (homogenous), and can have only one land use applied to it.
Each zone has a number of boundaries (which can be shared between a
neighbouring zone; each boundary has a set of co-ordinates.   The co-ordinate
information is used for display purposes only: the landuse allocation part
could operate without this co-ordinate information.

It is entirely up to the user to specify zone attributes: these should be
attributes that influence landuse allocation decisions.   They can include
obvious things like soil, altitude, and current land use, as well as more
subtle things like which zones are next to it, distances from roads or points,
etc.   The full power of Prolog is available to enable this information to be
asserted as facts or inferred from other information: for example, land
capability can be inferred from soil, altitude, etc; contiguity can be
inferred from the sharing of boundaries; distances can be inferred fromthe
co-ordinate information.

Allocation of landuse to a zone is controlled by two types of rule. The
'compatible-with' rule specifies which landuse is compatible with which zone.
This draws on zone attributes, but can also draw on the attributes of any
other zones in the area, including those which are known to be contiguous.
Most importantly, the 'compatible-with' rules can also draw on knowledge about
the allocations that have been made so far in the allocation process: for
example, there may be a requirement that all allocations of a certain type of
landuse are contiguous, which is expressed as 'landuse U is compatible-with
zone Z if Z is next to Z1 and Z1 has already been allocated U' (but in Prolog,
of course).   So, these rules offer great expressiveness: you could check for
things like preferred ratios between different landuses, for example. The
challenge is to see if there are things you would like to express that can NOT
be captured with these rules.

The second type of allocation rule is 'requirement-failure': it checks at the
end of the allocation process for things that can only be sensibly checked at
that stage: e.g. that at least one of the zones has forestry, or that there is
a minimum area of maize; or that the total revenue exceeds a certain amount.
So, the user can supply information specific to a particular area; and, at a
higher level, rules about allocation (which presumably will apply to many
areas of the same type).   At a still higher level, the user can specify how
the allocation process is actually to proceed.   In particular, LUPES needs to
know which zone to try to allocate next.   I could have built in one
particular strategy (e.g. take the next one off a list of all the zones), but
it is important for me that LUPES should mimic to the extent possible the way
that human landuse planners go about there task. Therefore, the user can
specify (in Prolog) 'any' arbitrary strategy for choosing the next zone to try
- e.g. choose the most constrained zone; the one with the largest area; a zone
that is next to one that has already  \ been allocated a landuse; or whatever.
There are various ramifications here, but this should give you a feel of the
way in which I'm trying to give the user as much control as possible.

There's currently a demo version of LUPES running on a Mac II using LPA
MacProlog in colour.   It provides extra facilities for user interaction - the
user can directly allocate a landuse to a zone (provided its compatible); find
out all the zones that can have a certain landuse applied to them; etc.
However, I stress that what I have now is essentially a rather crude
prototype, and is not in a form that I'd be prepared to release. However, it
is being used in a current M.Sc. dissertation, and is the basis for a Ph.D. on
farm forestry in Scotland that has just started.

\section{News from around the world}
Contributions to this section are particularly welcome. Personal
experience from living and working in a country, or even having
visited for a short time, is always interesting, and this is also a
good forum to pass on informal knowledge gleaned from other sources.

\subsection{Computing Facilities in South America}
{\bf Ehud Reiter} tells us what he did on his holidays:

I spent November - January in South America, mostly on holiday but I did
visit some university computer science departments.  The quality was quite
variable.  The best place I visited was the University of the Andes
(Venezuela), where there were rooms full of Sparcstations and the
faculty published regularly 
in international journals.  The worst was the Central University of Quito
(Ecuador), which had lots of tear gas (gassings at least once a week), but
very few computers (20 PC's and two old 16-bit Data General minis for 70,000
students).  The quality of the graduates reflected the availability of
resources: University of the Andes computer students seemed to usually know
what they were doing, while Central University of Quito students typically
had to be retrained from scratch after graduation if they got a computer job.
I suspect, unfortunately, that most South American universities are a lot
closer to the Central University of Quito than the University of the Andes.

It did seem to me that there was a substantial amount of demand for computers
and computer skilled people, mostly for standard business applications
(spreadsheets, word processing, accounts, etc) on PC's or large IBM machines.
I did not run across any AI applications.  Among the biggest problems faced
by potential computer users was high prices (higher than in the UK) and
the lack of manuals or how-to books in Spanish.

\subsection{ Mail connections to Africa}
The following correspondence took place the net newsgroup
soc.culture.african, and is of varying ages and veracity.  I thought
some of you might be interested in this aspect of computing in
developing countries.

{\bf John Pearce}, of Silma, Inc. Cupertino CA originally asked the
following:\\
``I will be leaving for Zimbabwe on a Fulbright in January
to teach Computer Science at the University in Harare.
Naturally, the University suffers from lack of equipment,
expertise, and ability to import equipment due to currency
controls. I would like to remedy this situation somewhat by
trying to set up a simple network of PCs with a file server
and possibly an Internet connection. I would appreciate
hearing suggestions from anyone about
 
1. computer networks or hosts anywhere in Southern Africa \\
2. corporations that might be interested in donating equipment\\
3. computer companies that do business in the region''

{\bf Philip Machanick} of the Computer Science Department, Stanford
University replied:\\
``It is possible to reach South African universities via fidonet.
A better setup is being worked on. At the moment, you may find
political obstacles to dealing with SA universities from Zimbabwe,
but this could change while you are there. If you are interested,
e-mail me and I'll suppply you with addresses/phone numbers of people
who'll be interested in your work in Zimbabwe.
I heard a rumour at one point that Apple was thinking of the
possibility of donations of used (but not yet obsolete) computers
to the third world. Just a rumour - it won't hurt to ask.
Here's one contact: Pan Africa Computer Services, the Apple
distributor for Southern Africa. Philip Noyce can be reached at
(508) 256-6887.
You may also want to look out for a copy of Computers in Africa,
published by Africa File, 21 Mill Lane, London, NW 6 1NT, UK (phone
London 794-5308). Subscriptions were 32 pounds a year when I last checked.
I don't have the most recent issue with me, but they have a fair number
of ads, which gives you an idea of who sells in Zimbabwe.
I would check to see if you can bring the equipment with you, rather than
getting it there. I have an idea that a scholarship like Fulbright entitles
you to bring in equipment duty free (?). Prices in Zimbabwe were extrememly 
inflated when I last investigated this issue, owing to restrictions on foreign
currency.''

{\bf F.F. Jacot Guillarmod} of Rhodes University, Grahamstown RSA responded:\\
``I would like to point out that the newsgroup 'soc.culture.african' is
being received and distributed to the following academic sites in
South Africa: Rhodes University Grahamstown, University of Cape Town,
University of Natal Durban. 
As soon as a dedicated TCP/IP data line is in place, the University of
Fort Hare in Alice will also be receiving netnews.  More distant is a
dialup uucp link to the University of Zimbabwe in Harare (early 1991),
and a dial up uucp link to the University of Namibia in Windhoek.''

\subsection{Editor's comment}
The note
about dialup links to Zimbabwe is of interest as in the latest
`African Technology Forum' Anuradha Vedantham writes that ``Since the
government outlaws modems as a national security measure, no
electronic mail contact is possible from Zimbabwe''. I understood that
the University of Zimbabwe at one point had a modem link in its
medical library, to access overseas library databases, but that this
was discontinued bbecause of cost. Maybe someone in the know can
enlighten us as to the truth of the matter. It would be 
interesting to have further information as to the extent of the
electronic mail network around the globe, especially to developing
countries. Opening channels of quick and effective communications is
one of the most successful ways of initiating technology transfer.

\section{Next meeting:}
Please send suggestions for future meetings to KK. Unfortunately we
have no finances to pay for visiting speakers, but if you know of
anyone interesting who'll be in town (or within the sort of vicinity
that it wouldn't bankrupt us to get them here) and would like to speak
to the group at some point, please get in touch. Perhaps {\em you}
would like to give a short talk...

Dan Walker's talk on March 7th has been postponed. The next meeting
will be:

On {\bf Thursday May 2nd}
at 5.30 pm in Room F10, AI Dept, 80 South Bridge,
Teresa Anderson will speak on {\em ``AI for Rural Electrification in Nepal:
Experiences with laptops 14000 feet up in the Annapurnas''}  and give
a software demonstration. 

\section{Journals and Literature Additions}
 Another journal that might be of interest to readers is:
 {\bf  VITA (Voulunteers in technical Assistance) News}, although
 rumour (ER) has it that his has been electronicised. Does anyone have
 further information?

The following new items have come into our
possession. Contact ER if you'd like to look at them.
\begin{enumerate}
\item African Technology Forum, volume 3 number 3 containing a short
article about the AI for Development Group, December 1990.
\item `The Development of Information Systems in the Third World', MSc
Thesis by R. A. Felts from LSE Systems Research Group, 1986/7.
\item `Health Informatics in Developing Countries', PhD thesis by Dayo
Forster from LSE Department of Information Systems, 1990.
\item `AI Research in China: a review', by Xinsong, J.
and Guoning, S. and Yu, C, from Proceedings of IJCAI 1983
\end{enumerate}
\section{Contact Personnel}
\begin{tabular}{l|l}
{\bf Software library:} & {\bf Newsletter, overall co-ordination,
meetings:}\\ 
Howard Beck &  Kathleen King\\
Artificial Intelligence Applications Institute &  Department of
Artificial Intelligence \\
University of Edinburgh & University of Edinburgh\\
80 South Bridge & 80 South Bridge\\
Edinburgh EH1 1HN & Edinburgh EH1 1HN\\
031-225-7774 x208 & 031 225 7774 x236\\
hab@uk.ac.ed.aiai & kk@uk.ac.ed.aipna\\
\hline
{\bf Addresses and contacts, funding:} & {\bf Distribution of
information, newsgroup etc:}\\ 
Robert Muetzelfeldt & Olly Morgan\\
Department of Forestry and Natural Resources &  School of Agriculture\\
University of Edinburgh & University of Edinburgh\\
Kings Buildings & Kings Buildings\\
 Mayfield Road &  West Mains Road\\
Edinburgh   EH9 3JU & Edinburgh   EH9 3JU\\
 031 667 1081 & 031 667 1081\\
 R.Muetzelfeldt@uk.ac.edinburgh &  O.Morgan@uk.ac.ed\\
\hline
{\bf Literature resource and bibliography:} &\\
 Ehud Reiter &\\
 Department of Artificial Intelligence &\\
 University of Edinburgh &\\
 80 South Bridge &\\
 Edinburgh EH1 1HN &\\
 031 225 7774 x294 &\\
reiter@uk.ac.ed.aipna &


\end{tabular}

\end{document}




